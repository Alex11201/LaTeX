\documentclass[12pt]{article} 

%Формат файла
\usepackage[paperheight=297mm,
   paperwidth=210mm,
   top=20mm,
   bottom=20mm,
   left=15mm,
   right=15mm]{geometry}


%Текст
\usepackage[fontsize=12pt]{fontsize}
\usepackage[english, russian]{babel}
\usepackage[T2A]{fontenc}
\usepackage{color}
\usepackage{transparent}
\usepackage{amsthm}
\parindent=0cm

\theoremstyle{definition}
\newtheorem{theorem}{Теорема}[section]
\newtheorem{lemma}[theorem]{Лемма}
\newtheorem{definition}{Определение}
\newtheorem{axiom}{Аксиома}
\newtheorem{statement}[theorem]{Утверждение}
\newtheorem{consequence}{Следствие}[subsection]
\renewcommand\qedsymbol{$\blacksquare$}

%Картинки
\usepackage{graphicx}
\usepackage{wrapfig}
\usepackage{subcaption}
\usepackage{tikz}
\usepackage{tkz-euclide}
\usepackage{pgfplots}
\usetikzlibrary {arrows.meta}
\usetikzlibrary{calc}
\usetikzlibrary{through}
\usetikzlibrary{intersections}
\usetikzlibrary{decorations.markings}
\usetikzlibrary{positioning}
\usetikzlibrary {3d}


%Математика
\usepackage{amsmath}
\usepackage{amsfonts}
\usepackage{amssymb}


%Всякое
\usepackage{relsize}
\usepackage{enumerate}
\usepackage[inline]{enumitem}
\usepackage{hyperref}

%Мат команды
\newcommand{\N}{\mathbb{N}}
\newcommand{\Z}{\mathbb{Z}}
\newcommand{\Q}{\mathbb{Q}}
\newcommand{\R}{\mathbb{R}}

%Оглавление
\title{\textbf{Тестовый файл}}\date{}\author{}

\hypersetup{
    colorlinks,
    citecolor=black,
    filecolor=black,
    linkcolor=black,
    urlcolor=black
}

\providecommand{\dotdiv}{% Don't redefine it if available
  \mathbin{% We want a binary operation
    \vphantom{+}% The same height as a plus or minus
    \text{% Change size in sub/superscripts
      \mathsurround=0pt % To be on the safe side
      \ooalign{% Superimpose the two symbols
        \noalign{\kern-.25ex}% but the dot is raised a bit
        \hidewidth$\smash{\cdot}$\hidewidth\cr % Dot
        \noalign{\kern.5ex}% Backup for vertical alignment
        $-$\cr % Minus
      }%
    }%
  }%
}

\pgfplotsset{compat=1.18}
\begin{document}

\maketitle
\tableofcontents
\label{toc}
\newpage

\section{Стереометрия}

\begin{theorem}
    Линии пересечения двух параллельных плоскостей третьей плоскостью параллельны:
    $$\alpha\parallel\beta,\,\gamma\cap\alpha=a,\,\gamma\cap\beta=b\Longrightarrow a\parallel b$$
\end{theorem}
\begin{proof}
    $ $\newline
    \begin{center}
        \begin{tikzpicture}[z={(0:10mm)},x={(30:5mm)}]
            \begin{scope}[canvas is zx plane at y=2]
                \draw[opacity=0.2] (0,-3) grid (7,3);
            \end{scope}
            \begin{scope}[canvas is zx plane at y=0]
                \draw[opacity=0.2] (0,-3) grid (7,3);
            \end{scope}
            \begin{scope}[plane origin={(0,0,3)}, plane x={(1,0,3)},
                plane y={(0,1,3.5)},
                canvas is plane]
                \fill[color=white] (-3,-1) rectangle (3,3);
                \draw[opacity=0.2] (-3,-1) grid (3,3);
            \end{scope}
            \begin{scope}[canvas is zx plane at y=2]
                \fill[color=white] (4,-3) rectangle (7,3);
                \draw[opacity=0.2] (4,-3) grid (7,3);
                \draw (4,-3) -- (4,3);
                \draw (4,0) node[above] {$a$};
                \draw (1,-3) -- (1,3);
                \draw (1,0) node[above] {$c$};
                \draw (8,-0.5) node {$\alpha$};
                \draw (4.5,3.3) node[above] {$\gamma$};
            \end{scope}
            \begin{scope}[canvas is zx plane at y=0]
                \fill[color=white] (3,-3) rectangle (7,3);
                \draw[opacity=0.2] (3,-3) grid (7,3);
                \draw (3,-3) -- (3,3);
                \draw (3,0) node[above] {$b$};
                \draw (8,-0.5) node {$\beta$};
            \end{scope}
            \foreach \point in {}{
    \fill \point circle (1.8pt);
}
          \end{tikzpicture}
    \end{center}
    \begin{center}
        $\alpha\parallel\beta\Longrightarrow\exists\,c\subset\alpha:\,\,c\parallel b.$ По теореме о крыше $a\parallel c,\,b\parallel c.$
    \end{center}
\end{proof}

\subsection{Сечения}


\begin{definition}
    Углом между наклонной и плоскостью называется угол между наклонной и её проекцией на данную плоскость. 
\end{definition}
\begin{theorem}
    Угол между наклонной к плоскости и её проекцией на эту плоскость есть наименьший из углов между наклонной и каждой прямой, лежащей в этой плоскости.
\end{theorem}
\begin{theorem}
    Пусть прямая $a$ образует с плоскостью $\pi$ угол $\alpha$. Прямая $b\subset\pi$ образует с прямой $a$ угол $\varphi$, а с её проекцией на плоскость $\pi$ – угол $\beta$. Тогда $\cos\varphi=\cos\alpha\cdot\cos\beta$.
\end{theorem}
\begin{proof}
    $ $\newline
        \begin{center}
            \begin{tikzpicture}[z={(0:10mm)},x={(40:5mm)}]
                \begin{scope}[canvas is zx plane at y=0,fill=red]
                    \draw[opacity=0.2] (0,-3) grid (8,3);
                    \draw (8.5,-0.5) node {$\pi$};
                \end{scope}
                \coordinate (O) at (-1,0,2);
                \coordinate (A) at (2,2,5);
                \coordinate (L) at (2,0,5);
                \coordinate (B) at (-1,0,5);
                \draw ($(A)!-0.1!(O)$) -- (O);
                \draw[dashed] (O) -- ($(A)!1.215!(O)$);
                \draw ($(A)!1.215!(O)$) -- ($(A)!1.5!(O)$);
                \draw ($(O)!-0.66!(L)$) -- ($(O)!1.33!(L)$);
                \tkzMarkAngle[size=1.35cm](L,O,A);
                \tkzMarkAngle[size=1.25cm](L,O,A);
                \tkzMarkAngle[size=1.15cm](L,O,A);
                \tkzMarkAngle[size=1.65cm](B,O,L);
                \tkzMarkAngle[size=1.55cm](B,O,L);
                \tkzMarkAngle[size=0.45cm](B,O,A);
                \draw (0,0.4,3.2) node {$\alpha$};
                \draw (-0.9,0.2,3.9) node {$\beta$};
                \draw (-1,0.2,2.8) node {$\varphi$};
                \draw (-1,0,0) -- (-1,0,8);
                \draw (A) -- (L) -- (B);
                \draw (B) ++(0.5,0,0) -- ++(0,0,0.5) -- ++(-0.5,0,0);
                \draw (L) ++(0,0.3,0) -- ++(-0.21,0,-0.21) -- ++(0,-0.3,0);
                \draw (A) node[above left] {$A$};
                \draw (O) node[above] {$O$};
                \draw (B) node[below] {$B$};
                \draw (L) node[below right] {$L$};
                \draw ($(A)!.5!(O)$) node[above left] {$a$};
                \draw ($(O)!.65!(L)$) node[above] {$l$};
                \draw ($(B)!-0.5!(O)$) node[below] {$b$};
                \foreach \point in {(A),(B),(O),(L)}{
        \fill \point circle (1.8pt);
    }
              \end{tikzpicture}
        \end{center}
        Пусть $A$ – произвольная точка на прямой $a$, $O$ – точка пересечения прямой $a$ с $\pi$, $L$ – основание перпендикуляра из $A$ на $\pi$, а $B$ – основание перпендикуляра из $L$ на $b$. Без ограничения общности положим $OA=1$, тогда $OL=\cos\alpha$, а $OB=\cos\alpha\cdot\cos\beta$. Пусть $\delta=(ABL)$, тогда $BL\perp b$ по построению, $AL\perp b$, так как $b\subset\pi$, $AL\perp\pi\Longrightarrow b\perp\delta\Longrightarrow b\perp AB$. Отсюда $OB=\cos\varphi=\cos\alpha\cdot\cos\beta$.
\end{proof}
\begin{definition}
    Если среди всех расстояний между точками, одна из которых принадлежит фигуре $\Phi_1$, а другая – фигуре $\Phi_2$, существует наименьшее, то его называют между фигурами $\Phi_1$ и $\Phi_2$.
\end{definition}
\begin{theorem}
    Расстоянием от точки до плоскости является расстояние от данной точки до её проекции на данную плоскость.
\end{theorem}
\begin{definition}
    Общим перпендикуляром двух скрещивающихся прямых называется отрезок, концы которого лежат на данных прямых, перпендикулярный к ним.
\end{definition}
\begin{theorem}
    Общий перпендикуляр двух скрещивающихся прямых существует и единственен.
\end{theorem}
\begin{theorem}
    Расстояние между двумя скрещивающимися прямыми равно расстоянию от точки пересечения одной из этих прямых с перпендикулярной ей плоскостью до проекции другой прямой на эту плоскость.
\end{theorem}
\end{document}