\documentclass[12pt]{article} 

%Формат файла
\usepackage[paperheight=297mm,
   paperwidth=210mm,
   top=20mm,
   bottom=20mm,
   left=15mm,
   right=15mm]{geometry}


%Текст
\usepackage[fontsize=12pt]{fontsize}
\usepackage[english, russian]{babel}
\usepackage[T2A]{fontenc}
\usepackage{color}
\usepackage{transparent}
\usepackage{amsthm}
\parindent=0cm

\theoremstyle{definition}
\newtheorem{theorem}{Теорема}[section]
\newtheorem{lemma}[theorem]{Лемма}
\newtheorem{definition}{Определение}
\newtheorem{axiom}{Аксиома}
\newtheorem{statement}[theorem]{Утверждение}
\newtheorem{consequence}{Следствие}[subsection]
\renewcommand\qedsymbol{$\blacksquare$}

%Картинки
\usepackage{graphicx}
\usepackage{wrapfig}
\usepackage{subcaption}
\usepackage{tikz}
\usepackage{tkz-euclide}
\usepackage{pgfplots}
\usetikzlibrary {arrows.meta}
\usetikzlibrary{calc}
\usetikzlibrary{through}
\usetikzlibrary{intersections}
\usetikzlibrary{decorations.markings}
\usetikzlibrary{positioning}
\usetikzlibrary {3d}


%Математика
\usepackage{amsmath}
\usepackage{amsfonts}
\usepackage{amssymb}


%Всякое
\usepackage{relsize}
\usepackage{enumerate}
\usepackage[inline]{enumitem}
\usepackage{hyperref}

%Мат команды
\newcommand{\N}{\mathbb{N}}
\newcommand{\Z}{\mathbb{Z}}
\newcommand{\Q}{\mathbb{Q}}
\newcommand{\R}{\mathbb{R}}

%Оглавление
\title{\textbf{Тестовый файл}}\date{}\author{}

\hypersetup{
    colorlinks,
    citecolor=black,
    filecolor=black,
    linkcolor=black,
    urlcolor=black
}

\providecommand{\dotdiv}{% Don't redefine it if available
  \mathbin{% We want a binary operation
    \vphantom{+}% The same height as a plus or minus
    \text{% Change size in sub/superscripts
      \mathsurround=0pt % To be on the safe side
      \ooalign{% Superimpose the two symbols
        \noalign{\kern-.25ex}% but the dot is raised a bit
        \hidewidth$\smash{\cdot}$\hidewidth\cr % Dot
        \noalign{\kern.5ex}% Backup for vertical alignment
        $-$\cr % Minus
      }%
    }%
  }%
}

\pgfplotsset{compat=1.18}
\begin{document}

\maketitle
\tableofcontents
\label{toc}
\newpage





\section{Стереометрия}

\subsection{Введение}

\begin{axiom}
    В $\R^3$ существуют плоскости, причем для любой из них выполняются аксиомы планиметрии.
\end{axiom}
\begin{axiom}[Аксиома плоскости]
    Через любые три неколлинеарные точки пространства проходит плоскость, при чем только одна.
\end{axiom}
\begin{axiom}
    Прямая, проходящая через две точки плоскости полностью лежит в данной плоскости.
\end{axiom}
\begin{axiom}[Аксиома пересечения плоскостей]
    Если две плоскости имеют общую точку, то они пересекаются по прямой:
    $$M \in \alpha;\,M \in \beta \Longrightarrow \exists\,\,l:\,\,l\subset \alpha;\,l\subset \beta$$
\end{axiom}
\begin{axiom}[Аксиома расстояния]
    В любой из плоскостей, проходящих через две различные точки расстояние между этими точками одно и то же:
    $$A \neq B;\,\,\forall\,\alpha:\,\,A,\,B\in\alpha\,\,\rho_\alpha(A;\,\,B)=\text{const}$$
\end{axiom}
\begin{definition}
    Прямая и плоскость называются пересекающимися, если они имеют одну общую точку.
\end{definition}
\begin{definition}
    Две плоскости называются пересекающимися, если они имеют одну общую прямую.
\end{definition}
\begin{definition}
    Прямые, лежащие в одной плоскости и не имеющие общих точек называются параллельными.
\end{definition}

\subsection{Следствия из аксиом}

\begin{lemma}
    Через прямую и точку, не лежащую на ней, проходит плоскость, при том только одна:
    $$\forall\,A;\,\,\forall\,l:\,\,A\notin l\,\,\exists!\,\,\alpha:\,\,l\subset\alpha;\,\,A\in\alpha$$
\end{lemma}

\begin{proof}
    $ $\newline
    \begin{center}
        \begin{tikzpicture}[z={(0:10mm)},x={(50:5mm)}]
            \begin{scope}[canvas is zy plane at x=0,fill=blue]
            \end{scope}
            \begin{scope}[canvas is zx plane at y=0,fill=red]
                \coordinate (A) at (7,-1);
                \coordinate (D) at (0,-2);
                \coordinate (E) at (6,3);
                \coordinate (B) at ($(D)!0.25!(E)$);
                \coordinate (C) at ($(D)!0.75!(E)$);
                \draw ($(D)!0.5!(E)$) node[above] {$l$};
                \draw (A) node[above] {$A$};
                \draw (B) node[above] {$B$};
                \draw (C) node[above] {$C$};
              \draw[opacity=0.3] (0,-3) grid (8,3);
              \draw (D) -- (E);
            \end{scope}
            \foreach \point in {(A), (B), (C)}{
    \fill \point circle (1.8pt);
}
          \end{tikzpicture}
    \end{center}
    Рассмотрим $B,\,C\in l:\,\,l\subset\alpha\Longrightarrow B,\,C\in\alpha.$ По аксиоме плоскости $\exists!\,\,\alpha:\,\,A,\,B,\,C\in\alpha;\,\,l\subset\alpha.$
\end{proof}

\begin{lemma}
    Через две пересекающиеся прямые проходит плоскость, при том только одна:
    $$a\cap b = C\Longrightarrow\exists!\,\,\alpha:\,\,a,\,b\subset\alpha$$
\end{lemma}
\begin{lemma}
    Через две параллельные прямые проходит плоскость, при том только одна:
    $$a\parallel b\Longrightarrow\exists!\,\,\alpha:\,\,a,\,b\subset\alpha$$
\end{lemma}

\subsection{Скрещивающиеся прямые}

\begin{definition}
    Две прямые называются скрещивающимися, если у них нет общих точек и они не параллельны.
\end{definition}

\begin{theorem}[Признак скрещивающихся]
    Если одна прямая лежит в плоскости, а другая пересекают данную плоскость в точке, не лежащей на первой прямой, данные прямые скрещиваются:
    $$a\subset\alpha,\,b\cap\alpha=M:\,\,M\notin a\Longrightarrow a\dotdiv b$$
\end{theorem}
\begin{proof}
    $ $\newline
    \begin{center}
        \begin{tikzpicture}[z={(0:10mm)},x={(40:5mm)}]
            \begin{scope}[canvas is zy plane at x=0,fill=blue]
                \coordinate (M) at (3,0);
                \draw (M) node[above right] {$M$};
                \draw ($(0,1.5)!0.5!(3,0)$) node[above] {$b$};
                \draw (0,1.5) -- (3,0);
                \draw[dashed] (0,1.5) -- ($(0,1.5)!1.65!(3,0)$);
                \draw ($(0,1.5)!1.65!(3,0)$) -- ($(0,1.5)!2!(3,0)$);
            \end{scope}
            \begin{scope}[canvas is zx plane at y=0,fill=red]
                \draw ($(5,-3)!0.5!(6,3)$) node[above left] {$a$};
                \draw (5,-3) -- (6,3);
                \draw[opacity=0.2] (0,-3) grid (8,3);
            \end{scope}
            \foreach \point in {(M)}{
    \fill \point circle (1.8pt);
}
          \end{tikzpicture}
    \end{center}
    Пусть $a$ и $b$ не скрещиваются. Тогда $a\parallel b$ или $a\cap b\neq\varnothing:
    $
    \begin{align*}
        1.\,\,&a\parallel b\Longrightarrow\exists!\,\,\beta:\,\,a,\,b\subset\beta\Longrightarrow M\in\beta,\text{ при этом }a\subset\alpha\text{ (по условию); }\\
        &a\subset\beta\text{ (по предложению)}\Longrightarrow \alpha\cap\beta=a,\,M\in\alpha,\,M\in\beta\Longrightarrow M\in a,\text{ противоречие.}\\
        2.\,\,&a\cap\beta\neq\varnothing\Longrightarrow\exists\,k:\,\,k\in a,\,k\in b\Longrightarrow\exists\,\beta:\,\,a,\,b\subset\beta\Longrightarrow M\in\beta,\text{ при этом }a\subset\alpha\text{ (по условию); }\\
        &a\subset\beta\text{ (по предложению)}\Longrightarrow \alpha\cap\beta=a,\,M\in\alpha,\,M\in\beta\Longrightarrow M\in a,\text{ противоречие.}
    \end{align*}
\end{proof}
\begin{theorem}
    Пусть $a\parallel b;\,\,a\cap\alpha\neq\varnothing,$ тогда $b\cap\alpha\neq\varnothing$.
\end{theorem}
\begin{proof}
    $ $\newline
    \begin{center}
        \begin{tikzpicture}[z={(0:10mm)},x={(40:5mm)}]
            \begin{scope}[canvas is zy plane at x=0,fill=blue]
                \coordinate (A) at (2,0);
                \draw (A) node[above right] {$A$};
                \draw (-1,1.5) -- (A);
                \draw ($(-1,1.5)!0.35!(A)$) node[above right] {$a$};
                \draw[dashed] (-1,1.5) -- ($(-1,1.5)!1.65!(A)$);
                \draw ($(-1,1.5)!1.65!(A)$) -- ($(-1,1.5)!2!(A)$);
            \end{scope}
            \begin{scope}[canvas is zy plane at x=1,fill=blue]
                \coordinate (B) at (6,0);
                \draw (B) node[above right] {$B$};
                \draw (3,1.5) -- (B);
                \draw ($(3,1.5)!0.35!(B)$) node[above right] {$b$};
                \draw[dashed] (3,1.5) -- ($(3,1.5)!1.65!(B)$);
                \draw ($(3,1.5)!1.4!(B)$) -- ($(3,1.5)!2!(B)$);
            \end{scope}
            \begin{scope}[canvas is zx plane at y=0,fill=red]
                \draw ($(2,0)!-.5!(6,1)$) -- ($(2,0)!1.5!(6,1)$);
                \draw ($(2,0)!0.5!(6,1)$) node[above] {$l$};
                \draw[opacity=0.2] (0,-3) grid (8,3);
            \end{scope}
            \foreach \point in {(A), (B)}{
    \fill \point circle (1.8pt);
}
          \end{tikzpicture}
    \end{center}
    $$a\parallel b\Longrightarrow\exists!\,\,\beta:\,\,a,\,b\subset\beta,\,a\cap\alpha=A\Longrightarrow A\in\beta\Longrightarrow\exists\,l:\,\,l=\alpha\cap\beta$$
    $$l\cap a=A,\,a\parallel\beta\Longrightarrow l\cap\beta=B:\,\,B\in b,\,B\in l\Longrightarrow b\cap\alpha=B.$$
\end{proof}

\subsection{Параллельность прямой и плоскости}

\begin{definition}
    Прямая и плоскость называются параллельными, если они не имеют общих точек.
\end{definition}

\begin{theorem}[Признак параллельности прямой и плоскости]
    Если прямая, не лежащая в плоскости, параллельна какой-либо прямой, лежащей в этой плоскости, то эти прямая и плоскость параллельны:
    $$a\subset\alpha,\,b \not\subset \alpha,\,a\parallel b\Longrightarrow \alpha\parallel b$$
\end{theorem}

\begin{proof}
    $ $\newline
    \begin{center}
        \begin{tikzpicture}[z={(0:10mm)},x={(25:5mm)},y={(60:7mm)}]
            \begin{scope}[canvas is zx plane at y=0]
                \draw[opacity=0.2] (0,-3) grid (8,3);
                \draw (-0.5,-1.5) node[left] {$\alpha$};
            \end{scope}
            \begin{scope}[canvas is zy plane at x=0]
                \fill[color=white] (0,0) rectangle (8,4);
                \draw[opacity=0.2] (0,-2) grid (8,4);
                \coordinate (A1) at (0,0);
                \coordinate (A2) at (8,0);
                \coordinate (B1) at (0,3);
                \coordinate (B2) at (8,3);
                \draw (A1) -- (A2);
                \draw (B1) -- (B2);
                \draw ($(A1)!0.5!(A2)$) node[above] {$a$};
                \draw ($(B1)!0.5!(B2)$) node[above] {$b$};
                \draw (-0.25,2) node[left] {$\beta$};
            \end{scope}
            \begin{scope}[canvas is zx plane at y=0]
                \fill[color=white] (0,-3) rectangle (8,0);
                \draw[opacity=0.2] (0,-3) grid (8,0);
            \end{scope}
            \foreach \point in {}{
    \fill \point circle (1.8pt);
}
          \end{tikzpicture}
    \end{center}
    \begin{center}
        $a\parallel b\Longrightarrow\exists!\,\,\beta:\,\,a,\,b\subset\beta$. Пусть $b\cap\alpha\neq\varnothing$. Но $\alpha\cap\beta=a$. Противоречие.
    \end{center}
\end{proof}
\begin{theorem}[О линии пересечения плоскостей]
    Если плоскость проходит через прямую, параллельную другой плоскости, и пересекает эту плоскость, то линия пересечения плоскостей параллельна данной прямой:
    $$a\parallel \alpha,\,a\subset\beta,\,\alpha\cap\beta=b\Longrightarrow a\parallel b$$
\end{theorem}
\begin{proof}
    $ $\newline
    \begin{center}
        \begin{tikzpicture}[z={(0:10mm)},x={(25:5mm)},y={(60:7mm)}]
            \begin{scope}[canvas is zx plane at y=0]
                \draw[opacity=0.2] (0,-3) grid (8,3);
                \draw (-0.5,-1.5) node[left] {$\alpha$};
            \end{scope}
            \begin{scope}[canvas is zy plane at x=0]
                \fill[color=white] (0,0) rectangle (8,4);
                \draw[opacity=0.2] (0,-2) grid (8,4);
                \coordinate (A1) at (0,0);
                \coordinate (A2) at (8,0);
                \coordinate (B1) at (0,3);
                \coordinate (B2) at (8,3);
                \draw (A1) -- (A2);
                \draw (B1) -- (B2);
                \draw ($(A1)!0.5!(A2)$) node[above] {$b$};
                \draw ($(B1)!0.5!(B2)$) node[above] {$a$};
                \draw (-0.25,2) node[left] {$\beta$};
            \end{scope}
            \begin{scope}[canvas is zx plane at y=0]
                \fill[color=white] (0,-3) rectangle (8,0);
                \draw[opacity=0.2] (0,-3) grid (8,0);
            \end{scope}
            \foreach \point in {}{
    \fill \point circle (1.8pt);
}
          \end{tikzpicture}
    \end{center}
    \begin{center}
        $a\parallel\alpha\Longrightarrow a\cap b=\varnothing.$ Пусть $a\dotdiv b$, тогда $a\cap\alpha\neq\varnothing$, но $a\parallel \alpha$. Противоречие.
    \end{center}
\end{proof}

\begin{theorem}[О крыше]
    Если через каждую из двух параллельных прямых проведена плоскость, причём эти плоскости пересекаются, то линия их пересечения параллельна каждой из данных прямых:
    $$a\parallel b,\,a\subset \alpha,\,b\subset \beta,\,\alpha\cap\beta=c\Longrightarrow a\parallel c,\, b\parallel c$$
\end{theorem}
\begin{proof}
    $ $\newline
    \begin{center}
        \begin{tikzpicture}[z={(0:10mm)},x={(120:5mm)},y={(45:7mm)}]
            \begin{scope}[canvas is zx plane at y=0]
                \draw[opacity=0.2] (0,-3) grid (8,4);
                \coordinate (B1) at (0,-2);
                \coordinate (B2) at (8,-2);
                \draw (B1) -- (B2);
                \draw ($(B1)!0.77!(B2)$) node[above] {$b$};
                \draw (8.3,0) node[right] {$\beta$};
            \end{scope}
            \begin{scope}[canvas is zy plane at x=3]
                \fill[color=white] (0,0) rectangle (8,-6);
                \draw[opacity=0.2] (0,1) grid (8,-6);
                \coordinate (A1) at (0,-5);
                \coordinate (A2) at (8,-5);
                \draw (A1) -- (A2);
                \draw ($(A1)!0.5!(A2)$) node[above] {$a$};
                \draw (-0.3,-2) node[left] {$\alpha$};
            \end{scope}
            \begin{scope}[canvas is zx plane at y=0]
                \fill[color=white] (0,3) rectangle (8,4);
                \draw[opacity=0.2] (0,3) grid (8,4);
                \coordinate (C1) at (0,3);
                \coordinate (C2) at (8,3);
                \draw (C1) -- (C2);
                \draw ($(C1)!0.5!(C2)$) node[above] {$c$};
            \end{scope}
            \foreach \point in {}{
    \fill \point circle (1.8pt);
}
          \end{tikzpicture}
    \end{center}
    \begin{center}
        $a\parallel b,\,b\not\subset \alpha\Longrightarrow b\parallel \alpha\Longrightarrow b\parallel c\text{ (по теореме о линии пересечения плоскостей). Аналогично }c\parallel a.$
    \end{center}
\end{proof}
\setcounter{subsection}{8}

\begin{consequence}
    Параллельность прямых в пространстве транзитивна.
\end{consequence}
\begin{consequence}
    Если прямая параллельна каждой из двух пересекающихся плоскостей,
то она параллельна их линии пересечения.
\end{consequence}

\subsection{Параллельность плоскостей}

\begin{definition}
    Две плоскости называются параллельными, если они не имеют общих точек.
\end{definition}

\begin{theorem}[Признак параллельности плоскостей]
    Если две пересекающиеся прямые одной плоскости параллельны соответственно двум прямым другой плоскости, то эти плоскости параллельны:
    $$a,\,b\subset\alpha,\,a\cap b\neq\varnothing;\,\,c,\,d\subset\beta;\,\,a\parallel c,\,b\parallel d\Longrightarrow \alpha\parallel\beta$$
\end{theorem}
\begin{proof}
    $ $\newline
    \begin{center}
        \begin{tikzpicture}[z={(0:10mm)},x={(30:5mm)}]
            \begin{scope}[canvas is zx plane at y=2]
                \draw[opacity=0.2] (0,-3) grid (8,3);
                \draw (1,-3) -- (6,3);
                \draw (0,2) -- (8,-3);
                \draw (2,-0.8) node[left] {$a$};
                \draw (1.3,2) node {$b$};
                \draw (9,-0.5) node {$\alpha$};
            \end{scope}
            \begin{scope}[canvas is zx plane at y=0]
                \draw[opacity=0.2] (0,-3) grid (8,3);
                \draw (1,-3) -- (6,3);
                \draw (0,2) -- (8,-3);
                \draw (2,-0.8) node[left] {$c$};
                \draw (1.3,2) node {$d$};
                \draw (9,-0.5) node {$\beta$};
            \end{scope}
            \foreach \point in {}{
    \fill \point circle (1.8pt);
}
          \end{tikzpicture}
    \end{center}
    Пусть $\alpha\cap\beta\neq\varnothing$. Тогда по теореме о линии пересечения плоскостей $\alpha\cap\beta=k,\,k\parallel a\,\,(c\subset\beta);\,\,\alpha\cap\beta=l,\,l\parallel b\,\,(d\subset\beta).$ Противоречие. 
\end{proof}

\begin{theorem}
    Линии пересечения двух параллельных плоскостей третьей плоскостью параллельны:
    $$\alpha\parallel\beta,\,\gamma\cap\alpha=a,\,\gamma\cap\beta=b\Longrightarrow a\parallel b$$
\end{theorem}
\begin{proof}
    $ $\newline
    \begin{center}
        \begin{tikzpicture}[z={(0:10mm)},x={(30:5mm)}]
            \begin{scope}[canvas is zx plane at y=2]
                \draw[opacity=0.2] (0,-3) grid (7,3);
            \end{scope}
            \begin{scope}[canvas is zx plane at y=0]
                \draw[opacity=0.2] (0,-3) grid (7,3);
            \end{scope}
            \begin{scope}[plane origin={(0,0,3)}, plane x={(1,0,3)},
                plane y={(0,1,3.5)},
                canvas is plane]
                \fill[color=white] (-3,-1) rectangle (3,3);
                \draw[opacity=0.2] (-3,-1) grid (3,3);
            \end{scope}
            \begin{scope}[canvas is zx plane at y=2]
                \fill[color=white] (4,-3) rectangle (7,3);
                \draw[opacity=0.2] (4,-3) grid (7,3);
                \draw (4,-3) -- (4,3);
                \draw (4,0) node[above] {$a$};
                \draw (1,-3) -- (1,3);
                \draw (1,0) node[above] {$c$};
                \draw (8,-0.5) node {$\alpha$};
                \draw (4.5,3.3) node[above] {$\gamma$};
            \end{scope}
            \begin{scope}[canvas is zx plane at y=0]
                \fill[color=white] (3,-3) rectangle (7,3);
                \draw[opacity=0.2] (3,-3) grid (7,3);
                \draw (3,-3) -- (3,3);
                \draw (3,0) node[above] {$b$};
                \draw (8,-0.5) node {$\beta$};
            \end{scope}
            \foreach \point in {}{
    \fill \point circle (1.8pt);
}
          \end{tikzpicture}
    \end{center}
    \begin{center}
        $\alpha\parallel\beta\Longrightarrow\exists\,c\subset\alpha:\,\,c\parallel b.$ По теореме о крыше $a\parallel c,\,b\parallel c.$
    \end{center}
\end{proof}



\end{document}
