%Формат файла
\documentclass[12pt]{article} 
\usepackage[paperheight=297mm,
   paperwidth=210mm,
   top=20mm,
   bottom=20mm,
   left=15mm,
   right=15mm]{geometry}


%Текст
\usepackage[fontsize=12pt]{fontsize}
\usepackage[russian]{babel}
\usepackage{color}
\usepackage{transparent}
\usepackage{amsthm}
\parindent=0cm

\theoremstyle{definition}
\newtheorem{theorem}{Теорема}[section]
\newtheorem{lemma}[theorem]{Лемма}
\newtheorem{definition}{Определение}
\newtheorem{statement}[theorem]{Утверждение}
\newtheorem{consequence}{Следствие}[subsection]
\renewcommand\qedsymbol{$\blacksquare$}

%Картинки
\usepackage{graphicx}
\usepackage{wrapfig}
\usepackage{subcaption}
\usepackage{tikz}
\usepackage{tkz-euclide}
\usetikzlibrary {arrows.meta}
\usetikzlibrary{calc}
\usetikzlibrary{intersections}


%Математика
\usepackage{amsmath}
\usepackage{amsfonts}
\usepackage{mathabx}
\usepackage{amssymb}

%Всякое
\usepackage{relsize}
\usepackage{enumerate}
\usepackage[inline]{enumitem}
\usepackage{hyperref}

%Мат команды
\newcommand{\N}{\mathbb{N}}
\newcommand{\Z}{\mathbb{Z}}
\newcommand{\Q}{\mathbb{Q}}
\newcommand{\R}{\mathbb{R}}
\newcommand{\prob}{\mathbb{P}}
\newcommand{\verteq}{\rotatebox{90}{$\,=$}}
\newcommand{\equalto}[2]{\underset{\scriptstyle\overset{\mkern4mu\verteq}{#2}}{#1}}
\newcommand{\vertneq}{\rotatebox{90}{$\,\neq$}}
\newcommand{\notequalto}[2]{\underset{\scriptstyle\overset{\mkern4mu\vertneq}{#2}}{#1}}
\DeclareRobustCommand{\divby}{%
  \mathrel{\vbox{\baselineskip.65ex\lineskiplimit0pt\hbox{.}\hbox{.}\hbox{.}}}%
}
\makeatletter
\newenvironment{sqcases}{%
  \matrix@check\sqcases\env@sqcases
}{%
  \endarray\right.%
}
\def\env@sqcases{%
  \let\@ifnextchar\new@ifnextchar
  \left\lbrack
  \def\arraystretch{1.2}%
  \array{@{}l@{\quad}l@{}}%
}
\makeatother

%Оглавление
\title{\textbf{Алгебра}}\date{}

\hypersetup{
    colorlinks,
    citecolor=black,
    filecolor=black,
    linkcolor=black,
    urlcolor=black
}

\begin{document}

\maketitle
\tableofcontents
\label{toc}
\newpage

\section{Решение уравнений и неравенств.}

\subsection{Иррациональные уравнения}

\begin{align*}
&\sqrt{f(x)}=g(x)\Longleftrightarrow 
    \begin{cases}
        f(x)=(g(x))^2\\
        g(x)\geq0
    \end{cases} & &\sqrt{f(x)}=\sqrt{g(x)}\Longleftrightarrow
    \begin{cases}
        f(x)= g(x)\\
        f(x)\geq 0
    \end{cases}
\end{align*}

\subsection{Иррациональные неравенства}

$$\frac{f(x)}{g(x)}\leq 0\Longleftrightarrow \left[
\begin{gathered}
    \begin{cases}
        f(x)\geq 0\\
        g(x)<0
    \end{cases}\\
    \begin{cases}
        f(x)\leq 0\\
        g(x)>0
    \end{cases}
\end{gathered}
\right.$$

\begin{align*}
    &\sqrt{f(x)}>g(x) \Longleftrightarrow
    \begin{sqcases}
        \begin{cases}
            g(x)<0\\
            f(x)\geq 0
        \end{cases}\\
        \begin{cases}
            g(x)\geq 0\\
            f(x)>(g(x))^2
        \end{cases}
    \end{sqcases}& &\sqrt{f(x)}<g(x) \Longleftrightarrow
    \begin{cases}
        g(x)\geq 0\\
        f(x) \geq 0\\
        f(x) < (g(x))^2
    \end{cases}
\end{align*}

$$\sqrt{f(x)}>\sqrt{g(x)} \Longleftrightarrow 
    \begin{cases}
        f(x)>g(x)\\
        g(x)\geq 0
    \end{cases}$$

\subsection{Неравенства с модулем}

\begin{align*}
    &|f(x)|<a, \,\, a>0 \Longleftrightarrow 
    \begin{cases}
        f(x)>-a\\
        f(x)<a
    \end{cases}& 
    &|f(x)|>a \Longleftrightarrow
    \begin{sqcases}
        f(x)>a\\
        f(x)<-a
    \end{sqcases}
\end{align*}

\begin{align*}
    &|f(x)|\leq|g(x) \Longleftrightarrow (f(x)-g(x))\cdot(f(x)+g(x))\leq0\\
    &|f(x)|+|g(x)|>|f(x)+g(x)| \Longleftrightarrow f(x)\cdot g(x) < 0\\
    &|f(x)|+|g(x)|\leq|f(x)+g(x)| \Longleftrightarrow f(x)\cdot g(x) \geq 0
\end{align*}

\section{Многочлены}

\begin{definition}
    Многочленом от переменной $x$ над $K$ называется выражение вида: $f(x)=a_nx^n+a_{n-1}x^{n-1}+\ldots+a_1x+a_0,\, a_k\in K\text{ -- коэффициент многочлена},\, a_n\neq0$.
\end{definition}

\begin{definition}
    Наибольшее $k$ такое, что $a_k\neq0$, называется степенью многочлена $f$:
    \begin{align*}
        &k = deg\,f\\
        &a_0\text{ -- свободный член}\\
        &a_nx^n\text{ -- старший член}\\
        &a_n\text{ -- старший коэффициент}
    \end{align*}
\end{definition}

\begin{definition}
    Два многочлена называются равными, если их коэффициенты при соответственных степенях $x$ равны.
\end{definition}

\begin{definition}
    \begin{align*}
        &f(x)=a_nx^n+a_{n-1}x^{n-1}+\ldots+a_1x+a_0\\
        &g(x)=b_nx^n+b_{n-1}x^{n-1}+\ldots+b_1x+b_0
    \end{align*}
    Суммой многочленов $f(x)$ и $g(x)$ называется: $$n(x)=(a_n+b_n)x^n+(a_{n-1}+b_{n-1})x^{n-1}+\ldots+(a_1+b_1)x+(a_0+b_0)$$ Произведением многочленов $f(x)$ и $g(x)$ называется: $$S(x)=d_{2n}x^{2n}+d_{2n-1}x^{2n-1}+\ldots+d_1x+d_0\text{, где} d_k=a_0b_k+a_1b_{k-1}+\ldots+a_{k-1}b_1+a_kb_0=\sum_{i=0}^ka_ib_{k-i}$$.
\end{definition}

\begin{statement}
    Пусть $def\,f(x)\neq0,\,deg\,g(x)\neq0$, тогда:
    \begin{align*}
        &1.\,\,deg(f(x)+g(x))\leq \max {\{deg\,f,\,deg\,g\}}\\
        &2.\,\,f(x)\cdot g(x) \neq0\\
        &3.\,\,deg(f(x)\cdot g(x))=deg\,f(x)+deg\,g(x)
    \end{align*}
\end{statement}
\begin{proof}
    \begin{align*}
    1.\,\,&\text{Пусть }deg\,f=def\,g=n\\
    &f(x)+g(x)=(a_n+b_n)x^n+(a_{n-1}+b_{n-1})x^{n-1}+\ldots+(a_1+b_1)x+(a_0+b_0)\\
    &\text{Если }k>n \text{, то} a_k=0,\,b_k=0\text{, то есть }(a_k+b_k)=0\\\\
    &\text{Пусть }deg\,f=n,\,deg\,g=m,\,m<n\\
    &\text{Если }k>n\text{, то }a_k+b_k=0\text{, так как }b_{m+1}=b_{m+2}=\ldots=b_{n-1}=b_n=0\\
    &\text{Тогда }a_n+b_n=a_n\neq0\\\\
    2.\,\,&f(x)\cdot g(x)=\notequalto{a_nb_mx^{n+m}}{0}+\underbrace{\ldots\ldots}_{\text{степень}<n}
    \end{align*}
\end{proof}

\begin{definition}
    Многочлен $f(x)$ делится на многочлен $g(x)$, если существует такой многочлен $h(x)$, что $h(x)\cdot g(x)=f(x)$.
\end{definition}
\begin{statement}
    Свякий многочлен $f(x)\neq0$ делится на самого себя.
\end{statement}
\begin{statement}
    Если $f(x)$ делится на $g(x)$, а $g(x)$ делится на $f(x)$, то $f(x)=c\cdot g(x),\,c\in K$.
\end{statement}
\begin{definition}
    Число $x_0$ является корнем $f(x)$, если $f(x_0)=0$.
\end{definition}
\begin{theorem}[Теорема Безу]
    Остаток от деления многочлена $P(x)$ на двучлен $(x-a)$ равен $P(a)$.
\end{theorem}
\begin{proof}
    \begin{align*}
    &P(x)=(x-a)\cdot q(x)+r\\
    &P(a)=0\cdot p(x)+r=r
\end{align*}
\end{proof}
\setcounter{subsection}{4}
\begin{consequence}
    Число $a$ является корнем многочлена $P(x)$ тогда и только тогда, когда $P(x)$ делится на $(x-a)$.
\end{consequence}

\section{Множества}
\begin{definition}
    Множества равномощны, если между ними существует биекция.
\end{definition}
\begin{definition}
    Множества $A$ и $B$ называются равными, если $A\subseteq B,\,B\subseteq A.$
\end{definition}
\begin{definition}
    Множества, равномощные $\N$, называются счетными.
\end{definition}
\begin{definition}
    Декартовым произведением множеств $A$ и $B$ называется множество $A\times B=\{x\,|\,x=(a,b),\,a\in A,\,b\in B\}$.
\end{definition}
\begin{definition}
    Число $a$ называется числом кратности $k$ многочлена $f(x)$, если $f(x)$ делится на $(x-a)^k$, но не делится на $(x-a)^{k+1}$.
\end{definition}

\section{Числовые последовательности}
\begin{definition}
    Бесконечной числовой последовательностью $(a_n)$ называется отображение $\N \to \R$.
\end{definition}
\begin{definition}
    Конечной числовой последовательностью $(a_n)$ называется отображение\\ $a:\,\{1,\,2,\,\ldots,\,k\}\to \R$.
\end{definition}

\begin{definition}
    Множество $M,\,M\subset\R$ называется ограниченным сверху, если $\exists\, c:\,\forall x\in M:\,x \leq c$.
\end{definition}
\begin{definition}
    Множество $M,\,M\subset\R$ называется ограниченным снизу, если $\exists\, c:\,\forall x\in M:\,x \geq c$.
\end{definition}
\begin{definition}
    Множество $M,\,M\subset\R$ называется ограниченным, если оно ограничено сверху и снизу.
\end{definition}
\begin{definition}
    Последовательность $a_n$ называется ограниченной, если $a(\N)$ ограничено. 
\end{definition}
\begin{definition}
    Последовательность $a_n$ называется называется монотонно возрастающей, если $\forall n \in \N:\, a_{n+1}>a_n$.
\end{definition}
\begin{theorem}
    Пусть все элементы последовательности $a_n$ положительны. Последовательность $a_n$ возстает тогда и только тогда, когда $\dfrac{a_{n+1}}{a_n}>1$.
\end{theorem}

\subsection{Аксиоматика действительных чисел}

\begin{definition}
    Пусть $M \subset \R,\,M$ ограничено. Тогда наименьшая из верхних граней множества $M$ называется точной верхней гранью:
    $$a=\sup M\Longleftrightarrow \forall x \in M:\, x \leq a,\, \forall \varepsilon>0\,\,\, \exists\, x \in M:\, x>a-\varepsilon$$
\end{definition}
\begin{definition}
    Пусть $M \subset \R,\,M$ ограничено. Тогда наибольшая из нижних граней множества $M$ называется точной нижней гранью:
    $$a=\inf M\Longleftrightarrow \forall x \in M:\, x \geq a,\, \forall \varepsilon>0\,\,\, \exists\, x \in M:\, x<a+\varepsilon$$
\end{definition}

\begin{theorem}
    Пусть $a:\,\N \to \R,\, a(\N)$ ограничена. Тогда:
    $$\exists\, x_0\in \R:\, \forall \varepsilon >0:\, a^{-1}\equalto{((x_0-\varepsilon;\,x_0+\varepsilon))}{U_\varepsilon(x_0)}\text{ бесконечно}$$
\end{theorem}
\begin{proof}
    Если $\exists\, x_0:\,a^{-1}(x_0)$ бесконечно, то доказано. Если $\exists\, x_0:\,a^{-1}(x_0)$ конечно или пусто, то:
    \begin{center}
        \begin{tikzpicture}
            \draw[-{Stealth[scale = 1.5]}] (0,0) -- (10,0) node[right] {$x$};
            \coordinate (a0) at (2,0);
            \coordinate (b0) at (8,0);
            \coordinate (c0) at (5,0);
            \coordinate (c1) at (6.5,0);
            \draw (a0) node[below] {$a_0$};
            \draw (b0) node[below] {$b_0$};
            \draw (c0) node[below] {$c_0$};
            \draw (c1) node[below] {$c_1$};
            \draw [
                decoration={
                brace,
                mirror,
                raise=1.2cm
            },
            decorate
        ] (a0) -- (b0) 
        node [pos=0.5,anchor=north,yshift=-1.25cm] {$I_0$};
        \draw [
                decoration={
                brace,
                mirror,
                raise=0.5cm
            },
            decorate
        ] (c0) -- (b0) 
        node [pos=0.5,anchor=north,yshift=-0.55cm] {$I_1$}; 
            \foreach \point in {(a0), (b0), (c0), (c1)}{
    \fill \point circle (1.8pt);
            }
        \end{tikzpicture}
    \end{center}
    Отметим на числовой прямой $a_0=\inf a(\N)$ и $b_0=\sup a(\N)$, а также середину $a_0b_0$, то есть $c_0=\dfrac{a_0+b_0}{2}$. Разделим один из получившихся отрезков (отметим $c_0$ и $b_0$, как $a_1$ и $b_1$ соотвественно) пополам, получив $c_1=\dfrac{a_1+b_1}{2}$. Данный процесс можно продолжать, получая следующую конструкцию:
    $$[a_0;\,b_0]\supset [a_1;\,b_1]\supset \ldots \supset [a_n;\,b_n]\supset \dots$$
    Теперь необходимо доказать следующее:
    $$\bigcap_{n=0}^{\infty}[a_n;\,b_n]\neq \emptyset$$
    \begin{align*}
        1.\,\,&a_0\leq a_1 \leq \ldots \leq a_n \leq \ldots\\
        2.\,\,&a(\N)\text{ ограничена сверху } b_i\text{ элементом}\\
        3.\,\,&a(\N)\text{ имеет точную верхнюю грань } M_1=\sup a(\N)\\
        &\text{и точную нижнюю грань } M_2 = \inf a(\N)\\
        4.\,\,&M_1\leq M_2\\
        5.\,\,&[M_1;\,M_2]\subset \bigcap_{n=0}^{\infty}[a_n;\,b_n]\\
        6.\,\,&\text{Пусть } M_1<M_2\text{, тогда } \exists\, n:\, b_n-a_n<M_2-M_1.\\
        &\text{Получаем противоречие, значит } M_1=M_2=M.\\
        7.\,\,&\text{Возьмем такое } n \text{, что } b_n-a_n<\varepsilon. \text{ Тогда } [a_n;\,b_n]\subset U_{\varepsilon}(M).\\
        &\text{То есть }\forall \varepsilon>0:\, a^{-1}(U_{\varepsilon}(M))\text{ бесконечно.}
    \end{align*}
\end{proof}
\begin{definition}
    Число $x$ называется частичным пределом последовательности $a(\N)\to \R$, если $\forall \varepsilon>0:\, a^{-1}(U_{\varepsilon}(x))$ бесконечно.
\end{definition}

\subsection{Прогрессии}

\begin{definition}
    Арифметической прогрессией называется числовая последовательность, заданная формулой n-го члена:
    $$a_n=a_1+(n-1)\cdot d$$
\end{definition}

\begin{definition}
    Разностью арифметической прогрессии называется разность $a_{n+1}$ и $a_n$.
\end{definition}

\begin{statement}
    Пусть $(a_n)$ -- арифметическая прогрессия, тогда:
    $$a_{n+2}-a_{n+1}=a_{n+1}-a_n$$
\end{statement}

\begin{statement}
    Пусть $(a_n)$ -- арифметическая прогрессия, тогда:
    $$\forall n \in \N,\,n\geq 2 \,\,\,\forall k \in \N,\, k<n:\, a_n=\frac{a_{n+k}+a_{n-k}}{2}$$
\end{statement}

\begin{theorem}
    Сумма первых $n$ членов арифметической прогрессии равна:
    $$S_n=n\cdot \left(a_1+\frac{(n-1)\cdot d}{2} \right)=n \cdot \frac{a_1 + a_n}{2}$$
\end{theorem}

\begin{proof}
    \begin{align*}
        &S_n=a_1+a_2+a_3+\ldots+a_n=\\
        =&a_1+(a_1+d)+(a_1+2d)+\ldots+(a_1+(n-1)\cdot d)=\\
        =&n\cdot a_1+d\cdot \left( \frac{(n-1)\cdot n}{2} \right)=\\
        =&n\cdot \left(a_1+\frac{(n-1)\cdot d}{2} \right)=\\
        =&n \cdot \frac{a_1 + (a_1 + (n-1)\cdot d)}{2}=\\
        =&n\cdot \frac{a_1 + a_n}{2}
    \end{align*}
\end{proof}

\begin{definition}
    Арифметической прогрессией называется числовая последовательность, заданная формулой n-го члена:
    $$b_n=b_1 \cdot q^{n-1},\, b_1\neq0,\, q\neq0$$
\end{definition}

\begin{statement}
    Пусть $(b_n)$ -- геометрическая прогрессия, тогда: 
    $$\forall n\in \N,\, n\geq 2:\,b_n^2=b_{n-1}\cdot b_{n+1}$$
\end{statement}
\begin{proof}
    $$b_{n-1}\cdot b_{n+1}=b_1\cdot q^{n-2}\cdot b_1 \cdot q^n=b_1^2\cdot q^{2n-2}=(b_1\cdot q^{n-1})^2$$
\end{proof}
\begin{theorem}
    Сумма первых $n$ членов геометрической прогрессии равна:
    $$S_n=b_1\cdot \frac{1-q^n}{1-q},\, q\neq 1$$
\end{theorem}
\begin{proof}
    \begin{align*}
        &S_n=b_1+b_2+b_3+\ldots+b_n=\\
        =&b_1+b_1\cdot q+b_1\cdot q^2+\ldots+b_1\cdot q^{n-1}=\\
        =&b_1\cdot(1+q+q^2+\ldots+q^n)=\\
        =&b_1\cdot \frac{1-q^n}{1-q}
    \end{align*}
\end{proof}

\section{Эквивалентность и группы}

\begin{definition}
    Пусть $M$ -- множество, тогда множество $R \subset \{(a,\,b)\,|\,a,\,b\in M\}$ упорядоченных пар элементов $M$ называется бинарным отношением на $M$. 
\end{definition}

\begin{definition}
    Бинарное отношение называется отношением эквивалентности, если оно удовлетворяет свойствам:
    \begin{align*}
        1.\,\,&\text{Рефлексивность } a\sim a\\
        2.\,\,&\text{Симметричность } a\sim b \Longleftrightarrow b\sim a\\
        3.\,\,&\text{Транзитивность } a\sim b,\, b\sim c \Longleftrightarrow a\sim c
    \end{align*}
\end{definition}
\begin{theorem}[Малая теорема Ферма]
    $\forall n\in \N,\, p\in \prob:\, n^{p-1}\equiv_p 1$
\end{theorem}
\begin{definition}
    Бинарной операцией $\times$ на множестве $M$ называется отображение из множества упорядоченных пар $M^2=\{(a,\,b)\,|\,a,\,b\in M\}$ в множество $M$.
\end{definition}
\begin{definition}
    Пара $G(M;\,\times)$, $M$ -- множество, $\times$ -- бинарная операция, называется группой, если выполняются свойства:
    \begin{align*}
        1.\,\,&\forall a,\,b\in M:\,(a\times b)\in M\\
        2.\,\,&\exists\, e\in M\,\,\,\forall a\in M:\, e\times a=a\\
        3.\,\,&\forall a\in M\,\,\, \exists\, a^{-1}\in M:\, a\times a^{-1}=a^{-1}\times a=e\\
        4.\,\,&\forall a,\,b,\,c \in M:\,(a\times b)\times c=a\times(b\times c)= (a\times c)\times b
    \end{align*}
\end{definition}
\section{Пределы}
\begin{definition}
    Число $A$ называется пределом $(x_n)$, если:
    $$\forall \varepsilon >0\,\,\,\exists\,N\in\N:\,\forall n>N\,\,\,|x_n-A|<\varepsilon$$
\end{definition}
\begin{theorem}
    $$\lim_{n\to\infty}(x_n+y_n)=\lim_{n\to\infty}(x_n)+\lim_{n\to\infty}(y_n)$$
\end{theorem}
\begin{proof}
    Пусть $x_n\to a;\,y_n\to b$. По определению $N_a(\varepsilon):\,\forall n>N_a(\varepsilon)\,\,\,|x_n-a|<\varepsilon,\,N_b(\varepsilon):\,\forall n>N_b(\varepsilon)\,\,\,|y_n-b|<\varepsilon$. Рассмотрим $N_c(\varepsilon):\,\forall n>N_c(\varepsilon)\,\,\,|x_n+y_n-a-b|<\varepsilon:\,|x_n+y_n-a-b|\leq|x_n-a|+|y_n-b|\leq 2\varepsilon$ при $N_c=\max(N_a(\varepsilon);\,N_b(\varepsilon))$, то есть $2N_c(\varepsilon)$ -- это номер, с которого утверждение точно выполняется.
\end{proof}
\begin{theorem}[Теорема Вейерштрасса]
    Пусть $(x_n)$ монотонна, тогда:
    \begin{align*}
        1.\,\,&\text{Она имеет предел в }\bar{\R}=\R\cup\{-\infty;\,+\infty\}\\
        2.\,\,&\text{Если она ограничена, то она имеет вредел в }\R
    \end{align*}
\end{theorem}
\begin{proof}
    По определению монотонно возрастающей последовательности: $\forall n:\, x_{n+1}>x_n$, пусть $(x_n)$ не ограничена, то есть $\nexists\,m:\,\forall n\,\,\,x_n<m$, тогда $\sup(x_n)=+\infty$, а значит $(x_n)\to \infty$. Пусть $\exists\,m:\,\forall n \,\,\,x_n\leq m$ и $m=\sup(x_n)$. Тогда $m=\lim_{n\to\infty}(x_n)$. Доказательство для монотонно убывающей последовательности аналогично.
\end{proof}
\end{document}