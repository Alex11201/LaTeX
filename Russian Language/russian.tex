\documentclass[12pt]{article} 

%Формат файла
\usepackage[paperheight=297mm,
   paperwidth=210mm,
   top=20mm,
   bottom=20mm,
   left=15mm,
   right=15mm]{geometry}


%Текст
\usepackage[fontsize=12pt]{fontsize}
\usepackage[english, russian]{babel}
\usepackage[T2A]{fontenc}
\usepackage{color}
\usepackage{transparent}
\usepackage{amsthm}
\usepackage{multicol}
\usepackage{csquotes}
\usepackage[e]{esvect}
\renewcommand{\mkbegdispquote}[2]{\itshape}
\parindent=0cm

\theoremstyle{definition}
\newtheorem{rrule}{Правило}[section]
\newtheorem{algorithm}[rrule]{Алгоритм}
\newtheorem{definition}{Определение}

\theoremstyle{remark}
\newtheorem*{example}{Пример}
\newtheorem*{exception}{Исключения}

%Картинки
\usepackage{graphicx}
\usepackage{wrapfig}
\usepackage{subcaption}
\usepackage{tikz}
\usepackage{tkz-euclide}
\usepackage{pgfplots}
\usetikzlibrary {arrows.meta}
\usetikzlibrary{calc}
\usetikzlibrary{through}
\usetikzlibrary{intersections}
\usetikzlibrary{decorations.markings}
\usetikzlibrary{positioning}
\usetikzlibrary{3d}
\usetikzlibrary{perspective}

\makeatletter
\tikzoption{canvas is plane}[]{\@setOxy#1}
\def\@setOxy O(#1,#2,#3)x(#4,#5,#6)y(#7,#8,#9)%
  {\def\tikz@plane@origin{\pgfpointxyz{#1}{#2}{#3}}%
   \def\tikz@plane@x{\pgfpointxyz{#4}{#5}{#6}}%
   \def\tikz@plane@y{\pgfpointxyz{#7}{#8}{#9}}%
   \tikz@canvas@is@plane
  }
\makeatother  


%Математика
\usepackage{amsmath}
\usepackage{amsfonts}
\usepackage{amssymb}
\usepackage[makeroom]{cancel}

%Всякое
\usepackage{relsize}
\usepackage{enumerate}
\usepackage[inline]{enumitem}
\usepackage{hyperref}
\usepackage{rslingu}

%Мат команды
\newcommand{\N}{\mathbb{N}}
\newcommand{\Z}{\mathbb{Z}}
\newcommand{\Q}{\mathbb{Q}}
\newcommand{\R}{\mathbb{R}}
\newcommand*\diff{\mathop{}\!\mathrm{d}}

%Оглавление
\title{\textbf{Русский язык}}\date{}

\hypersetup{
    colorlinks,
    citecolor=black,
    filecolor=black,
    linkcolor=black,
    urlcolor=black
}

\renewcommand{\labelenumii}{\arabic{enumi}.\arabic{enumii}}
\renewcommand{\labelenumiii}{\arabic{enumi}.\arabic{enumii}.\arabic{enumiii}}
\renewcommand{\labelenumiv}{\arabic{enumi}.\arabic{enumii}.\arabic{enumiii}.\arabic{enumiv}}

\pgfplotsset{compat=1.18}
\setlist[enumerate]{itemsep=0mm}
\begin{document}


\maketitle
\tableofcontents
\label{toc}
\newpage

\section{Орфография}

\subsection{Правописание НЕ с различными частями речи}

\begin{algorithm}[Правописание НЕ с существительными]
    $ $\par\nobreak\ignorespaces
    \begin{center}
        \begin{tikzpicture}
            \def\gap {-0.7};
            \draw (-.25,0) |- (-7,\gap) |- +(0,\gap) coordinate (A);
            \draw (.25,0) |- (7,\gap) |- +(0,\gap) coordinate (B);
            \draw (A) |- +(7,\gap) coordinate (C);
            \draw (C) +(-.25,0) |- (-7,\gap*4) |- +(0,\gap) coordinate (D);
            \draw (C) +(.25,0) |- (5,\gap*4) |- +(0,\gap) coordinate (E);
            \draw (E) |- +(-5,\gap) coordinate (F);
            \draw (F) +(-.25,0) |- (-5,\gap*7) |- +(0,\gap) coordinate (G);
            \draw (F) +(.25,0) |- (5,\gap*7) |- +(0,\gap) coordinate (H);
            \draw (G) -- +(0,-1) coordinate (I);
            \draw (H) -- +(0,-1) coordinate (J);
            \draw (D) |- (I);
            \draw (B) |- (J);
            \draw (0,0) node[fill=white, draw] {Употребляется без НЕ};
            \draw (C) node[fill=white, draw] {Есть противопоставление с союзом «а»};
            \draw (F) node[fill=white, draw] {Можно заменить синонимом без НЕ};
            \draw (I) node[fill=white, draw] {Раздельно};
            \draw (J) node[fill=white, draw] {Слитно};
            \foreach \point in {(A), (D), (H)}{
				\draw \point node[fill=white, draw] {Да};
			}
            \foreach \point in {(B), (E), (G)}{
				\draw \point node[fill=white, draw] {Нет};
			}
        \end{tikzpicture}
    \end{center}
\end{algorithm}

\begin{algorithm}[Правописание НЕ с прилагательными и наречиями на -о/-е]
    $ $\par\nobreak\ignorespaces
    \begin{center}
        \begin{tikzpicture}
            \def\gap {-0.7};
            \draw (-.25,0) |- (-7,\gap) |- +(0,\gap) coordinate (A);
            \draw (.25,0) |- (7,\gap) |- +(0,\gap) coordinate (B);
            \draw (A) |- +(7,\gap) coordinate (C);
            \draw (C) +(-.25,0) |- (-7,\gap*4) |- +(0,\gap) coordinate (D);
            \draw (C) +(.25,0) |- (5,\gap*4) |- +(0,\gap) coordinate (E);
            \draw (E) |- +(-5,\gap) coordinate (F);
            \draw (F) +(-.25,0) |- (-5,\gap*7) |- +(0,\gap) coordinate (G);
            \draw (F) +(.25,0) |- (5,\gap*7) |- +(0,\gap) coordinate (H);
            \draw (H) |- +(-5,\gap) coordinate (I);
            \draw (I) +(-.25,0) |- (-3,\gap*10) |- +(0,\gap) coordinate (J);
            \draw (I) +(.25,0) |- (3,\gap*10) |- +(0,\gap) coordinate (K);
            \draw (J) -- +(0,-1) coordinate (L);
            \draw (K) -- +(0,-1) coordinate (M);
            \draw (D) |- (L);
            \draw (G) |- (L);
            \draw (B) |- (M);
            \draw (0,0) node[fill=white, draw] {Употребляется без НЕ};
            \draw (C) node[fill=white, draw] {Есть противопоставление с союзом «а»};
            \draw (F) node[fill=white, draw] {Есть зависимые слова};
            \draw (I) node[fill=white, draw] {Можно заменить синонимом без НЕ};
            \draw (L) node[fill=white, draw] {Раздельно};
            \draw (M) node[fill=white, draw] {Слитно};
            \foreach \point in {(A), (D), (G), (K)}{
				\draw \point node[fill=white, draw] {Да};
			}
            \foreach \point in {(B), (E), (H), (J)}{
				\draw \point node[fill=white, draw] {Нет};
			}
        \end{tikzpicture}
    \end{center}
\end{algorithm}
\begin{exception}
    $ $\par\nobreak\ignorespaces
    \begin{enumerate}
        \item НЕ с прилагательными пишется раздельно, если они стоят в сравнительной форме.
        \item НЕ с прилагательными пишется раздельно, если они обозначают вкус или цвет.
    \end{enumerate}
\end{exception}

\begin{rrule}[Правописание НЕ с краткими прилагательными]
    $ $\par\nobreak\ignorespaces
    \begin{enumerate}
        \item НЕ с краткими прилагательными пишется как в полной форме прилагательного.
        \item НЕ с краткими прилагательными пишется раздельно, если они не употребляются в полной форме.
    \end{enumerate}
\end{rrule}

\begin{algorithm}[Правописание НЕ с причастиями]
    $ $\par\nobreak\ignorespaces
    \begin{center}
        \begin{tikzpicture}
            \def\gap {-0.7};
            \draw (-.25,0) |- (-7,\gap) |- +(0,\gap) coordinate (A);
            \draw (.25,0) |- (7,\gap) |- +(0,\gap) coordinate (B);
            \draw (A) |- +(7,\gap) coordinate (C);
            \draw (C) +(-.25,0) |- (-7,\gap*4) |- +(0,\gap) coordinate (D);
            \draw (C) +(.25,0) |- (5,\gap*4) |- +(0,\gap) coordinate (E);
            \draw (E) |- +(-5,\gap) coordinate (F);
            \draw (F) +(-.25,0) |- (-5,\gap*7) |- +(0,\gap) coordinate (G);
            \draw (F) +(.25,0) |- (5,\gap*7) |- +(0,\gap) coordinate (H);
            \draw (H) |- +(-5,\gap) coordinate (I);
            \draw (I) +(-.25,0) |- (-3,\gap*10) |- +(0,\gap) coordinate (J);
            \draw (I) +(.25,0) |- (3,\gap*10) |- +(0,\gap) coordinate (K);
            \draw (J) -- +(0,-1) coordinate (L);
            \draw (K) -- +(0,-1) coordinate (M);
            \draw (D) |- (L);
            \draw (G) |- (L);
            \draw (B) |- (M);
            \draw (0,0) node[fill=white, draw] {Употребляется без НЕ};
            \draw (C) node[fill=white, draw] {Краткое};
            \draw (F) node[fill=white, draw] {Есть противопоставление с союзом «а»};
            \draw (I) node[fill=white, draw] {Есть зависимые слова};
            \draw (L) node[fill=white, draw] {Раздельно};
            \draw (M) node[fill=white, draw] {Слитно};
            \foreach \point in {(A), (D), (G), (J)}{
				\draw \point node[fill=white, draw] {Да};
			}
            \foreach \point in {(B), (E), (H), (K)}{
				\draw \point node[fill=white, draw] {Нет};
			}
        \end{tikzpicture}
    \end{center}
\end{algorithm}

\begin{rrule}[Правописание НЕ с глаголами и деепричастями]
    НЕ с глаголами и деепричастиями пишется слитно только в тех случаях, когда они не употребляются без НЕ или начинаются с приставки недо-.
    \begin{example}
        Нанавидеть; \rsPrefix{недо}есть; негодуя; \rsPrefix{недо}любливая.
    \end{example}
\end{rrule}

\subsection{Правописание Н и НН в различных частях речи}

\begin{rrule}[Правописание Н и НН в отымённых прилагательных]
    $ $\par\nobreak\ignorespaces
    \begin{enumerate}
        \item НН пишется, если основа оканчивается на «н» и есть суффикс -н-.
        \begin{example}
            \rsRoot{\rsBase{картон}}\rsSuffix{н}ый; \rsRoot{\rsBase{лун}}\rsSuffix{н}ый; \rsRoot{\rsBase{камен}}\rsSuffix{н}ый.
        \end{example}
        \item НН пишется, когда есть суффикс -онн- или -енн-.
        \begin{example}
            Мысл\rsSuffix{енн}ый; листв\rsSuffix{енн}ый; революци\rsSuffix{онн}ый; позици\rsSuffix{онн}ый.
        \end{example}
        \begin{exception}
            Ветреный (человек).
        \end{exception}
        \item Н пишется, когда есть суффикс -ин-, -ан- или -ян-.
        \begin{example}
            Льв\rsSuffix{ин}ый; песч\rsSuffix{ан}ый; трав\rsSuffix{ян}ой.
        \end{example}
        \begin{exception}
            Стеклянный; оловянный; деревянный.
        \end{exception}
        \item В кратких прилагательных пишется столько же «н», сколько в полной форме.
    \end{enumerate}
\end{rrule}

\begin{algorithm}[Правописание Н и НН в причастиях]
    $ $\par\nobreak\ignorespaces
    \begin{center}
        \begin{tikzpicture}
            \def\gap {-0.7};
            \draw (-.25,0) |- (-8,\gap) |- +(0,\gap) coordinate (A);
            \draw (.25,0) |- (8,\gap) |- +(0,\gap) coordinate (B);
            \draw (B) |- +(-8,\gap) coordinate (C);
            \draw (C) +(-.25,0) |- (-6,\gap*4) |- +(0,\gap) coordinate (D);
            \draw (C) +(.25,0) |- (8,\gap*4) |- +(0,\gap) coordinate (E);
            \draw (D) |- +(6,\gap) coordinate (F);
            \draw (F) +(-.25,0) |- (-6,\gap*7) |- +(0,\gap) coordinate (G);
            \draw (F) +(.25,0) |- (6,\gap*7) |- +(0,\gap) coordinate (H);
            \draw (G) |- +(6,\gap) coordinate (I);
            \draw (I) +(-.25,0) |- (-6,\gap*10) |- +(0,\gap) coordinate (J);
            \draw (I) +(.25,0) |- (4,\gap*10) |- +(0,\gap) coordinate (K);
            \draw (J) |- +(6,\gap) coordinate (L);
            \draw (L) +(-.25,0) |- (-2,\gap*13-.3) |- +(0,\gap) coordinate (M);
            \draw (L) +(.25,0) |- (2,\gap*13 -.3) |- +(0,\gap) coordinate (N);
            \draw (M) -- +(0,-1) coordinate (O);
            \draw (N) -- +(0,-1) coordinate (P);
            \draw (A) |- (O);
            \draw (E) |- (P);
            \draw (H) |- (P);
            \draw (K) |- (P);
            \draw (0,0) node[fill=white, draw] {Краткое};
            \draw (C) node[fill=white, draw] {Есть приставки кроме не-, свеже-, младо-};
            \draw (F) node[fill=white, draw] {Есть зависимые слова};
            \draw (I) node[fill=white, draw] {Оканчивается на -ованный/-ёванный};
            \draw (L) node[fill=white, draw, align=center] {Образовано от бесприставочного\\глагола совершенного вида};
            \draw (O) node[fill=white, draw] {Н};
            \draw (P) node[fill=white, draw] {НН};
            \foreach \point in {(A), (E), (H), (K), (N)}{
				\draw \point node[fill=white, draw] {Да};
			}
            \foreach \point in {(B), (D), (G), (J), (M)}{
				\draw \point node[fill=white, draw] {Нет};
			}
        \end{tikzpicture}
    \end{center}
\end{algorithm}
\begin{exception}
    Кованый; жёваный; клёваный; посажёный, названый (брат).
\end{exception}

\subsection{Правописание гласных в различных частях слова}

\begin{rrule}[Чередование гласных в корне]
    $ $\par\nobreak\ignorespaces
    \begin{enumerate}
        \item Зависят от суффикса -а-: -бер-/-бир-; -дер-/-дир-; -мер-/-мир-; -пер-/-пир-; -тер-/-тир-; -блест-/-блист-; -стел-/-стил-; -чет-/-чит-; -жег-/-жиг-.
        \item Зависят от ударения: -г\'aр-/-гор-; -тв\'aр-/-твор-; -кл\'aн-/-клон-; -зар-/-з\'oр-.
        \begin{exception}
            $ $\par\nobreak\ignorespaces
            \begin{enumerate}
                \item В корнях -плав-/-плов-/-плыв- без ударения пишется «а» в словах плавучий, плавник, поплавок, жук-плавунец и производных от них; «о» пишется в словах пловец и пловчиха; «ы» пишется в словах выплыть, плывущий, плывун (слой грунта).
                \item Пр\'{и}\rsRoot{гарь}; \'{и}з\rsRoot{гарь}; в\'{ы}\rsRoot{гар}ки; \rsRoot{гар}ев\'{о}й; \'{у}\rsRoot{тварь}; \rsRoot{зар}\'{я}; \rsRoot{зар}н\'{и}ца.
            \end{enumerate}
        \end{exception}
        \item Зависят от последующей буквы: -кас-/-косн-; -лаг-/-лож-; -раст-/-ращ-/-рос-; -скак-/-скоч-
        \item Зависят от значения:
        \begin{enumerate}
            \item -мак-/-мок-: пишется «а» в значении погружения в жидкость; пишется «о» в значении пропускания жидкости.
            \begin{example}
                Макать перо в чернила; обмакнуть кисть; туфли промокают; непромокаемый плащ.
            \end{example}
            \begin{exception}
                В корне -моч- перед «ч» всегда пишется «о».
            \end{exception}
            \item -равн-/-ровн-: пишется «а» в значении «одинаковый, равный»; пишется «о» в значении «делать гладким, ровным».
            \begin{example}
                Сравнить числа; уравнять в правах; выровнять пол.
            \end{example}
            \begin{exception}
                Равнина.
            \end{exception}
        \end{enumerate}
    \end{enumerate}
\end{rrule}

\begin{rrule}[Правописание О и Ё после шипящих в корне]
    В корне под ударением после шипящих пишется «ё» только в тех случаях, когда можно подобрать проверочное слово с «е».
    \begin{example}
        Шёпот -- шептать; жёлтый -- желтеть; крыжовник; трущоба.
    \end{example}
\end{rrule}

\begin{rrule}[Правописание Ы и И после Ц]
    $ $\par\nobreak\ignorespaces
    \begin{enumerate}
        \item В корнях слов после «ц» пишется «и».
        \begin{exception}
            Цыган на цыпочках цыплёнку цыкнул «Цыц!»
        \end{exception}
        \item В суффиксах и окончаниях существительных и прилагательных после «ц» пишется «ы».
        \begin{example}
            Сестриц\rsSuffix{ын}, цариц\rsSuffix{ын}, огурц\rsEnding{ы}, молодц\rsEnding{ы}.
        \end{example}
    \end{enumerate}
\end{rrule}

\subsection{Правописание приставок}

\begin{rrule}[Правописание приставок на з/с]
    В приставках с чередованием з/с пишется «з», если следующий после приставки согласный звонкий и «с» в противном случае. Приставка с- пишется как перед глухими, так и перед звонкими согласными; приставки з- нет.
\end{rrule}

\begin{rrule}[Правописание приставок ПРЕ и ПРИ]
    $ $\par\nobreak\ignorespaces
    \begin{multicols}{2}
        Приставка при- пишется, если:
        \begin{enumerate}
            \item Имеет значение присоединения.
            \begin{example}
                Приклеить; прикрутить.
            \end{example}
            \item Имеет значение пространственной близости.
            \begin{example}
                Придворный; приморский.
            \end{example}
            \item Имеет значение неполноты действия:
            \begin{example}
                Прилечь; притормозить.
            \end{example}
        \end{enumerate}
        Приставка пре- пишется, если:
        \begin{enumerate}
            \item Имеет значение высокой степени качества или действия.
            \begin{example}
                Пренеприятный; преуспевать.
            \end{example}
            \item Имеет значение, близкое к значению приставки пере-.
            \begin{example}
                Прервать; пресечь; преграждать.
            \end{example}
        \end{enumerate}
    \end{multicols}
    \begin{exception}
        Преследовать; приготовить; причудливый; пресловутый; преувеличить; привередливый; неприступная (крепость) и другие.
    \end{exception}
\end{rrule}

\begin{rrule}[Правописание разделительных Ъ и Ь]
    $ $\par\nobreak\ignorespaces
    \begin{enumerate}
        \item Разделительный твёрдый знак пишется перед буквами «е», «ё», «ю», «я»:
        \begin{enumerate}
            \item После приставок, оканчивающихся на согласную.
            \begin{example}
                Предъявить; предъюбилейный.
            \end{example}
            \item В иноязычных словах.
            \begin{example}
                Адъютант; панъевропейский; инъекция; конъюнктура; объект; субъект.
            \end{example}
            \item В сложных словах, первая часть которых состоит из числительных.
            \begin{example}
                Двухъярусный; трехъязычный.
            \end{example}
        \end{enumerate}
        \item Разделительный мягкий знак пишется перед буквами «и», «е», «ё», «ю», «я»:
        \begin{enumerate}
            \item В корне слова.
            \begin{example}
                Льёт; воробьи; подьячий; лисьей; завьюженный.
            \end{example}
            \item В иноязычных словах перед «о».
            \begin{example}
                Павильон; медальон; компаньон; батальон; почтальон.
            \end{example}
        \end{enumerate}
    \end{enumerate}
\end{rrule}

\begin{rrule}[Правописание Ы и И после приставок]
    После русских приставок, оканчивающихся на согласную, кроме меж- и сверх- вместо «и» пишется «ы».
    \begin{example}
        Разыграть; безынициативный; сверхинтересный; межинститутский; дезинфекция.
    \end{example}
\end{rrule}

\end{document}