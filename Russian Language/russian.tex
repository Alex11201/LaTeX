\documentclass[12pt]{article} 

%Формат файла
\usepackage[paperheight=297mm,
   paperwidth=210mm,
   top=20mm,
   bottom=20mm,
   left=15mm,
   right=15mm]{geometry}


%Текст
\usepackage[fontsize=12pt]{fontsize}
\usepackage[english, russian]{babel}
\usepackage[T2A]{fontenc}
\usepackage{color}
\usepackage{transparent}
\usepackage{amsthm}
\usepackage{multicol}
\usepackage{csquotes}
\usepackage[e]{esvect}
\renewcommand{\mkbegdispquote}[2]{\itshape}
\parindent=0cm

\theoremstyle{definition}
\newtheorem{rrule}{Правило}[section]
\newtheorem{algorithm}[rrule]{Алгоритм}
\newtheorem{definition}{Определение}

\theoremstyle{remark}
\newtheorem*{example}{Пример}
\newtheorem*{exception}{Исключения}

%Картинки
\usepackage{graphicx}
\usepackage{wrapfig}
\usepackage{subcaption}
\usepackage{tikz}
\usepackage{tkz-euclide}
\usepackage{pgfplots}
\usetikzlibrary {arrows.meta}
\usetikzlibrary{calc}
\usetikzlibrary{through}
\usetikzlibrary{intersections}
\usetikzlibrary{decorations.markings}
\usetikzlibrary{positioning}
\usetikzlibrary{3d}
\usetikzlibrary{perspective}

\makeatletter
\tikzoption{canvas is plane}[]{\@setOxy#1}
\def\@setOxy O(#1,#2,#3)x(#4,#5,#6)y(#7,#8,#9)%
  {\def\tikz@plane@origin{\pgfpointxyz{#1}{#2}{#3}}%
   \def\tikz@plane@x{\pgfpointxyz{#4}{#5}{#6}}%
   \def\tikz@plane@y{\pgfpointxyz{#7}{#8}{#9}}%
   \tikz@canvas@is@plane
  }
\makeatother  


%Математика
\usepackage{amsmath}
\usepackage{amsfonts}
\usepackage{amssymb}
\usepackage[makeroom]{cancel}

%Всякое
\usepackage{relsize}
\usepackage{enumerate}
\usepackage[inline]{enumitem}
\usepackage{hyperref}
\usepackage{rslingu}

%Мат команды
\newcommand{\N}{\mathbb{N}}
\newcommand{\Z}{\mathbb{Z}}
\newcommand{\Q}{\mathbb{Q}}
\newcommand{\R}{\mathbb{R}}
\newcommand{\romannumeralcaps}[1]
    {\MakeUppercase{\romannumeral #1}}
\newcommand*\diff{\mathop{}\!\mathrm{d}}

%Оглавление
\title{\textbf{Русский язык}}\date{}

\hypersetup{
    colorlinks,
    citecolor=black,
    filecolor=black,
    linkcolor=black,
    urlcolor=black
}

\renewcommand{\labelenumii}{\arabic{enumi}.\arabic{enumii}}
\renewcommand{\labelenumiii}{\arabic{enumi}.\arabic{enumii}.\arabic{enumiii}}
\renewcommand{\labelenumiv}{\arabic{enumi}.\arabic{enumii}.\arabic{enumiii}.\arabic{enumiv}}

\pgfplotsset{compat=1.18}
\setlist[enumerate]{itemsep=0mm}
\begin{document}


\maketitle
\tableofcontents
\label{toc}
\newpage

\section{Анализ текста}

\subsection{Типы речи}

\begin{definition}
    Повествование -- это рассказ о событии, которое происходит в определенное время. Действия в тексте последовательны, логически связаны друг с другом.
\end{definition}
\begin{definition}
    Описание -- это изображение предмета, чувства, явления или события.
\end{definition}
\begin{definition}
    Рассуждение -- это развитие определенной мысли, объяснение явления и выражение собственного мнения.
\end{definition}

\subsection{Стили речи}

\begin{algorithm}[Определение стиля речи] К разговорному стилю речи относится только устная речь. К книжному стилю речи относятся:
    \begin{enumerate}
        \item Официально-деловой:
        \begin{enumerate}
            \item Используются речевые клише, официальная терминология. Цель -- официальное обращение.
            \item Сфера употребления: заявления, протоколы, уведомления и другие деловые документы.
        \end{enumerate}
        \item Публицистический:
        \begin{enumerate}
            \item Используется эмоциональная, выразительная речь. Цель -- воздействие на читателя.
            \item Сфера употребления: выступления ораторов, СМИ.
        \end{enumerate}
        \item Художественный:
        \begin{enumerate}
            \item Используются средства выразительности. 
            \item Употребляется только в художественных произведениях.
        \end{enumerate}
        \item Научный:
        \begin{enumerate}
            \item Используются термины и профессиональная лексика.
            \item Сфера употребления: учебники, лекции, научные статьи.
            \item Делится на академический и научно-популярный.
        \end{enumerate}
    \end{enumerate}
\end{algorithm}

\section{Морфология}

\subsection{Формы имён прилагательных}

\begin{rrule}[Образование сравнительной формы прилагательных]
    Для образования сравнительной формы прилагательного используются слова более (менее) или суффикс -е- (-ше-) и происходит чередование согласных. Невозможно использовать обе сравнительные формы одновременно.
    \begin{example}
        Мягкий -- мягче; строгий -- строже; тихий -- тише; богатый -- богаче; молодой -- моложе; густой -- гуще; низкий -- ниже; ранний -- раньше; старый -- старше.
    \end{example}
\end{rrule}

\begin{rrule}[Образование превосходной формы прилагательных]
    Для образования сравнительной формы прилагательного используются слова самый, наиболее (наименее) или суффикс -айш- (-ейш-) и происходит чередование согласных. Невозможно использовать обе превосходные формы одновременно.
    \begin{example}
        Крепкий -- крепчайший; редкий -- редчайший; низкий -- нижайший; добрый -- добрейший; умный -- умнейший; красивый -- красивейший. 
    \end{example}
\end{rrule}

\subsection{Формы имён числительных}

\begin{rrule}[Склонение имён числительных]
    \hfill 
    \begin{enumerate}
        \item В составных и сложных количественных числительных меняется каждая часть.
        \begin{example}
            Выступил перед девятьюстами шестьюдесятью семью зрителями.
        \end{example}
        \item Числительное «тысяча» склоняется как существительное первого склонения.
        \item Числительные «сто» и «сорок» в косвенных падежах имеют только одну форму -- «ста» и «сорока», но в составе сложных числительных «сто» изменяется иначе.
        \begin{example}
            Трёхсот, трёмстам, тремястами, о трёхстах.
        \end{example}
        \item При склонении составных порядковых числительных изменяется только их последняя часть.
        \begin{example}
            Две тысячи четырнадцатый год; к две тысячи четырнадцатому году; до две тысячи четырнадцатого года. 
        \end{example}
        \item Числительные «оба» («обе») и «полтора» («полторы») имеют две родовые формы.
    \end{enumerate}
\end{rrule}

\section{Орфография}

\subsection{Правописание гласных в различных частях слова}

\begin{rrule}[Чередование гласных в корне]
    \hfill
    \begin{enumerate}
        \item Зависят от суффикса -а-: -бер-/-бир-; -дер-/-дир-; -мер-/-мир-; -пер-/-пир-; -тер-/-тир-; -блест-/-блист-; -стел-/-стил-; -чет-/-чит-; -жег-/-жиг-.
        \item Зависят от ударения: -г\'aр-/-гор-; -тв\'aр-/-твор-; -кл\'aн-/-клон-; -зар-/-з\'oр-.
        \begin{exception}
            \hfill
            \begin{enumerate}
                \item В корнях -плав-/-плов-/-плыв- без ударения пишется «а» в словах плавучий, плавник, поплавок, жук-плавунец и производных от них; «о» пишется в словах пловец и пловчиха; «ы» пишется в словах выплыть, плывущий, плывун (слой грунта).
                \item Пр\'{и}\rsRoot{гарь}, \'{и}з\rsRoot{гарь}, в\'{ы}\rsRoot{гар}ки, \rsRoot{гар}ев\'{о}й, \'{у}\rsRoot{тварь}, \rsRoot{зар}\'{я}, \rsRoot{зар}н\'{и}ца.
            \end{enumerate}
        \end{exception}
        \item Зависят от последующей буквы: -кас-/-косн-; -лаг-/-лож-; -раст-/-ращ-/-рос-; -скак-/-скоч-
        \item Зависят от значения:
        \begin{enumerate}
            \item -мак-/-мок-: пишется «а» в значении погружения в жидкость; пишется «о» в значении пропускания жидкости.
            \begin{example}
                Макать перо в чернила; обмакнуть кисть; туфли промокают; непромокаемый плащ.
            \end{example}
            \begin{exception}
                В корне -моч- перед «ч» всегда пишется «о».
            \end{exception}
            \item -равн-/-ровн-: пишется «а» в значении «одинаковый, равный»; пишется «о» в значении «делать гладким, ровным».
            \begin{example}
                Сравнить числа; уравнять в правах; выровнять пол.
            \end{example}
            \begin{exception}
                Равнина.
            \end{exception}
        \end{enumerate}
    \end{enumerate}
\end{rrule}

\begin{rrule}[Правописание О и Ё после шипящих в корне]
    В корне под ударением после шипящих пишется «ё» только в тех случаях, когда можно подобрать проверочное слово с «е».
    \begin{example}
        Шёпот -- шептать; жёлтый -- желтеть; крыжовник; трущоба.
    \end{example}
\end{rrule}

\begin{rrule}[Правописание Ы и И после Ц]
    \hfill
    \begin{enumerate}
        \item В корнях слов после «ц» пишется «и».
        \begin{exception}
            Цыган на цыпочках цыплёнку цыкнул «Цыц!»
        \end{exception}
        \item В суффиксах и окончаниях существительных и прилагательных после «ц» пишется «ы».
        \begin{example}
            Сестриц\rsSuffix{ын}, цариц\rsSuffix{ын}, огурц\rsEnding{ы}, молодц\rsEnding{ы}.
        \end{example}
    \end{enumerate}
\end{rrule}

\begin{rrule}[Правописание О и А на конце наречий]
    \hfill
    \begin{enumerate}
        \item На конце наречий, образованных от приставок в-, на-, за-, пишется «о».
        \item На конце наречий, образованных от приставок из-, до-, с-, пишется «а».
    \end{enumerate}
\end{rrule}

\subsection{Правописание приставок в различных частях речи}

\begin{rrule}[Правописание приставок на з/с]
    \hfill
    \begin{enumerate}
        \item В приставках с чередованием з/с пишется «з», если следующий после приставки согласный звонкий и «с» в противном случае.
        \item Приставка с- пишется как перед глухими, так и перед звонкими согласными; приставки з- нет.
    \end{enumerate}
\end{rrule}

\begin{rrule}[Правописание приставок пре-/при-]
    \hfill
    \begin{multicols}{2}
        Приставка при- пишется, если:
        \begin{enumerate}
            \item Имеет значение присоединения.
            \begin{example}
                Приклеить, прикрутить.
            \end{example}
            \item Имеет значение пространственной близости.
            \begin{example}
                Придворный, приморский.
            \end{example}
            \item Имеет значение неполноты действия:
            \begin{example}
                Прилечь, притормозить.
            \end{example}
        \end{enumerate}
        Приставка пре- пишется, если:
        \begin{enumerate}
            \item Имеет значение высокой степени качества или действия.
            \begin{example}
                Пренеприятный, преуспевать.
            \end{example}
            \item Имеет значение, близкое к значению приставки пере-.
            \begin{example}
                Прервать, пресечь, преграждать.
            \end{example}
        \end{enumerate}
    \end{multicols}
    \begin{exception}
        Преследовать, приготовить, причудливый, пресловутый, преувеличить, привередливый, неприступная (крепость) и другие.
    \end{exception}
\end{rrule}

\begin{rrule}[Правописание разделительных Ъ и Ь]
    \hfill
    \begin{enumerate}
        \item Разделительный твёрдый знак пишется перед буквами «е», «ё», «ю», «я»:
        \begin{enumerate}
            \item После приставок, оканчивающихся на согласную.
            \begin{example}
                Предъявить, предъюбилейный.
            \end{example}
            \item В иноязычных словах.
            \begin{example}
                Адъютант, панъевропейский, инъекция, конъюнктура, объект, субъект.
            \end{example}
            \item В сложных словах, первая часть которых состоит из числительных.
            \begin{example}
                Двухъярусный, трехъязычный.
            \end{example}
        \end{enumerate}
        \item Разделительный мягкий знак пишется перед буквами «и», «е», «ё», «ю», «я»:
        \begin{enumerate}
            \item В корне слова.
            \begin{example}
                Льёт, воробьи, подьячий, лисьей, завьюженный.
            \end{example}
            \item В иноязычных словах перед «о».
            \begin{example}
                Павильон, медальон, компаньон, батальон, почтальон.
            \end{example}
        \end{enumerate}
    \end{enumerate}
\end{rrule}

\begin{rrule}[Правописание Ы и И после приставок]
    После русских приставок, оканчивающихся на согласную, кроме меж- и сверх- вместо «и» пишется «ы».
    \begin{example}
        Разыграть, безынициативный, сверхинтересный, межинститутский, дезинфекция.
    \end{example}
\end{rrule}

\subsection{Правописание окончаний глаголов}

\begin{algorithm}[Определение спряжения глагола]
    \hfill
    \begin{enumerate}
        \item Если личное окончание глагола в настоящем времени ударное, то спряжение определяется по нему: «е», «у», «ю» относятся к \romannumeralcaps{1} спряжению; «и», «а», «я» -- ко \romannumeralcaps{2} спряжению.
        \begin{example}
            \rsVerb*{Выбер\'{е}т}[\romannumeralcaps{1} спр.] -- \rsVerb*{выбер\'{у}т}[\romannumeralcaps{1} спр.]; \rsVerb*{пил\'{и}т}[\romannumeralcaps{2} спр.] -- \rsVerb*{пил\'{я}т}[\romannumeralcaps{2} спр.].
        \end{example}
        \item Если личное окончание глагола безударное, спряжение определяется по его инфинитиву: оканчивающиеся на -ить относятся ко \romannumeralcaps{2} спряжению, а иначе – к \romannumeralcaps{1} спряжению.
        \begin{example}
            \rsVerb*{Рисовать}[\romannumeralcaps{1} спр.], \rsVerb*{петь}[\romannumeralcaps{1} спр.], \rsVerb*{купить}[\romannumeralcaps{2} спр.], \rsVerb*{косить}[\romannumeralcaps{2} спр.].
        \end{example}
        \begin{exception}
            Глаголы брить, стелить, зиждиться и зыбиться относятся к \romannumeralcaps{1} спряжению.
        \end{exception}
        \item Спряжение глаголов с ударной приставкой вы- определяется спряжением этого глагола, если отбросить приставку.
        \begin{example}
            В\'{ы}спишься -- сп\'{и}шь; в\'{ы}глядит -- гляд\'{и}т.
        \end{example}
    \end{enumerate}
    \begin{exception}
        Глаголы гнать, держать, дышать, слышать, терпеть, видеть, ненавидеть, обидеть, зависеть, смотреть, вертеть относятся ко \romannumeralcaps{2} спряжению.
    \end{exception}
\end{algorithm}

\begin{rrule}[Правописание окончаний глаголов]
    \hfill
    \begin{enumerate}
        \item В личных окончаниях глаголов \romannumeralcaps{1} спряжения пишутся гласные «е», «у», «ю».
        \item В личных окончаниях глаголов \romannumeralcaps{2} спряжения пишутся гласные «и», «а», «я».
    \end{enumerate}
\end{rrule}

\subsection{Правописание суффиксов в различных частях речи}

\subsubsection{Правописание суффиксов существительных}

\begin{rrule}[Правописание суффиксов -чик-/-щик-] После букв «д», «т», «з», «с», «ж» пишется суффикс -чик-; в остальных случаях пишется суффикс -щик-.
    \begin{exception}
        Процентщик, асфальтщик, алебардщик, флейтщик. 
    \end{exception}
\end{rrule}

\begin{rrule}[Правописание суффиксов -ик-/-ек-]
    Если при склонении существительного гласная выпадает, то пишется суффикс -ек-, иначе – -ик-.
    \begin{example}
        Ботиночек -- ботиночка; ключик -- ключика.
    \end{example}
\end{rrule}

\begin{rrule}[Правописание суффиксов -иц-/-ец-]
    \hfill
    \begin{enumerate}
        \item В существительных мужского рода пишется суффикс -ец-.
        \item В существительных женского рода пишется суффикс -иц-.
        \item В существительных среднего рода пишется суффикс -ец-, если ударение падает после суффикса, и -иц- в противном случае.
        \begin{example}
            Ружь\rsSuffix{ец}\'{о}, кр\'{е}сл\rsSuffix{иц}е.
        \end{example}
    \end{enumerate}
\end{rrule}

\begin{rrule}[Правописание суффиксов -ичк-/-ечк-]
    \hfill
    \begin{enumerate}
        \item Если основа существительного оканчивается на -иц, то пишется -ичк-.
        \item Суффикс -ечк- пишется в существительных с другими основами, а также в словах на -мя. 
    \end{enumerate}
    \begin{example}
        Пуговица -- пуговичка; сито -- ситечко; семя -- семечко.
    \end{example}
\end{rrule}

\begin{rrule}[Правописание суффиксов -инк-/-енк-]
    \hfill
    \begin{enumerate}
        \item Суффикс -инк- пишется в существительных женского рода на -ин-.
        \item Суффикс -енк- пишется в существительных на -ня/-на. 
    \end{enumerate}
    \begin{example}
        Изюмина -- изюминка; беженка; башня -- башенка.
    \end{example}
\end{rrule}

\begin{rrule}[Правописание суффиксов -оньк-/-еньк-]
    \hfill
    \begin{enumerate}
        \item Суффикс -оньк- в существительных пишется после твёрдых согласных.
        \item Суффикс -еньк- в существительных пишется после мягких согласных и шипящих. 
    \end{enumerate}
    \begin{exception}
        Заинька, баиньки, паинька.
    \end{exception}
\end{rrule}

\begin{rrule}[Правописание О и Ё после шипящих в суффиксах]
    \hfill
    \begin{enumerate}
        \item В суффиксах под ударением после шипящих пишется «о».
        \item В суффиксах глаголов и отглагольных слов пишется «ё».
    \end{enumerate}
    \begin{example}
        Зайчонок, парчовый, смешон, растушёвывать, лишён, тушёнка, тренажёр, стажёр.
    \end{example}
\end{rrule}

\subsubsection{Правописание суффиксов прилагательных}

\begin{rrule}[Правописание суффиксов -ив-/-ев-]
    Под ударением в прилагательных пишется суффикс -ив-; без ударения – -ев-.
    \begin{exception}
        М\'{и}лостивый, юр\'{о}дивый.
    \end{exception}
\end{rrule}

\begin{rrule}[Правописание суффиксов -чив-/-лив-]
    В суффиксах -чив-/-лив- пишется «и»; суффиксов -чев-/-лев- нет.
    \begin{example}
        Участливый, улыбчивый.
    \end{example}
\end{rrule}

\begin{rrule}[Правописание суффиксов -ов-/-оват-/-овит-/-ев-/-еват-/-евит-]
    После твёрдых согласных (кроме «ц») пишется буква «о», а после мягких согласных, шипящих и «ц» -- буква «е».
    \begin{example}
        Деловой, угловатый, домовитый, речевой, рыжеватый, глянцевитый.
    \end{example}
\end{rrule}

\begin{rrule}[Правописание суффиксов -оньк-/-еньк-]
    Суффикс -оньк- в прилагательных пишется после «г», «к», «х»; в других случаях пишется -еньк-.
    \begin{example}
        Мяконький, плохонький, худенький, славненький.
    \end{example}
\end{rrule}

\subsubsection{Правописание суффиксов глаголов}

\begin{rrule}[Правописание суффиксов -ова-/-ева-/-ыва-/-ива-]
    \hfill
    \begin{enumerate}
        \item В неопределённой форме и форме прошедшего времени глагола пишется суффикс -ова- (-ева-), если глагол в форме 1-го лица единственного числа настоящего времени оканчивается на -ую (-юю).
        \begin{example}
            Исповедовать -- исповедую; ночевать -- ночую.
        \end{example}
        \item Если же в форме 1-го лица единственного числа настоящего времени глагол оканчивается на -ываю (-иваю), пишется суффикс -ыва- (-ива-).
        \begin{example}
            Прокладывать -- прокладываю; разучивать -- разучиваю.
        \end{example}
        \item Глаголы с данными суффиксами надо отличать от глаголов несовершенного вида с ударным суффиксом -ва-, перед которым пишется та безударная гласная, которая стоит под ударением в корне глагола совершенного вида (НО из-за большого количества исключений, можно запомнить, что если -ва- ударный, то перед ним пишется «е»).
        \begin{example}
            Задевать -- задеть; развивать -- развить; запивать -- запить; запевать -- запеть, НО затмевать -- затмить; продлевать -- продлить; застревать -- застрять; разевать -- разинуть и так далее.
        \end{example}
    \end{enumerate}
\end{rrule}

\subsubsection{Правописание суффиксов причастий}

\begin{rrule}[Правописание суффиксов -ущ-/-ющ-/-ащ-/-ящ-]
    В действительных причастиях настоящего времени в \romannumeralcaps{1} спряжении пишется суффикс -ущ- (-ющ-), а во \romannumeralcaps{2} спряжении – суффикс -ащ- (-ящ-).
    \begin{example}
        Читающий, держащий.
    \end{example}
\end{rrule}

\begin{rrule}[Правописание суффиксов -ем-/-ом-/-им-]
    В страдательных причастиях настоящего времени в \romannumeralcaps{1} спряжении пишется суффикс -ем- (-ом-), а во \romannumeralcaps{2} спряжении – суффикс -им-.
    \begin{example}
        Читаемый, хранимый.
    \end{example}
\end{rrule}

\begin{rrule}[Правописание суффиксов -вш-/-ш-]
    В действительных причастиях прошедшего времени, образованных от глаголов с основой на гласную пишется суффикс -вш-, а на согласную -- суффикс -ш-.
    \begin{example}
        Писавший -- писать; нёсший -- нести.
    \end{example}
\end{rrule}

\begin{rrule}[Правописание суффиксов -нн-/-енн-]
    \hfill
    \begin{enumerate}
        \item В страдательных причастиях прошедшего времени пишется суффикс -нн-, если оно образовано от глагола на -ать/-ять.
        \begin{example}
            Прочитанный -- прочитать; сделанный -- сделать.
        \end{example}
        \item Суффикс -енн- пишется, если глагол на -ить/-еть/-ти/-чь.
        \begin{example}
            Увиденный -- увидеть; поклеенный -- поклеить.
        \end{example}
    \end{enumerate}
\end{rrule}

\subsection{Слитное и раздельное написание НЕ с различными частями речи}

\begin{algorithm}[Правописание НЕ с существительными]
    \hfill
    \begin{center}
        \begin{tikzpicture}
            \def\gap {-0.7};
            \draw (-.25,0) |- (-7,\gap) |- +(0,\gap) coordinate (A);
            \draw (.25,0) |- (7,\gap) |- +(0,\gap) coordinate (B);
            \draw (A) |- +(7,\gap) coordinate (C);
            \draw (C) +(-.25,0) |- (-7,\gap*4) |- +(0,\gap) coordinate (D);
            \draw (C) +(.25,0) |- (5,\gap*4) |- +(0,\gap) coordinate (E);
            \draw (E) |- +(-5,\gap) coordinate (F);
            \draw (F) +(-.25,0) |- (-5,\gap*7) |- +(0,\gap) coordinate (G);
            \draw (F) +(.25,0) |- (5,\gap*7) |- +(0,\gap) coordinate (H);
            \draw (G) -- +(0,-1) coordinate (I);
            \draw (H) -- +(0,-1) coordinate (J);
            \draw (D) |- (I);
            \draw (B) |- (J);
            \draw (0,0) node[fill=white, draw] {Употребляется без НЕ};
            \draw (C) node[fill=white, draw] {Есть противопоставление с союзом «а»};
            \draw (F) node[fill=white, draw] {Можно заменить синонимом без НЕ};
            \draw (I) node[fill=white, draw] {Раздельно};
            \draw (J) node[fill=white, draw] {Слитно};
            \foreach \point in {(A), (D), (H)}{
				\draw \point node[fill=white, draw] {Да};
			}
            \foreach \point in {(B), (E), (G)}{
				\draw \point node[fill=white, draw] {Нет};
			}
        \end{tikzpicture}
    \end{center}
\end{algorithm}

\begin{algorithm}[Правописание НЕ с прилагательными и наречиями на -о/-е]
    \hfill
    \begin{center}
        \begin{tikzpicture}
            \def\gap {-0.7};
            \draw (-.25,0) |- (-7,\gap) |- +(0,\gap) coordinate (A);
            \draw (.25,0) |- (7,\gap) |- +(0,\gap) coordinate (B);
            \draw (A) |- +(7,\gap) coordinate (C);
            \draw (C) +(-.25,0) |- (-7,\gap*4) |- +(0,\gap) coordinate (D);
            \draw (C) +(.25,0) |- (5,\gap*4) |- +(0,\gap) coordinate (E);
            \draw (E) |- +(-5,\gap) coordinate (F);
            \draw (F) +(-.25,0) |- (-5,\gap*7) |- +(0,\gap) coordinate (G);
            \draw (F) +(.25,0) |- (5,\gap*7) |- +(0,\gap) coordinate (H);
            \draw (H) |- +(-5,\gap) coordinate (I);
            \draw (I) +(-.25,0) |- (-3,\gap*10) |- +(0,\gap) coordinate (J);
            \draw (I) +(.25,0) |- (3,\gap*10) |- +(0,\gap) coordinate (K);
            \draw (J) -- +(0,-1) coordinate (L);
            \draw (K) -- +(0,-1) coordinate (M);
            \draw (D) |- (L);
            \draw (G) |- (L);
            \draw (B) |- (M);
            \draw (0,0) node[fill=white, draw] {Употребляется без НЕ};
            \draw (C) node[fill=white, draw] {Есть противопоставление с союзом «а»};
            \draw (F) node[fill=white, draw] {Есть слова-магниты};
            \draw (I) node[fill=white, draw] {Можно заменить синонимом без НЕ};
            \draw (L) node[fill=white, draw] {Раздельно};
            \draw (M) node[fill=white, draw] {Слитно};
            \foreach \point in {(A), (D), (G), (K)}{
				\draw \point node[fill=white, draw] {Да};
			}
            \foreach \point in {(B), (E), (H), (J)}{
				\draw \point node[fill=white, draw] {Нет};
			}
        \end{tikzpicture}
    \end{center}
\end{algorithm}
\begin{example}[Слова-магниты]
    Отнюдь не, далеко не, вовсе не, ничуть не, а также отрицательные местоимения.
\end{example}
\begin{exception}
    \hfill
    \begin{enumerate}
        \item НЕ с прилагательными пишется раздельно, если они стоят в сравнительной форме.
        \item НЕ с прилагательными пишется раздельно, если они обозначают вкус или цвет.
    \end{enumerate}
\end{exception}

\begin{rrule}[Правописание НЕ с краткими прилагательными]
    \hfill
    \begin{enumerate}
        \item НЕ с краткими прилагательными пишется как в полной форме прилагательного.
        \item НЕ с краткими прилагательными пишется раздельно, если они не употребляются в полной форме.
        \begin{example}
            Не рад, не готов, не должен, не обязан, не прав, не намерен, не способен.
        \end{example}
    \end{enumerate}
\end{rrule}

\begin{algorithm}[Правописание НЕ с причастиями]
    \hfill
    \begin{center}
        \begin{tikzpicture}
            \def\gap {-0.7};
            \draw (-.25,0) |- (-7,\gap) |- +(0,\gap) coordinate (A);
            \draw (.25,0) |- (7,\gap) |- +(0,\gap) coordinate (B);
            \draw (A) |- +(7,\gap) coordinate (C);
            \draw (C) +(-.25,0) |- (-7,\gap*4) |- +(0,\gap) coordinate (D);
            \draw (C) +(.25,0) |- (5,\gap*4) |- +(0,\gap) coordinate (E);
            \draw (E) |- +(-5,\gap) coordinate (F);
            \draw (F) +(-.25,0) |- (-5,\gap*7) |- +(0,\gap) coordinate (G);
            \draw (F) +(.25,0) |- (5,\gap*7) |- +(0,\gap) coordinate (H);
            \draw (H) |- +(-5,\gap) coordinate (I);
            \draw (I) +(-.25,0) |- (-3,\gap*10) |- +(0,\gap) coordinate (J);
            \draw (I) +(.25,0) |- (3,\gap*10) |- +(0,\gap) coordinate (K);
            \draw (J) -- +(0,-1) coordinate (L);
            \draw (K) -- +(0,-1) coordinate (M);
            \draw (D) |- (L);
            \draw (G) |- (L);
            \draw (B) |- (M);
            \draw (0,0) node[fill=white, draw] {Употребляется без НЕ};
            \draw (C) node[fill=white, draw] {Краткое};
            \draw (F) node[fill=white, draw] {Есть противопоставление с союзом «а»};
            \draw (I) node[fill=white, draw] {Есть зависимые слова};
            \draw (L) node[fill=white, draw] {Раздельно};
            \draw (M) node[fill=white, draw] {Слитно};
            \foreach \point in {(A), (D), (G), (J)}{
				\draw \point node[fill=white, draw] {Да};
			}
            \foreach \point in {(B), (E), (H), (K)}{
				\draw \point node[fill=white, draw] {Нет};
			}
        \end{tikzpicture}
    \end{center}
\end{algorithm}

\begin{rrule}[Правописание НЕ с глаголами и деепричастями]
    НЕ с глаголами и деепричастиями пишется слитно только в тех случаях, когда они не употребляются без НЕ или начинаются с приставки недо-.
    \begin{example}
        Нанавидеть, \rsPrefix{недо}есть, негодуя, \rsPrefix{недо}любливая.
    \end{example}
\end{rrule}

\subsection{Слитное, дефисное и раздельное написание слов}

\begin{rrule}[Правописание союзов «тоже», «также»]
    Союзы «тоже», «также» пишутся слитно, если они взаимозаменяемы, а также их можно заменить союзом «и».
    \begin{example}
        Он \textsl{тоже} студент; он \textsl{также} студент; \textsl{и} он студент, НО то же... что; то же самое; одно и то же; так же, как.
    \end{example}
\end{rrule}

\begin{rrule}[Правописание союза «зато»]
    Союз «зато» пишется слитно, если он имеет значение противопоставления (можно заменить союзом «но»).
    \begin{example}
        Мал золотник, \textsl{зато} дорог.
    \end{example}
\end{rrule}

\begin{rrule}[Правописание союза «чтобы» («чтоб»)]
    Союз «чтобы» («чтоб») пишется слитно, если его можно заменить на «для того чтобы».
    \begin{example}
        Я пришёл, \textsl{чтобы} дать вам волю, НО во что бы то ни стало.
    \end{example}
\end{rrule}

\begin{rrule}[Правописание союзов «причём», «притом»]
    Союзы «причём», «притом» пишутся слитно, если они взаимозаменяемы, а также их можно заменить союзом «к тому же».
    \begin{example}
        Сын у меня красивый, \textsl{причём} умный; сын у меня красивый, \textsl{притом} умный, сын у меня красивый, \textsl{к тому же} умный, НО ни при чём.
    \end{example}
\end{rrule}

\begin{rrule}[Правописание союзов «оттого», «потому»]
    Союзы «причём», «притом» пишутся слитно, если они взаимозаменяемы.
    \begin{example}
        Мне грустно \textsl{потому}, что весело тебе; мне грустно \textsl{оттого}, что весело тебе.
    \end{example}
\end{rrule}

\begin{rrule}[Правописание наречий наречий]
    \hfill 
    \begin{enumerate}
        \item Наречие пишется слитно, если:
        \begin{enumerate}
            \item Оно образовано от собирательного числительного с приставкой «на» или «в».
            \begin{example}
                Надвое, вдвое, впятером, НО по двое, по трое.
            \end{example}
            \item Оно образовано от местоимения, полного прилагательного, начинающегося с согласной, или краткого прилагательного с приставкой.
            \begin{example}
                Вничью, впрочем, зачастую, досуха, набело.
            \end{example}
        \end{enumerate}
        \item Наречие пишется раздельно, если:
        \begin{enumerate}
            \item Оно образовано от существительного, сохранившего падежные формы, с предлогом.
            \begin{example}
                На корточках -- на корточки; за границу -- за границей.
            \end{example}
            \item Оно образовано от существительного с предлогом, между которыми можно вставить определение.
            \begin{example}
                В (полную) меру; на (полном) скаку; до (самой) смерти.
            \end{example}
            \item Оно образовано от предлога «в» и существительного или прилагательного, начинающегося с гласной.
            \begin{example}
                В обнимку, в упор, в ударе.
            \end{example}
            \item Оно образовано от предлога «в» или «на» и существительного в предложном падеже.
            \begin{example}
                На днях, на радостях, в потёмках.
            \end{example} 
        \end{enumerate}
        \item Наречие пишется через дефис, если:
        \begin{enumerate}
            \item Оно образовано от двух одинаковых или близких по смыслу слов.
            \begin{example}
                Нежданно-негаданно, красным-красно.
            \end{example}
            \item Оно образовано с помощью приставки по- и суффикса -ому- (-ему-) или -и-.
            \begin{example}
                По-хорошему, по-городскому, по-русски.
            \end{example}
            \begin{exception}
                Потому, поэтому, почему, посему.
            \end{exception}
            \item Оно образовано с помощью приставки во- (в-) и суффикса -ых- (-их-).
            \begin{example}
                Во-первых, в-третьих.
            \end{example}
        \end{enumerate}
    \end{enumerate}
\end{rrule}

\begin{rrule}[Дефисное написание существительных] Имя существительное пишется через дефис, если:
    \begin{enumerate}
        \item Оно образовано из двух самостоятельных слов.
        \begin{example}
            Диван-кровать, кресло-качалка.
        \end{example}
        \item Оно имеет пол- и вторую часть, начинающуюся с «л», гласной или заглавной буквы.
        \begin{example}
            Пол-Сибири, пол-яблока, пол-листа.
        \end{example}
        \item Оно имеет иноязычные элементы вице-, экс-, унтер-, обер-.
        \begin{example}
            Вице-президент, экс-супруг.
        \end{example}
        \item Оно обозначает стороны света, некоторые термины, а также сложные единицы измерения.
        \begin{example}
            Северо-восток, юго-запад, динамо-машина, премьер-министр, тонно-километр.
        \end{example}
    \end{enumerate}
\end{rrule}

\begin{rrule}[Дефисное написание прилагательных] Имя прилагательное пишется через дефис, если:
    \begin{enumerate}
        \item Оно образовано от независимых слов (между которыми можно поставить союз «и»).
        \begin{example}
            Финско-русский, англо-немецкий.
        \end{example}
        \item Оно образовано от существительных, которые пишутся через дефис.
        \item Оно обозначает вкус или цвет.
        \item Его первая часть оканчивается на -ико.
        \begin{example}
            Физико-математический.
        \end{example}
    \end{enumerate}
\end{rrule}

\begin{rrule}[Раздельное и дефисное написание неопределённых местоимений и наречий]
    С приставкой кое- и суффиксами -то, -либо, -нибудь неопределённые местоимения и наречия пишутся через дефис, а иначе -- раздельно.
    \begin{example}
        Кое-кому, что-то, кто-нибудь, кем-либо.
    \end{example}
\end{rrule}

\subsection{Правописание Н и НН в различных частях речи}

\begin{rrule}[Правописание Н и НН в отымённых прилагательных]
    \hfill
    \begin{enumerate}
        \item НН пишется, если основа оканчивается на «н» и есть суффикс -н-.
        \begin{example}
            \rsRoot{\rsBase{картон}}\rsSuffix{н}ый, \rsRoot{\rsBase{лун}}\rsSuffix{н}ый, \rsRoot{\rsBase{камен}}\rsSuffix{н}ый.
        \end{example}
        \item НН пишется, когда есть суффикс -онн- или -енн-.
        \begin{example}
            Мысл\rsSuffix{енн}ый, листв\rsSuffix{енн}ый, революци\rsSuffix{онн}ый, позици\rsSuffix{онн}ый.
        \end{example}
        \begin{exception}
            Ветреный (человек).
        \end{exception}
        \item Н пишется, когда есть суффикс -ин-, -ан- или -ян-.
        \begin{example}
            Льв\rsSuffix{ин}ый, песч\rsSuffix{ан}ый, трав\rsSuffix{ян}ой.
        \end{example}
        \begin{exception}
            Стеклянный, оловянный, деревянный.
        \end{exception}
        \item В кратких прилагательных пишется столько же «н», сколько в полной форме.
    \end{enumerate}
\end{rrule}

\begin{algorithm}[Правописание Н и НН в причастиях]
    \hfill
    \begin{center}
        \begin{tikzpicture}
            \def\gap {-0.7};
            \draw (-.25,0) |- (-8,\gap) |- +(0,\gap) coordinate (A);
            \draw (.25,0) |- (8,\gap) |- +(0,\gap) coordinate (B);
            \draw (B) |- +(-8,\gap) coordinate (C);
            \draw (C) +(-.25,0) |- (-6,\gap*4) |- +(0,\gap) coordinate (D);
            \draw (C) +(.25,0) |- (8,\gap*4) |- +(0,\gap) coordinate (E);
            \draw (D) |- +(6,\gap) coordinate (F);
            \draw (F) +(-.25,0) |- (-6,\gap*7) |- +(0,\gap) coordinate (G);
            \draw (F) +(.25,0) |- (6,\gap*7) |- +(0,\gap) coordinate (H);
            \draw (G) |- +(6,\gap) coordinate (I);
            \draw (I) +(-.25,0) |- (-6,\gap*10) |- +(0,\gap) coordinate (J);
            \draw (I) +(.25,0) |- (4,\gap*10) |- +(0,\gap) coordinate (K);
            \draw (J) |- +(6,\gap) coordinate (L);
            \draw (L) +(-.25,0) |- (-2,\gap*13-.3) |- +(0,\gap) coordinate (M);
            \draw (L) +(.25,0) |- (2,\gap*13 -.3) |- +(0,\gap) coordinate (N);
            \draw (M) -- +(0,-1) coordinate (O);
            \draw (N) -- +(0,-1) coordinate (P);
            \draw (A) |- (O);
            \draw (E) |- (P);
            \draw (H) |- (P);
            \draw (K) |- (P);
            \draw (0,0) node[fill=white, draw] {Краткое};
            \draw (C) node[fill=white, draw] {Есть приставки кроме не-, свеже-, младо-};
            \draw (F) node[fill=white, draw] {Есть зависимые слова};
            \draw (I) node[fill=white, draw] {Оканчивается на -ованный/-ёванный};
            \draw (L) node[fill=white, draw, align=center] {Образовано от бесприставочного\\глагола совершенного вида};
            \draw (O) node[fill=white, draw] {Н};
            \draw (P) node[fill=white, draw] {НН};
            \foreach \point in {(A), (E), (H), (K), (N)}{
				\draw \point node[fill=white, draw] {Да};
			}
            \foreach \point in {(B), (D), (G), (J), (M)}{
				\draw \point node[fill=white, draw] {Нет};
			}
        \end{tikzpicture}
    \end{center}
\end{algorithm}
\begin{exception}
    Кованый, жёваный, клёваный, посажёный, названый (брат).
\end{exception}

\section{Пунктуация}

\begin{rrule}[Обособленные определения]
    \hfill 
    \begin{enumerate}
        \item Определительный оборот (причастный оборот или прилагательное с зависимыми словами) в постпозиции (после определяемого слова) обособляется.
        \begin{example}
            Сказки, \rsAttribute{\textsl{прочитанные в детстве}}, помнятся целую жизнь. Недвижны стояли леса, \rsAttribute{полные мрака}.
        \end{example}
        \item Согласованное определение (определительный оборот или одиночное причастие) в препозиции (перед определяемым словом) обособляется, если оно:
        \begin{enumerate}
            \item Имеет добавочное обстоятельственное значение причины, времени или уступки.
            \begin{example}
                \rsAttribute{\textsl{Охваченный дымом}}, город погрузился в панику.
            \end{example}
            \item Относится к личному местоимению.
            \begin{example}
                \rsAttribute{\textsl{Напуганный внезапным зовом}}, я упустил из рук чашку.
            \end{example}
            \item Отделено от определяемого слова другими членами предложения.
            \begin{example}
                \rsAttribute{\textsl{Всё еще незаметный}}, на вершинах сосен занимался рассвет.
            \end{example}
        \end{enumerate}
        \item Несогласованное определение обособляется, если оно:
        \begin{enumerate}
            \item Относится к личному местоимению или имени собственному.
            \begin{example}
                Он, \rsAttribute{\textsl{с его непоколебимостью и решительностью в речи}}, быстро убедил присутствующих.
            \end{example}
            \item Выражено простой формой сравнительной степени прилагательного.
            \begin{example}
                Голубое, \rsAttribute{\textsl{светлее летнего неба}}, пальто смотрелось ярко на фоне серого пейзажа.
            \end{example}
        \end{enumerate}
    \end{enumerate}
\end{rrule}

\begin{rrule}[Обособленные приложения]
    \hfill 
    \begin{enumerate}
        \item Приложение обособляется, если оно:
        \begin{enumerate}
            \item Выражено нарицательным с зависимыми словами.
            \begin{example}
                А мать, \textsl{родом какая-то книжна с восточной кровью}, страдала чем-то вроде чёрной меланхолии.
            \end{example}
            \item Относится к имени собственному или личному местоимению.
            \begin{example}
                \textsl{Упрямец во всём}, Илья Матвеевич оставался упрямцем и в учении. Бедняжка, он вовсе её не заслуживает.
            \end{example}
            \item Выражено именем собственным.
            \begin{example}
                В разговор изредка вставляет слово Любина тётка, \textsl{Ксения Фроловна Горина}.
            \end{example}
            \item Выступает в роли пояснительного слова.
            \begin{example}
                Ухаживала за мной одна девушка, \textsl{полька}.
            \end{example}
        \end{enumerate}
        \item Тире вместо запятой при обособлении приложения ставится, если:
        \begin{enumerate}
            \item Оно находится в конце предложения.
            \begin{example}
                Семья хозяина состояла из жены и двух детей-подростков -- \textsl{мальчика и девочки}.
            \end{example}
            \item Его можно заменить на «а именно», «то есть».
            \begin{example}
                В дальнем углу светилось жёлтое пятно -- \textsl{огонь в окне квартиры Серафимы}...
            \end{example}
        \end{enumerate}
    \end{enumerate}
\end{rrule}

\begin{rrule}[Тире в неполных предложениях]
    Тире в неполных предложениях ставится при параллелизме конструкций.
    \begin{example}
        Во всех окнах — любопытные, на крышах — мальчишки. Первый план кажется выгоднее, а второй -- гуманнее.
    \end{example}
\end{rrule}

\begin{rrule}[Тире между подлежащим и сказуемым]
    Тире между подлежащим и сказуемым ставится, если они выражены именами существительными в именительном падеже с нулевой связкой.
    \begin{example}
        \rsNoun{\rsSubject{Тамань}}[им.п.] -- самый скверный \rsPredicate{\rsNoun{городишко}[им.п.]} из всех приморских городов России.
    \end{example}
\end{rrule}

\begin{rrule}[Пунктуация при междометиях]
    Междометия выделяются запятыми или восклицательным знаком, НО их нужно отличать от частиц.
    \begin{example}
        \textsl{Ах}, одно мне только нехорошо: хочу поглядеть на тебя хоть одним глазом, да боюсь... НО \textsl{Ах} ты, горе, горе моё!
    \end{example}
\end{rrule}

\begin{rrule}[Пунктуация при обращениях]
    Обращения выделяются запятыми; личные местоимения не являются обращениями.
    \begin{example}
        Что тебе надобно, \textsl{старче}? Если \textsl{вы} любите осень, то знаете, что осенью вода в реках приобретает от холода яркий синий цвет.
    \end{example}
\end{rrule}

\subsection{Пунктуация при однородных членах предложения}

\begin{algorithm}[Определение однородных членов предложения]
    \hfill 
    \begin{enumerate}
        \item Однородные члены предложения отвечают на один вопрос и относятся к одному слову.
        \item Однородные члены предложения связаны сочинительной или бессоюзной связью.
        \item Однородные определения дают характеристику предметов с одной стороны.
    \end{enumerate}
\end{algorithm}

\begin{rrule}[Пунктуация при однородных членах предложения]
    \hfill 
    \begin{enumerate}
        \item Запятая при однородных членах предложения ставится, если:
        \begin{enumerate}
            \item Есть повторяющиеся соединительные или разделительные союзы.
            \begin{example}
                Я выполнил задание по русскому, \textsl{и} по английскому, \textsl{и} по математике.
            \end{example}
            \item Есть бессоюзная связь.
            \begin{example}
                В парке было свежо, тихо, спокойно. 
            \end{example}
            \item Однородные члены связаны противительным союзом в значении «но».
            \begin{example}
                Эти конфеты дешевые, \textsl{но} очень вкусные. 
            \end{example}
        \end{enumerate}
        \item Запятая при однородных членах предложения НЕ ставится, если:
        \begin{enumerate}
            \item Между двумя однородными членами есть не противительный союз.
            \begin{example}
                Из всех занятий на свете больше всего я люблю читать \textsl{и} танцевать. 
            \end{example}
            \item Группы однородных членов разбиты на пары.
            \begin{example}
                В магазине мы купили все к столу: фрукты и овощи, рыбу и мясо, конфеты и печенье. 
            \end{example}
        \end{enumerate}
        \item Пунктуация при обобщающем слове:
        \begin{enumerate}
            \item Тире ставится, если обобщающее слово стоит после однородных членов или если после однородных членов предложение продолжается.
            \begin{example}
                Одежда, обувь, документы — всё необходимое уже лежит в чемодане. Везде: в гостинной, на кухне, в ванной — лежали ее вещи.
            \end{example}
            \item Двоеточие ставится, если обобщающее слово стоит перед однородными членами.
            \begin{example}
                В лесу было много разной живности: зверей, жуков и птиц.
            \end{example}
        \end{enumerate}
    \end{enumerate}
\end{rrule}

\begin{rrule}[Запятая при неоднородных членах предложения]
    Если в предложении после прилагательного идёт причастный оборот, между ними ставится запятая.
    \begin{example}
        Деревянная, запирающаяся на щеколду дверь, НО запирающаяся на щеколду деревянная дверь.
    \end{example}
\end{rrule}

\subsection{Пунктуация в сложных предложениях}

\subsubsection{Бессоюзные сложные предложения}

\begin{rrule}[Двоеточие в БСП]
    Двоеточие в БСП ставится, если:
    \begin{enumerate}
        \item Вторая часть предложения раскрывает содержание первой.
        \begin{example}
            Приближалась роковая минута: полет направлялся к воротам дворца.
        \end{example}
        \item Вторая часть предложения указывает основание или причину того, о чём говорится в первой части.
        \begin{example}
            Я глядел в огонь и боялся поднять голову: казалось, кто-то смотрит на меня из темноты.
        \end{example}
    \end{enumerate}
\end{rrule}

\begin{rrule}[Тире в БСП]
    Тире в БСП ставится, если:
    \begin{enumerate}
        \item Первое предложение обозначает время или условие совершения действия во второй части.
        \begin{example}
            Будешь летом в Ялте — непременно погуляй по Ботаническому саду.
        \end{example}
        \item Вторая часть предложения имеет значение следствия, результата, следующего из первой части.
        \begin{example}
            Фея дотронулась до тыквы волшебной палочкой — та превратилась в карету.
        \end{example}
        \item Между двумя частями предложения есть противопоставление или сравнение.
        \begin{example}
            Служить бы рад — прислуживаться тошно. Молвит слово — соловей поёт.
        \end{example}
    \end{enumerate}
\end{rrule}

\subsubsection{Сложносочинённые и сложноподчинённые предложения}

\begin{rrule}[Запятая в сложных предложениях]
    Запятая в сложносочинённых и сложноподчинённых предложениях НЕ ставится, если:
    \begin{enumerate}
        \item Есть общий второстепенный член, вводное слово, относящееся к обеим частям предложения или общее придаточное предложение.
        \begin{example}
            \rsNoun*{\textsl{У него}}[общ.вт.чл.] осунулось лицо и отяжелели веки. \rsNoun*{\textsl{Может быть}}[ввод.сл.], завтра будет хорошая погода и дождь не пойдёт. \rsAdverbial{\textsl{Когда часы пробили полночь}}, улица опустела и в последнем доме погас свет.
        \end{example}
        \item У предложений общая структура.
        \begin{enumerate}
            \item У неопределённо-личных предложений один и тот же производитель действия.
            \begin{example}
                Раньше погоду узнавали по телевизору и слушали передачи по радио.
            \end{example}
            \item У безличных предложений синонимы в обеих частях.
            \begin{example}
                \textsl{Нужно} скорее собирать вещи и \textsl{необходимо} позвонить в службу спасения.
            \end{example}
            \item Между назывными предложениями.
            \begin{example}
                Мороз и солнце!
            \end{example}
        \end{enumerate}
        \item В вопросительных, восклицательных или побудительных предложениях.
        \begin{example}
            Как правильно жить на свете и почему в мире так много зла? Как он любил её тогда и как любит по сей день! Пусть ещё явятся люди и пусть все спорят.
        \end{example}
    \end{enumerate}
\end{rrule}

\begin{rrule}[Запятая на стыке союзов]
    Запятая на стыке союзов (два подчинительных или сочинительный и подчинительный) ставится, если отсутствует вторая часть подчинительного союза (то, но, так). При этом запятые на границах частей предложения ставятся всегда.
    \begin{example}
        Любка смелая, \textsl{и, когда} бабушка зазевается, она часто уходит далеко от дома. Любка смелая, \textsl{и когда} бабушка зазевается, \textsl{то} она часто уходит далеко от дома.
    \end{example}
\end{rrule}

\section{Средства выразительности}

\subsection{Лексические средства выразительности}

\begin{definition}
    Метафора -- слово или выражение в переносном значении.
    \begin{example}
        Между тем луна начала \textsl{одеваться} тучами и на море поднялся туман.
    \end{example}
\end{definition}
\begin{definition}
    Развёрнутая метафора -- изображение цельной картины с помощью слов в переносном значении.
    \begin{example}
        Море играло маленькими волнами, рождая их, \textsl{украшая бахромой пены}, сталкивая друг с другом и разбивая в мелкую пыль.
    \end{example}
\end{definition}
\begin{definition}
    Эпитет -- образное, красочное определение.
    \begin{example}
        Вы видите, я играю в ваших глазах самую \textsl{жалкую} и \textsl{гадкую} роль.
    \end{example}
\end{definition}
\begin{definition}
    Олицетворение -- наделение человеческими чувствами неодушевлённых предметов или животных.
    \begin{example}
        Хрустит и \textsl{взвизгивает} гравий под колёсами.
    \end{example}
\end{definition}
\begin{definition}
    Сравнение -- сопоставление понятий с помощью сравнительного союза (как, точно, словно и др.) или формы творительного падежа.
    \begin{example}
        Она, \textsl{как змея}, скользнула между моими руками. По сугробам летит \textsl{стрелой} заяц...
    \end{example}
\end{definition}
\begin{definition}
    Метонимия -- замена слова или понятия другим словом, имеющим причинную связь с первым. Один из видов метонимии -- синекдоха -- обозначение большего через меньшее или наоборот.
    \begin{example}
        Большего всего береги и копи \textsl{копейку}...
    \end{example}
\end{definition}
\begin{definition}
    Гипербола -- образное преувеличение.
    \begin{example}
        Потому что я с Севера, что ли, что луна там \textsl{огромней в сто раз}...
    \end{example}
\end{definition}
\begin{definition}
    Литота -- образное преуменьшение.
    \begin{example}
       \textsl{ Ниже тоненькой былиночки} надо голову клонить.
    \end{example}
\end{definition}
\begin{definition}
    Ирония -- употребление слова или выражения в противоположном смысле.
    \begin{example}
        Отколе, \textsl{умная}, берёшь ты, голова?
    \end{example}
\end{definition}
\begin{definition}
    Перифраз (перифраза) -- замена слова описательным оборотом.
    \begin{example}
        Город на неве (Санкт-Петербург); царь зверей (лев).
    \end{example}
\end{definition}

\subsection{Синтаксические средства выразительности}

\begin{definition}
    Анафора -- одинаковое начало предложений или частей предложения.
    \begin{example}
        \textsl{Влюбиться} не значит любить. \textsl{Влюбиться} можно и ненавидя.
    \end{example}
\end{definition}
\begin{definition}
    Эпифора -- одинаковое окончание предложений или частей предложения.
    \begin{example}
        Яркий снег сиял \textsl{в долине}, -- снег растаял и ушёл; вешний знак блестит \textsl{в долине}, -- знак увянет и уйдёт.
    \end{example}
\end{definition}
\begin{definition}
    Инверсия -- нарушение порядка слов.
    \begin{example}
        \textsl{Роняет лес} багряный свой убор.
    \end{example}
\end{definition}
\begin{definition}
    Антитеза -- противопоставление.
    \begin{example}
        \textsl{Левая} сторона лица моего обращена к солнцу, и ей \textsl{тепло}, а \textsl{правой холодно}.
    \end{example}
\end{definition}
\begin{definition}
    Градация -- расположение слов или выражений в порядке возрастания или убывания значения.
    \begin{example}
        Взорвали, взрыли, смыли, смели.
    \end{example}
\end{definition}
\begin{definition}
    Парцелляция -- смысловое членение текста.
    \begin{example}
        Джинсы, твидовый пиджак и хорошая рубашка. Очень хорошая.
    \end{example}
\end{definition}
\begin{definition}
    Синтаксический параллелизм -- одинаковое построение фраз или предложений.
    \begin{example}
        В синем море волны плещут, в синем небе звёзды блещут.
    \end{example}
\end{definition}
\begin{definition}
    Риторический вопрос -- вопрос, не требующий ответа.
    \begin{example}
        А судьи кто?
    \end{example}
\end{definition}
\begin{definition}
    Риторическое восклицание -- восклицание, играющее роль усиления эмоционального восприятия.
    \begin{example}
        О как мне всё это противно!
    \end{example}
\end{definition}
\begin{definition}
    Вопросно-ответная форма -- форма изложения, в которой чередуются вопрос и ответ.
    \begin{example}
        Что любовь? Пустые грёзы, бред несбыточной мечты.
    \end{example}
\end{definition}
\begin{definition}
    Эллипсис -- пропуск слова.
    \begin{example}
        Татьяна (бежит) в лес, медведь (бежит) за нею.
    \end{example}
\end{definition}
\begin{definition}
    Лексический повтор -- намеренное повторение одного слова.
    \begin{example}
        Зима \textsl{ждала}, \textsl{ждала} природа.
    \end{example}
\end{definition}
\begin{definition}
    Однородные члены предложения -- использование однородных членов для художественной выразительности.
    \begin{example}
        На лугу росли и \textsl{желтые}, и \textsl{красные}, и \textsl{голубые} цветы.
    \end{example}
\end{definition}
\begin{definition}
    Умолчание -- обрыв высказывания для передачи взволнованности и эмоциональности.
    \begin{example}
        Зверь припал... И из пасмурных недр кто-то спустит сейчас курки... Вдруг прыжок... И двуногого недруга разрывают на части клыки.
    \end{example}
\end{definition}
\begin{definition}
    Оксюморон -- сочетание несовместных по смыслу слов.
    \begin{example}
        О, как \textsl{мучительно} тобою \textsl{счастлив} я.
    \end{example}
\end{definition}

\end{document}