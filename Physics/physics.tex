%Формат файла
\documentclass[12pt]{article} 
\usepackage[paperheight=297mm,
   paperwidth=210mm,
   top=20mm,
   bottom=20mm,
   left=15mm,
   right=15mm]{geometry}


%Текст
\usepackage[fontsize=12pt]{fontsize}
\usepackage[russian]{babel}
\usepackage{color}
\usepackage{transparent}
\usepackage{amsthm}
\usepackage{multicol}
\usepackage{setspace}
\usepackage[e]{esvect}
\parindent=0cm

\theoremstyle{definition}
\newtheorem{theorem}{Теорема}[section]
\newtheorem{lemma}[theorem]{Лемма}
\newtheorem{definition}{Определение}
\newtheorem{law}[theorem]{Закон}
\newtheorem{formula}[theorem]{Формула}
\newtheorem{statement}[theorem]{Утверждение}
\newtheorem{consequence}{Следствие}[subsection]
\renewcommand\qedsymbol{$\blacksquare$}

%Картинки
\usepackage{graphicx}
\usepackage{wrapfig}
\usepackage{subcaption}
\usepackage{tikz}
\usepackage{tkz-euclide}
\usetikzlibrary {arrows.meta}
\usetikzlibrary{calc}
\usetikzlibrary{intersections}
\usepackage[european,siunitx]{circuitikz}


%Математика
\usepackage{amsmath}
\usepackage{amsfonts}
\usepackage{amssymb}

%Всякое
\usepackage{relsize}
\usepackage{enumerate}
\usepackage[inline]{enumitem}
\usepackage{hyperref}
\usepackage{multirow}
\usepackage{booktabs}
\usepackage{physics}
\usepackage{mhchem}

%Мат команды
\newcommand{\N}{\mathbb{N}}
\newcommand{\Z}{\mathbb{Z}}
\newcommand{\Q}{\mathbb{Q}}
\newcommand{\R}{\mathbb{R}}
\newcommand{\prob}{\mathbb{P}}
\newcommand{\verteq}{\rotatebox{90}{$\,=$}}
\newcommand{\equalto}[2]{\underset{\scriptstyle\overset{\mkern4mu\verteq}{#2}}{#1}}
\newcommand{\vertneq}{\rotatebox{90}{$\,\neq$}}
\newcommand{\notequalto}[2]{\underset{\scriptstyle\overset{\mkern4mu\vertneq}{#2}}{#1}}
\DeclareMathOperator{\const}{const}
\makeatletter
\newenvironment{sqcases}{%
  \matrix@check\sqcases\env@sqcases
}{%
  \endarray\right.%
}
\def\env@sqcases{%
  \let\@ifnextchar\new@ifnextchar
  \left\lbrack
  \def\arraystretch{1.2}%
  \array{@{}l@{\quad}l@{}}%
}
\makeatother

%Оглавление
\title{\textbf{Физика}}\date{}

\hypersetup{
    colorlinks,
    citecolor=black,
    filecolor=black,
    linkcolor=black,
    urlcolor=black
}

\begin{document}

\maketitle
\tableofcontents
\label{toc}
\newpage

\section{Обозначения в системе СИ}

\setlength{\columnsep}{1.5cm}
\begin{center}
    \begin{multicols}{2}
        \begin{itemize}[leftmargin=*, itemsep=0pt]
            \item Длина: [м] -- метр.
            \item Масса: [кг] -- килограмм.
            \item Время: [с] -- секунда.
            \item Сила электрического тока: [А] -- ампер.
            \item Температура: [$^\circ$C] -- градус Цельсия.
            \item Термодинамическая температура: [K] -- кельвин.
            \item Частота: [Гц] -- герц.
            \item Сила: [Н] -- ньютон.
            \item Энергия, механическая работа, количество теплоты: [Дж] -- джоуль.
            \item Мощность: [Вт] -- ватт.
            \item Давление: [Па] -- паскаль.
            \item Электрический заряд: [Кл] -- кулон.
            \item Разность потенциалов: [В] -- вольт.
            \item Сопротивление: [Ом] -- ом.
            \item Электроёмкость: [Ф] -- фарад.
            \item Магнитный поток: [Вб] -- вебер.
            \item Магнитная индукция: [Тл] -- тесла.
            \item Индуктивность: [Гн] -- генри.
        \end{itemize}
    \end{multicols}
\end{center}

\section{Теплопередача}

\begin{definition}
    Теплота – кинетическая часть внутренней энергии вещества, определяемая интенсивным хаотическим движением молекул и атомов, из которых это вещество состоит.
\end{definition}
\begin{definition}
    Количество теплоты – часть внутренней энергии, которую тело получает или теряет при теплопередаче.
\end{definition}
\begin{definition}
    Теплопередача – физический процесс передачи тепловой энергии от более горячего тела к более холодному.
\end{definition}

\begin{center}
    \begin{tikzpicture}
        \draw[white] (-7,0) -- (7,0);
        \draw (0,0) node[above] {\textbf{Теплопередача}};
        \draw[-{Stealth[scale = 1.5]}] (-0.5,0) -- (-0.5,-0.5) -- (-5,-0.5) -- (-5,-1);
        \draw[-{Stealth[scale = 1.5]}] (0.5,0) -- (0.5,-0.5) -- (5,-0.5) -- (5,-1);
        \draw[-{Stealth[scale = 1.5]}] (0,0) -- (0,-1);
        \draw (-5,-1) node[below] {Излучение};
        \draw (0,-1) node[below] {Конвекция};
        \draw (5,-1) node[below] {Теплопроводность};
        \draw[-{Stealth[scale = 1.5]}] (-0.25,-1.7) -- (-0.25,-2.2) -- (-3,-2.2) -- (-3,-2.7);
        \draw[-{Stealth[scale = 1.5]}] (0.25,-1.7) -- (0.25,-2.2) -- (3,-2.2) -- (3,-2.7);
        \draw (-3,-2.7) node[below] {Естественная};
        \draw (3,-2.7) node[below] {Вынужденная};
    \end{tikzpicture}
\end{center}

\begin{definition}
    Излучение — вид теплопередачи, при котором происходит передача внутренней энергии c помощью энергии электромагнитных волн.
\end{definition}
\begin{definition}
    Конвекция — вид теплопередачи, обусловленный потоками жидкости или газа.
\end{definition}
\begin{definition}
    Теплопроводность — передача внутренней энергии от одной части тела к другой или от одного тела к другому при контакте.
\end{definition}

\begin{formula}
    $$Q=c\cdot m\cdot \Delta t$$
    \begin{center}
        \begin{minipage}{12cm}
            $Q$ [Дж] -- количество теплоты, $c$ $\left[\frac{\text{Дж}}{\text{кг}\cdot {^\circ \text{C}}}\right]$ -- удельная теплоемкость, $m$ [кг] -- масса вещества, $\Delta t$ $[^\circ$C] -- разность температур.
        \end{minipage}
    \end{center}
\end{formula}

\subsection{Агрегатное состояние}

\begin{definition}
    Агрегатное состояние вещества — физическое состояние вещества, зависящее от соответствующего сочетания температуры и давления.
\end{definition}
\begin{definition}
    Переход вещества из жидкого состояния в твердое называется кристаллизацией.
\end{definition}
\begin{definition}
    Переход вещества из жидкого состояния в газообразное называется парообразованием.
\end{definition}
\begin{definition}
    Переход вещества из твердого состояния в жидкое называется плавлением.
\end{definition}
\begin{definition}
    Переход вещества из твердого состояния в газообразное называется сублимацией.
\end{definition}
\begin{definition}
    Переход вещества из газообразного состояния в жидкое называется конденсацией.
\end{definition}
\begin{definition}
    Переход вещества из газообразного состояния в твердое называется десублимацией.
\end{definition}
\begin{definition}
    Насыщенный пар — пар, находящийся в динамическом равновесии со своей жидкостью.
\end{definition}
\begin{center}
    \begin{tikzpicture}
        \draw[-{Stealth[scale = 1.5]}] (-0.5,0) -- (11,0) node[right] {Время [мин]};
        \draw[-{Stealth[scale = 1.5]}] (0, -0.5) -- (0,7) node[above] {Температура [$^\circ$C]};
        \coordinate (O) at (0,0);
        \coordinate (t0) at (0,1);
        \coordinate (t1) at (0,4);
        \coordinate (t2) at (0,6);
        \draw (O) node[below left] {$O$};
        \draw (t0) node[left] {$t_0$};
        \draw (t1) node[left] {$t_1$};
        \draw (t2) node[left] {$t_2$};
        \coordinate (A) at (2,4);
        \coordinate (B) at (4,4);
        \coordinate (C) at (5,6);
        \coordinate (D) at (6,4);
        \coordinate (E) at (8,4);
        \coordinate (F) at (10,1);
        \tkzLabelSegment[above left](t0,A){$a$};
        \tkzLabelSegment[above](A,B){$b$};
        \tkzLabelSegment[above left](B,C){$c$};
        \tkzLabelSegment[above right](C,D){$d$};
        \tkzLabelSegment[above](D,E){$e$};
        \tkzLabelSegment[above right](E,F){$f$};
        \node[right, align=left] at (10,4.5) {
              $a$ -- нагрев\\
              $b$ -- кипение\\
              $c$ -- нагрев пара\\
              $d$ -- охлаждение пара\\
              $e$ -- конденсация\\
              $f$ -- охлаждение жидкости
          };
        \draw (t0) -- (A) -- (B) -- (C) -- (D) -- (E) -- (F);
        \foreach \point in {(O), (t0), (t1), (t2), (A), (B), (C), (D), (E), (F)}{
    \fill \point circle (1.8pt);
}
    \end{tikzpicture}
\end{center}

\subsection{Удельная теплота}

\begin{definition}
    Удельная теплота -- скалярная физическая величина, обозначающая количество теплоты, требуемое для смены агрегатного состояния единицы массы.
\end{definition}
\begin{formula}
    $$Q=q\cdot m$$
    \begin{center}
        \begin{minipage}{12cm}
            $Q$ [Дж] – количество теплоты, $q$ [Дж/кг] -- удельная теплота сгорания, $m$ [кг] – масса вещества.
        \end{minipage}
    \end{center}
\end{formula}
\begin{formula}
    $$Q=L\cdot m$$
    \begin{center}
        \begin{minipage}{12cm}
            $Q$ [Дж] – количество теплоты, $L$ [Дж/кг] -- удельная теплота парообразования, $m$ [кг] – масса вещества.
        \end{minipage}
    \end{center}
\end{formula}
\begin{formula}
    $$Q=\lambda\cdot m$$
    \begin{center}
        \begin{minipage}{12cm}
            $Q$ [Дж] – количество теплоты, $\lambda$ [Дж/кг] -- удельная теплота плавления, $m$ [кг] – масса вещества.
        \end{minipage}
    \end{center}
\end{formula}

\section{Электричество}

\begin{definition}
    Электрический ток -- упорядоченное движение заряженных частиц. Направление электрического тока определяется движением положительных зарядов.
\end{definition}
\begin{law}[Ома]
    $$I=\frac{U}{R}$$
    \begin{center}
        \begin{minipage}{12cm}
            \centering
            $I$ [А] – сила тока, $U$ [В] -- напряжение, $R$ [Ом] – сопротивление.
        \end{minipage}
    \end{center}
\end{law}

\subsection{Электрическое поле}

\begin{definition}
    Поле -- материальная среда, передающая воздействие тел друг на друга.
\end{definition}
\begin{definition}
    Электростатическое поле -- поле, передающее взаимодействие одного неподвижного электрического заряда на другой.
\end{definition}
\begin{definition}
    Электрическая сила – сила, с которой электрическое поле одного заряда действует на внесенный в него другой электрический заряд. Сила воздействия электрического поля на заряд уменьшается по мере удаления.
\end{definition}

\subsection{Источник тока}

\begin{definition}
    Источник тока -- устройство, в котором происходит преобразование какого-либо вида энергии в электрическую энергию.
\end{definition}

\begin{center}
    \begin{tikzpicture}
        \draw[white] (-8,0) -- (8,0);
        \draw (0,0) node[above] {\textbf{Источник тока}};
        \draw[-{Stealth[scale = 1.5]}] (-0.75,0) -- (-0.75,-0.5) -- (-6,-0.5) -- (-6,-1) node[below] {Механический};
        \draw[-{Stealth[scale = 1.5]}] (-0.25,0) -- (-0.25,-1) -- (-3,-1) -- (-3,-1.5) node[below] {Тепловой};
        \draw[-{Stealth[scale = 1.5]}] (0.25,0) -- (0.25,-1) -- (3,-1) -- (3,-1.5) node[below] {Световой};
        \draw[-{Stealth[scale = 1.5]}] (0.75,0) -- (0.75,-0.5) -- (6,-0.5) -- (6,-1) node[below] {Химический};
    \end{tikzpicture}
\end{center}

\subsection{Проводники}

\begin{definition}
    Проводники -- вещества, обладающие свободными носителями заряда. При помещении проводящего тела в электрическое поле, свободные носители заряда приходят в движение, возникает электрический ток, который существует до тех пор, пока поле внутри проводника отлично от нуля.\bigskip

    \textit{В изолированном теле носители заряда с течением времени распределяются таким образом, что создаваемое ими электрическое поле полностью компенсирует внешнее поле внутри проводника, а полное поле становится равным нулю.}
\end{definition}
\begin{definition}
    Диэлектрики -- вещества, в которых все носители заряда связаны в нейтральных молекулах. При помещении во внешнее электрическое поле диэлектрики поляризуются, что приводит к ослаблению поля внутри них.
\end{definition}

\begin{formula}
    $$A=U\cdot I\cdot t$$
    \begin{center}
        \begin{minipage}{11cm}
            $A$ [Дж] – работа тока, $U$ [В] -- напряжение, $I$ [А] – сила тока, $t$ [с] -- время.
        \end{minipage}
    \end{center}
\end{formula}
\begin{formula}
    $$P=U\cdot I$$
    \begin{center}
        \begin{minipage}{12cm}
            \centering
            $P$ [Вт] – мощность тока, $U$ [В] -- напряжение, $I$ [А] – сила тока.
        \end{minipage}
    \end{center}
\end{formula}
\begin{law}[Джоуля-Ленца]
    $$Q=I^2\cdot R\cdot t$$
    \begin{center}
        \begin{minipage}{11cm}
            $Q$ [Дж] – количество теплоты, $I$ [А] -- сила тока, $R$ [Ом] – сопротивление, $t$ [с] -- время.
        \end{minipage}
    \end{center}
\end{law}
\newsavebox{\circuita}
\sbox{\circuita}{
    \begin{circuitikz}
        \draw[white] (-8,0) -- (8,0);
        \draw (-7.5,0) to[R=$R_1$, o-*] (-5,0)
        to[R=$R_2$, *-o] (-2.5,0);
        \draw (2.5,0) to[short, o-*] (3.75,0) -- (3.75,0.5)
        to[R=$R_1$] (6.25,0.5) -- (6.25,0)
        to[short, *-o] (7.5, 0);
        \draw (3.75,0) -- (3.75,-0.5)
        to[R=$R_2$] (6.25,-0.5) -- (6.25,0);
    \end{circuitikz}
}
\spacing{1.7}
\begin{center}
    \begin{tikzpicture}
        \draw[white] (-8,0) -- (8,0);
        \draw (0,0) node[above] {\textbf{Соединение проводников}};
        \draw[-{Stealth[scale = 1.5]}] (-0.25,0) -- (-0.25,-0.5) -- (-5,-0.5) -- (-5,-1);
        \draw[-{Stealth[scale = 1.5]}] (0.25,0) -- (0.25,-0.5) -- (5,-0.5) -- (5,-1);
        \draw (-5,-1) node[below] {Последовательное};
        \draw (5,-1) node[below] {Параллельное};
        \node at (0,-2.75){\usebox{\circuita}};
    \end{tikzpicture}
\end{center}
\begin{center}
    \begin{tikzpicture}
        \node[align=center] at (-5,0) {
            $I=I_1=I_2$\\
            $U=U_1+U_2$\\
            $R=R_1+R_2$
        };
        \node[align=center] at (5,0) {
            $I=I_1+I_2$\\
            $U=U_1=U_2$\\
            $\dfrac{1}{R}=\dfrac{1}{R_1}+\dfrac{1}{R_2}$
        };
\end{tikzpicture}
\end{center}
\singlespacing

\begin{formula}
    $$R=\frac{\rho\cdot l}{S}$$
    \begin{center}
        \begin{minipage}{12cm}
            $R$ [Ом] – сопротивление, $\rho$ [Ом $\cdot$ м] -- удельное сопротивление, $l$ [м] – длина проводника, $S$ [м$^2$] -- площадь поперечного сечения проводника.
        \end{minipage}
    \end{center}
\end{formula}

\subsection{Напряженность электрического поля}

\begin{law}[Закон сохранения заряда]
    В замкнутой системе алгебраическая сумма зарядов остаётся постоянной:
    $$\sum_{i=1}^{n}q_i=\const$$
\end{law}

\begin{law}[Закон Кулона]
    $$\vv{F}=k\cdot \frac{|q_1|\cdot|q_2|}{\varepsilon\cdot r^2}$$
    \begin{center}
        \begin{minipage}{15cm}
            $\vv{F}$ [Н] -- сила взаимодействия зарядов, $k$ -- постоянная Кулона $\left(\approx 9\cdot 10^9\,\frac{\text{Н}\cdot\text{м}^2}{\text{Кл}^2}\right)$, $q_1$ и $q_2$ [Кл] -- точечные заряды тел, $r$ [м] -- расстояние между зарядами, $\varepsilon$ -- относительная диэлектрическая проницаемость среды (равна 1 для воздуха).
        \end{minipage}
    \end{center}
\end{law}
\begin{formula}
    $$k=\frac{1}{4\pi\varepsilon_0}$$
    \begin{center}
        \begin{minipage}{13cm}
            $k$ $\left[\frac{\text{Н}\cdot\text{м}^2}{\text{Кл}^2}\right]$ – постоянная Кулона, $\varepsilon_0$ [Ф/м] -- электрическая постоянная.
        \end{minipage}
    \end{center}
\end{formula}
\begin{definition}
    Потенциалом электрического поля называется его характеристика, которая показывает, какой потенциальной энергией обладает единичный электрический заряд, помещенный в данную точку пространства.
\end{definition}
\begin{formula}
    $$\varphi=\frac{E}{q}$$
    \begin{center}
        \begin{minipage}{12cm}
            $\varphi$ [Дж/Кл] – потенциал, $E$ [Дж] -- энергия заряда, $q$ [Кл] -- величина заряда.
        \end{minipage}
    \end{center}
\end{formula}
\begin{definition}
    Разность потенциалов электрического поля (между точками 1 и 2) -- отношение работы электрического поля по перемещению пробного заряда из точки 1 в точку 2 к величине этого заряда.
    $$U_{12}=\varphi_1-\varphi_2=\frac{A_{12}}{q}$$
\end{definition}
\begin{definition}
    Напряженность -- отношение силы, с которой поле воздействует на точечный заряд к величине этого заряда.
\end{definition}
\begin{formula}
    $$\vv{E}=\frac{\vv{F}}{q}$$
    \begin{center}
        \begin{minipage}{12cm}
            $\vv{E}$ [Н/Кл] – напряженность поля, $\vv{F}$ [Н] -- сила воздействия поля, $q$ [Кл] – точечный заряд.
        \end{minipage}
    \end{center}
\end{formula}
\begin{law}[Принцип суперпозиции]
    Если в данной точке пространства электрическое поле создано несколькими зарядами и напряженность поля каждого заряда равна $\vv{E_1},\,\vv{E_2},\ldots$, то результирующая напряженность этого поля равна векторной сумме напряженностей составляющих его полей.
\end{law}

\subsection{Конденсаторы}

\begin{definition}
    Электроемкость -- физическая величина, характеризующая способность проводников накапливать заряд.
\end{definition}
\begin{formula}
    $$C=\frac{q}{U}$$
    \begin{center}
        \begin{minipage}{12cm}
            $C$ [Ф] -- электроемкость, $q$ [Кл] -- заряд пластины конденсатора, $U$ [В] – напряжение.
        \end{minipage}
    \end{center}
\end{formula}
\begin{formula}
    $$C=\frac{\varepsilon\cdot\varepsilon_0\cdot S}{d}$$
    \begin{center}
        \begin{minipage}{12cm}
            $C$ [Ф] -- электроемкость, $\varepsilon$ -- диэлектрическая проницаемость, $\varepsilon_0$ -- электрическая постоянная, $d$ [м] -- расстояние между пластинами конденсатора.
        \end{minipage}
    \end{center}
\end{formula}
\begin{formula}
    $$W=\dfrac{q\cdot U}{2}=\dfrac{C\cdot U^2}{2}=\dfrac{q^2}{2C}$$
    \begin{center}
        \begin{minipage}{12cm}
            $W$ [Дж] -- энергия заряженного конденсатора, $q$ [Кл] -- заряд пластины конденсатора, $U$ [В] -- разность потенциалов, $C$ [Ф] -- электроемкость.
        \end{minipage}
    \end{center}
\end{formula}

\begin{center}
    \begin{circuitikz}
        \draw (-7.5,0) to[capacitor, o-*] (-5,0)
        to[capacitor, *-o] (-2.5,0);
        \draw (2.5,0) to[short, o-*] (3.75,0) -- (3.75,0.75)
        to[capacitor] (6.25,0.75) -- (6.25,0)
        to[short, *-o] (7.5, 0);
        \draw (3.75,0) -- (3.75,-0.75)
        to[capacitor] (6.25,-0.75) -- (6.25,0);
        \node[align=center] at (-5,-2.25) {
            $\dfrac{1}{C}=\dfrac{1}{C_1}+\dfrac{1}{C_2}$
        };
        \node[align=center] at (5,-2.25) {
            $C=C_1+C_2$
        };
    \end{circuitikz}
\end{center}

\subsection{Магнитное поле}

\begin{definition}
    Магнитное поле -- особый вид материи, существующий вокруг любого проводника с током. Неподвижные электрические заряды создают электрическое поле, а подвижные -- электрическое и магнитное поля.
\end{definition}
\begin{definition}
    Вихревое поле -- поле, силовые линии которого замкнуты.
\end{definition}
\begin{definition}
    Магнитные линии -- это воображаемые линии, вдоль которых располагаются оси магнитных стрелок, помещённых в магнитное поле. Они показывают направление магнитного поля в каждой точке пространства.
\end{definition}
\begin{definition}
    Магнитная индукция -- векторная физическая величина, которая показывает, с какой силой магнитное поле действует на движущиеся заряженные частицы.
\end{definition}
\begin{center}
    \begin{tikzpicture}
        \draw[white] (-8,0) -- (8,0);
        \draw (0,0) node[above] {\textbf{Электромагнитная сила}};
        \draw[-{Stealth[scale = 1.5]}] (-0.25,0) -- (-0.25,-0.5) -- (-5,-0.5) -- (-5,-1);
        \draw[-{Stealth[scale = 1.5]}] (0.25,0) -- (0.25,-0.5) -- (5,-0.5) -- (5,-1);
        \draw (-5,-1) node[below] {Сила Лоренца};
        \draw (5,-1) node[below] {Сила Ампера};
    \end{tikzpicture}
\end{center}
\begin{definition}
    Сила Ампера -- сила, с которой магнитное поле воздействует на проводник с током.
\end{definition}
\begin{formula}
    $$F_\text{А}=I B l\cdot \sin \alpha$$
    \begin{center}
        \begin{minipage}{12cm}
            $F_\text{А}$ [Н] -- сила Ампера, $I$ [А] -- сила тока, $B$ [Тл] -- магнитная индукция, $l$ [м] -- длина проводника, $\alpha$ - угол между проводником и линиями магнитной индукции.
        \end{minipage}
    \end{center}
\end{formula}
\begin{definition}
    Сила Лоренца -- сила, с которой магнитное поле действует на движущуюся заряженную частицу.
\end{definition}
\begin{formula}
    $$F_\text{Л}=q v B \cdot \sin \alpha$$
    \begin{center}
        \begin{minipage}{12cm}
            $F_\text{Л}$ [Н] -- сила Лоренца, $q$ [Кл] -- заряд частицы, $v$ [м/с] -- скорость частицы, $B$ [Тл] -- магнитная индукция, $\alpha$ -- угол между $\vv{v}$ и $\vv{B}$.
        \end{minipage}
    \end{center}
\end{formula}
\begin{law}[Правило правой руки]
    Если обхватить проводник правой рукой так, чтобы оттопыренный большой палец указывал направление тока, то остальные пальцы покажут направление огибающих проводник линий магнитной индукции поля, создаваемого этим током, а значит и направление вектора магнитной индукции, направленного везде по касательной к этим линиям.\bigskip

    \textit{Иными словами, если ток направлен от наблюдателя, линии магнитной индукции направлены по часовой стрелке.}
\end{law}
\begin{law}[Правило левой руки]
    Если расположить ладонь левой руки так, чтобы линии индукции магнитного поля входили во внутреннюю сторону ладони, перпендикулярно к ней, а четыре пальца направлены по току, то отставленный на 90° большой палец укажет направление силы, действующей со стороны магнитного поля на проводник с током.
\end{law}
\begin{definition}
    Поверхностная плотность заряда -- скалярная физическая величина, которая характеризует количество заряда на единицу площади поверхности.
    $$\sigma=\frac{q}{S}$$
    $\sigma$ [Кл/м$^2$] -- поверхностная плотность заряда, $q$ [Кл] -- заряд, $S$ [м$^2$] -- площадь поверхности.
\end{definition}
\begin{definition}
    Циклотронный (Ларморовский) радиус -- радиус кругового движения заряженной частицы в однородном магнитном поле.
    $$R=\frac{mv}{qB}$$
    $R$ [м] -- циклотронный радиус, $m$ [кг] -- масса заряженной частицы, $v$ [м/с] -- скорость частицы, $q$ [Кл] -- заряд частицы, $B$ [Тл] -- магнитная индукция.
\end{definition}
\begin{proof}
    Из второго закона Ньютона:
    $$m\cdot \frac{v^2}{R}=qvB\Longrightarrow R=\frac{mv}{qB}$$ 
\end{proof}
\begin{definition}
    Циклотронная (Ларморовская) частота -- частота обращения заряженной частицы в однородном магнитном поле.
    $$\nu=\frac{qB}{2\pi m}$$
    $\nu$ [Гц] -- циклотронная частота, $q$ [Кл] -- заряд частицы, $B$ [Тл] -- магнитная индукция, $m$ [кг] -- масса заряженной частицы.
\end{definition}
\begin{proof}
    По определению периода вращения:
    $$T=\frac{2\pi R}{v}=\frac{2\pi mv}{qBv}=\frac{2\pi m}{qB}=\frac{1}{\nu}\Longrightarrow \nu=\frac{qB}{2\pi m}$$
\end{proof}

\subsection{Индукционный ток}

\begin{definition}
    Магнитный поток -- мера общего магнитного поля, проходящего сквозь заданную площадь.
    $$\Phi=|\vv{B}\cdot \vv{S}|=B\cdot S\cdot |\cos\alpha|$$
    $\Phi$ [Вб] -- магнитный поток, $B$ [Тл] -- магнитная индукция, $S$ [м$^2$] -- площадь, $\alpha$ -- угол между $\vv{B}$ и $\vv{S}$ (вектор площади).
\end{definition}
\begin{formula}
    $$\Phi=N\cdot B\cdot S\cdot |\cos\alpha|$$
    \begin{center}
        \begin{minipage}{12cm}
            $\Phi$ [Вб] -- магнитный поток катушки, $N$ -- количество ветков катушки, $B$ [Тл] -- магнитная индукция, $S$ [м$^2$] -- площадь, $\alpha$ -- угол между $\vv{B}$ и $\vv{S}$.
        \end{minipage}
    \end{center}
\end{formula}
\begin{law}[Электромагнитной индукции]
    При всяком изменении магнитного потока через замкнутый контур в контуре возникает индукционный ток.
\end{law}
\begin{law}[Правило Ленца]
    Индукционный ток, возникающий в замкнутом контуре, всегда такого направления, что собственное магнитное поле этого тока препятствует изменению внешнего магнитного потока.
\end{law}
\begin{definition}
    ЭДС (электродвижущая сила) индукции -- это величина, которая возникает в замкнутом проводящем контуре при изменении магнитного потока через этот контур и вызывает протекание индукционного тока.
    $$E_\text{инд.}=-\frac{\Delta\Phi}{\Delta t}=\left|\frac{\Delta\Phi}{\Delta t}\right|$$
    $E_\text{инд.}$ [В] -- ЭДС индукции, $\Phi$ [Вб] -- магнитный поток, $t$ [с] -- время.
\end{definition}
\begin{definition}
    Эффект Холла — это возникновение в электрическом проводнике разности потенциалов (напряжения Холла) на краях образца, помещённого в поперечное магнитное поле, при протекании тока, перпендикулярного полю.
\end{definition}
\begin{definition}
    ЭДС самоиндукции – ЭДС, возникающая в катушке при изменении силы тока через неё.
\end{definition}
\begin{definition}
    Индуктивность катушки – физическая величина, характеризующая способность катушки накапливать энергию в магнитном поле и сопротивляться изменению протекающего через неё тока.
    $$L=\frac{\Phi}{I}$$
    $L$ [Гн] -- индуктивность катушки, $\Phi$ [Вб] -- магнитный поток, $I$ [А] -- сила тока.
\end{definition}
\begin{formula}
    $$U_L=L\cdot\frac{\Delta I}{\Delta t}$$
    \begin{center}
        \begin{minipage}{12cm}
            $U_L$ [В] -- напряжение катушки, $L$ [Гн] -- индуктивность катушки, $I$ [А] -- сила тока, $t$ [с] -- время.
        \end{minipage}
    \end{center}
\end{formula}
\begin{statement}[Свойства катушки]
    $ $\par\nobreak\ignorespaces
    \begin{itemize}
        \item В установившемся режиме ($I=\const$) $U_L=0$.
        \item Сила тока через катушку скачком не меняется ($I(t)$ -- непрерывная функция).
        \item При протекании тока в катушке запасается энергия магнитного поля.
    \end{itemize}
\end{statement}
\begin{formula}
    $$W_L=\frac{LI^2}{2}$$
    \begin{center}
        \begin{minipage}{12cm}
            $W_L$ [Дж] -- энергия катушки, $L$ [Гн] -- индуктивность катушки, $I$ [А] -- сила тока.
        \end{minipage}
    \end{center}
\end{formula}

\section{Оптика}

\begin{definition}
    Свет — электромагнитное излучение, воспринимаемое человеческим глазом.
\end{definition}
\begin{center}
    \begin{tikzpicture}
        \draw[white] (-8,0) -- (8,0);
        \draw (0,0) node[above] {\textbf{Источники света}};
        \draw[-{Stealth[scale = 1.5]}] (-0.25,0) -- (-0.25,-0.5) -- (-5,-0.5) -- (-5,-1);
        \draw[-{Stealth[scale = 1.5]}] (0.25,0) -- (0.25,-0.5) -- (5,-0.5) -- (5,-1);
        \draw (-5,-1) node[below] {Естественные};
        \draw (5,-1) node[below] {Исскуственные};
    \end{tikzpicture}
\end{center}
\begin{definition}
    Точечный источник света -- источник света, размерами которого можно пренебречь.
\end{definition}
\begin{definition}
    Световой луч -- линия, вдоль которой рассматривается свет.
\end{definition}
\begin{definition}
    Тень -- область пространства, куда не попадает свет.
\end{definition}
\begin{definition}
    Полутень -- слабоосвещенное пространство.
\end{definition}
\begin{definition}
    Плоское зеркало -- плоская поверхность, отражающая свет.
\end{definition}
\begin{definition}
    Абсолютный показатель преломления -- отношение скорости света в веществе к скорости света в вакууме.
\end{definition}
\begin{definition}
    Относительный показатель преломления -- отношение абсолютных показателей преломления двух сред.
\end{definition}
\begin{center}
    \begin{tikzpicture}
        \coordinate (O) at (0,0);
        \coordinate (A) at (-2.5,2);
        \coordinate (B) at (0,3);
        \coordinate (C) at (0,-3);
        \coordinate (D) at (1.5,-3);
        \coordinate (F) at (4,0);
        \draw (-4,0) -- (4,0);
        \draw (0,-3) -- (0,3);
        \draw (A) -- (O);
        \draw (D) -- (O);
        \fill (O) circle (1.8pt);
        \tkzMarkAngle[size=.65cm](B,O,A);
        \tkzLabelAngle(B,O,A){$\alpha$};
        \tkzMarkAngle[size=.65cm](C,O,D);
        \tkzMarkAngle[size=.55cm](C,O,D);
        \tkzLabelAngle(C,O,D){$\beta$};
        \tkzMarkRightAngle(F,O,B);
        \draw (-4,0) node[above right] {$n_1$};
        \draw (-4,0) node[below right] {$n_2$};
        \node[right, align=left] at (5,.75) {
              $\alpha$ -- угол падения\\
              $\beta$ -- угол преломления\\
              $n$ -- коэффициент преломления
        };
        \draw (6.25,-1) node[right] {$\dfrac{\sin \alpha}{\sin \beta}=n_{21}=\dfrac{n_2}{n_1}$};
    \end{tikzpicture}
\end{center}
\begin{definition}
    Действительное изображение -- изображение, находящееся на пересечении лучей, выходящих из источника света.
\end{definition}
\begin{definition}
    Мнимое изображение -- изображение, находящееся на пересечении продолжений лучей.
\end{definition}
\begin{formula}
    $$n=\left[\frac{360^\circ-\alpha}{\alpha}\right]$$
    \begin{center}
        \begin{minipage}{12cm}
            \centering
            $n$ -- количество отражений, $\alpha$ – угол падения.
        \end{minipage}
    \end{center}
\end{formula}

\subsection{Линзы}

\begin{center}
    \begin{tikzpicture}
        \draw[white] (-8.7,0) -- (8.7,0);
        \def\uplen{4.25}
        \def\downlen{2}
        \def\step{0.5}
        \def\texth{0.7}
        \draw (0,0) node[above] {\textbf{Линзы}};

        \draw[-{Stealth[scale = 1.5]}] (-\step / 2,0) -- (-\step / 2,-\step) -- (-\step - \uplen,-\step) -- (-\step-\uplen,-\step * 2);

        \draw[-{Stealth[scale = 1.5]}] (\step / 2,0) -- (\step / 2,-\step) -- (\step + \uplen,-\step) -- (\step+\uplen,-\step * 2);

        \draw (-\step-\uplen,-\step * 2) node[below] {Собирающие};
        \draw (\step+\uplen,-\step * 2) node[below] {Рассеивающие};

        \draw[-{Stealth[scale = 1.5]}] (-\step-\uplen-\step,-\step*2 - \texth) -- (-\step-\uplen-\step,-\step*3 - \texth) -- (-2*\step-\uplen-\downlen,-\step*3 - \texth) -- (-2*\step-\uplen-\downlen,-\step*4 - \texth);

        \draw[-{Stealth[scale = 1.5]}] (\step-\uplen-\step,-\step*2 - \texth) -- (\step-\uplen-\step,-\step*3 - \texth) -- (-\uplen+\downlen,-\step*3 - \texth) -- (-\uplen+\downlen,-\step*4 - \texth);

        \draw[-{Stealth[scale = 1.5]}] (-\uplen-\step,-\step*2 - \texth) -- (-\uplen-\step,-\step*4 - \texth * 2);

        \draw[-{Stealth[scale = 1.5]}] (-\step+\uplen+\step,-\step*2 - \texth) -- (-\step+\uplen+\step,-\step*3 - \texth) -- (\uplen-\downlen,-\step*3 - \texth) -- (\uplen-\downlen,-\step*4 - \texth);

        \draw[-{Stealth[scale = 1.5]}] (\step+\uplen+\step,-\step*2 - \texth) -- (\step+\uplen+\step,-\step*3 - \texth) -- (\uplen+\downlen+2*\step,-\step*3 - \texth) -- (\uplen+\downlen+2*\step,-\step*4 - \texth);

        \draw[-{Stealth[scale = 1.5]}] (\uplen+\step,-\step*2 - \texth) -- (\uplen+\step,-\step*4 - \texth * 2);

        \draw (-2*\step-\uplen-\downlen,-\step*4 - \texth) node[below] {Двояковыпуклые};
        \draw (-\uplen-\step,-\step*4 - \texth * 2) node[below] {Выпуклые};
        \draw (-\uplen+\downlen,-\step*4 - \texth) node[below] {Выпукло-вогнутые};

        \draw (\uplen-\downlen,-\step*4 - \texth) node[below] {Вогнуто-выпуклые};
        \draw (\uplen+\downlen+2*\step,-\step*4 - \texth) node[below] {Двояковогнутые};
        \draw (\uplen+\step,-\step*4 - \texth * 2) node[below] {Вогнутые};
    \end{tikzpicture}
\end{center}

\begin{definition}
    Оптический центр линзы – это точка, проходя через которую лучи не испытывают преломления.
\end{definition}
\begin{definition}
    Оптической осью называется любая прямая, проходящая через оптический центр линзы.
\end{definition}
\begin{definition}
    Главной оптической осью называется оптическая ось, перпендикулярная линзе.
\end{definition}
\begin{definition}
    Главным фокусом $F$ называется точка, в которой пересекаются лучи, падающие на линзу параллельно её главной оптической оси.
\end{definition}
\begin{definition}
    Фокусным расстоянием называется расстояние от оптического центра линзы до её фокуса.
\end{definition}
\begin{definition}
    Фокальной плоскостью называется плоскость, перпендикулярная главной оптической оси, проходящая через её главный фокус.
\end{definition}

\section{Механика}

\begin{definition}
    Материальная точка – тело, размерами которого можно пренебречь в рамках данной задачи. Тело можно считать материальной точкой, если его размеры много меньше пройденного расстояния или при поступательном движении.
\end{definition}
\begin{definition}
    Система отсчёта – совокупность тела отсчёта, системы координат и часов.
\end{definition}
\begin{definition}
    Траекторией называется линия, вдоль которой тело или материальная точка изменяет своё положение.
\end{definition}
\begin{definition}
    Путём называется длина траектории, пройденной телом.
\end{definition}
\begin{definition}
    Перемещением называется вектор, соединяющий начальную и конечную точки траектории.
\end{definition}

\subsection{Законы Ньютона}

\begin{definition}
    Инерциальными системами отсчёта называются системы отсчёта, в которых тела движутся равномерно или находятся в состоянии покоя, при одинаковых начальных условиях движутся одинаково, и изменение скорости тела происходит в результате действия на него других тел.
\end{definition}

\begin{law}[Первый закон Ньютона]
    Существуют такие инерциальные системы отсчета, в которых всякое тело сохраняет состояние покоя или равномерного прямолинейного движения до тех пор, пока действие других тел не заставит его изменить это состояние. Моделью является материальная точка, а явлением — состояние покоя или равномерного прямолинейного движения.
\end{law}
\begin{law}[Второй закон Ньютона]
    Под действием силы тело приобретает такое ускорение, что его произведение на массу тела равно действующей силе. Моделью является материальная точка, а явлением — движение с ускорением.
\end{law}
\begin{law}[Третий закон Ньютона]
    Силы, с которыми взаимодействующие тела действуют друг на друга, равны по модулю и направлены по одной прямой в противоположные стороны. Моделью является система двух материальных точек, а явлением — взаимодействие тел.
\end{law}

\subsection{Прямолинейное движение}

\begin{center}
    \begin{tikzpicture}
        \draw[white] (-8,0) -- (8,0);
        \draw (0,0) node[above] {Прямолинейное движение};
        \draw[-{Stealth[scale = 1.5]}] (-0.25,0) -- (-0.25,-0.5) -- (-5,-0.5) -- (-5,-1);
        \draw[-{Stealth[scale = 1.5]}] (0.25,0) -- (0.25,-0.5) -- (5,-0.5) -- (5,-1);
        \draw (-5,-1) node[below] {Равномерное};
        \draw (5,-1) node[below] {Равноускоренное};
        \draw (-5,-1.7) node[below] {$\vv{r}(t)=\vv{r_0}+\vv{v_0}t$};
        \draw (-5,-2.4) node[below] {$\vv{v}=\vv{v_0}$};
        \draw (5,-1.7) node[below] {$\vv{r}(t)=\vv{r_0}+\vv{v_0}t+\frac{\vv{a}t^2}{2}$};
        \draw (5,-2.4) node[below] {$\vv{v}(t)=\vv{v_0}+\vv{a}t$};
    \end{tikzpicture}
\end{center}

\subsection{Криволинейное движение}

\begin{definition}
    Центростремительное ускорение — компонента ускорения точки, характеризующая быстроту изменения направления вектора скорости для траектории с кривизной:
\end{definition}
\begin{definition}
    Радиус кривизны траектории - радиус окружности, по которой тело двигается в определенный промежуток времени при криволинейном движении.
\end{definition}
\begin{formula}
    $$R_\text{крив.}=\dfrac{v^2}{a_n}$$
    \begin{center}
        \begin{minipage}{12cm}
            $R_\text{крив.}$ [м] -- радиус кривизны траектории, $v$ [м/с] –- скорость тела, $a_n$ [м/с$^2$] -- полное ускорение тела.
        \end{minipage}
    \end{center}
\end{formula}


\begin{law}[Закон перемещения тела]
    $ $\par\nobreak\ignorespaces
    \begin{itemize}
        \item Равномерное прямолинейное движение: $\vv{r}(t)=\vv{r_0}+\vv{v_0}t$; $\vv{v}=\vv{v_0}$
        \item Равноускоренное прямолинейное движение: $\vv{r}(t)=\vv{r_0}+\vv{v_0}t+\dfrac{\vv{a}t^2}{2}$; $\vv{v}(t)=\vv{v_0}+\vv{a}t$
        \item Равномерное движение по окружности: $a_n=\dfrac{v^2}{R}$
        \item Неравномерное движение по окружности: $\vv{a}=\vv{a_\tau} + \vv{a_n};\,\,|\vv{a}|=\sqrt{a_n^2+a_\tau^2}$
    \end{itemize}
\end{law}



\subsection{Импульс и энергия}

\begin{definition}
    Импульс материальной точки – векторная физическая величина, являющаяся мерой механического движения тела.
\end{definition}
\begin{formula}
    $$\vv{p}=m\vv{v}$$
    \begin{center}
        \begin{minipage}{12cm}
            $\vv{p}$ [кг $\cdot$ м/с] -- импульс материальной точки, $m$ [кг] -- её масса, $\vv{v}$ [м/с] -- её скорость.
        \end{minipage}
    \end{center}
\end{formula}

\begin{definition}
    Импульсом системы материальных точек называется векторная величина, равная сумме импульсов всех материальных точек системы:
    $$\vv{p}_\text{сис.}=\sum_{i}m_i\vv{v_i}$$
\end{definition}
\begin{law}[Закон сохранения импульса]
    Сумма импульсов всех тел системы есть величина постоянная, если векторная сумма внешних сил, действующих на систему тел, равна нулю:
    $$\vv{F}_\text{внеш.}=0\Longleftrightarrow \vv{p}_\text{сис.}=\vv{\const}$$
\end{law}
\begin{proof}
    $$\Delta \vv{p}_\text{сис.}=\sum_{i}\Delta \vv{p_i}=\Delta t\cdot \sum_{i} \vv{F_i}\Longrightarrow \sum_{i} \vv{F_i} = 0 \Longleftrightarrow \Delta \vv{p}_\text{сис.} = 0 \Longleftrightarrow \vv{p}_\text{сис.}=\vv{\const}$$
\end{proof}

\begin{theorem}[Об изменении кинетической энергии]
    В инерциальной системе отсчёта для материальной точки работа всех сил, действующих на точку, равна изменению её кинетической энергии:
    $$A_{\text{всех сил}}=\Delta K$$
\end{theorem}
\begin{definition}
    Потенциальные силы -- силы, работа которых не зависит от траектории точки приложения этих сил и закона её движения, а целиком определяется начальным и конечным положениями данной точки.\bigskip

    \textit{Например: сила тяжести, сила упругости, сила электрического воздействия.}
\end{definition}
\begin{formula}
    $$A_{\text{пот.}}=-\Delta\Pi$$
    \begin{center}
        \begin{minipage}{12cm}
            $A_{\text{пот.}}$ [Дж] -- работа потенциальных сил, $\Delta\Pi$ [Дж] -- изменение потенциальной энергии.
        \end{minipage}
    \end{center}
\end{formula}
\begin{definition}
    Непотенциальные силы -- силы, работа которых зависит от траектории движения тела.\bigskip

    \textit{Например: сила трения.}
\end{definition}
\begin{law}[Закон сохранения энергии]
    В замкнутой системе тел, где действуют только потенциальные силы, полная механическая энергия остается постоянной:
    $$A_{\text{непот.}}=0\Longleftrightarrow E=\const$$
\end{law}
\begin{proof}
    $$A_{\text{пот.}} + A_{\text{непот.}} = \Delta K \Longrightarrow A_{\text{непот.}} = \Delta K + \Delta \Pi =\Delta E \Longrightarrow A_{\text{непот.}}=0\Longleftrightarrow E=\const$$
\end{proof}
\begin{definition}
    Абсолютно упругое соударение -- тела после соударения разлетаются без потерь энергии.\bigskip

    \textit{Выполняются ЗСИ и ЗСЭ.}
\end{definition}
\begin{definition}
    Абсолютно неупругое соударение -- тела после соударения движутся как единое тело.
\end{definition}
\begin{definition}
    Неупругое столкновение -- тела после соударения разлетаются с потерями энергии.\bigskip

    \textit{Выполняется ЗСИ. ЗСЭ принимает вид: $E_1=E_2 + Q$.}
\end{definition}

\subsection{Механические колебания}

\begin{definition}
    Механические колебания -- механическое движение, периодически повторяющееся вблизи положения равновесия.
\end{definition}
\begin{definition}
    Полное колебание -- возвращение в начальную точку с тем же направлением скорости.
\end{definition}
\begin{definition}
    Период колебаний -- время одного колебания.
\end{definition}
\begin{definition}
    Частота -- число колебаний в секунду.
\end{definition}
\begin{definition}
    Циклическая частота -- число колебаний за $2\pi$ секунд.
\end{definition}
\begin{definition}
    Амплитуда -- максимальное отхождение от положения равновесия.
\end{definition}
\begin{definition}
    Гармонические колебания -- колебания, идущие по закону синуса / косинуса.
\end{definition}
\begin{theorem}
    Уравнение незатухающих гармонических колебаний:
    $$\ddot{x}+\omega^2 x=0$$
    $\omega$ [1/с] -- циклическая частота.
\end{theorem}
\begin{theorem}
    Решением уравнения незатухающих гармонических колебаний является:
    $$x(t)=A\sin (\omega t + \varphi_0)$$
    $A$ [м] -- амплитудное значение колебаний, $\omega$ [1/с] -- циклическая частота, $t$ [с] -- время, $\varphi_0$ -- начальная фаза, $\omega t+\varphi_0$ -- фаза колебаний.
\end{theorem}
\setcounter{subsection}{12}
\begin{consequence}
    Для пружинного маятника $\omega=\sqrt{\dfrac{k}{m}}\Longrightarrow T=2\pi\sqrt{\dfrac{m}{k}}$. Для математического маятника $\omega=\sqrt{\dfrac{g}{l}}\Longrightarrow T = 2\pi\sqrt{\dfrac{l}{g}}$.
\end{consequence}
\setcounter{subsection}{5}
\begin{theorem}
    Уравнения фазовых кривых гармонических колебаний:
    $$\left(\frac{x(t)}{x_{\max}}\right)^2+\left(\frac{v(t)}{v_{\max}}\right)^2=1;\,\,\left(\frac{a(t)}{a_{\max}}\right)^2+\left(\frac{v(t)}{v_{\max}}\right)^2=1$$
\end{theorem}

\subsection{Волны}

\begin{definition}
    Механическая волна -- распространение колебаний в упругой среде.
\end{definition}
\begin{definition}
    Поперечная волна -- волна, в которой направление колебаний частицы перпендикулярно направлению распространения.\bigskip

    \textit{Поперечные волны распространяются только в твёрдых телах.}
\end{definition}
\begin{definition}
    Продольная волна -- волна, в которой направление колебаний частицы параллельно направлению распространения.\bigskip

    \textit{Продольные волны распространяются в жидкостях, твёрдых телах и газах.}
\end{definition}
\begin{definition}
    Колебания в одной фазе -- колебания, в которых скорости в каждый момент времени сонаправлены.
\end{definition}
\begin{definition}
    Колебания в разнофазе -- колебания, в которых скорости в каждый момент времени противоположно направлены.
\end{definition}
\begin{definition}
    Колебания со сдвигом фазы -- колебания, в которых скорости в разные моменты времени сонаправлены или противоположно направлены.
\end{definition}
\begin{definition}
    Длина волны -- расстояние между двумя ближайшими точками, колеблющимися в одной фазе.
\end{definition}
\begin{definition}
    Скорость распространения волны -- скорость передачи энергии волной.
\end{definition}
\begin{formula}
    $$\lambda=v\cdot T=\frac{v}{\nu}$$
    \begin{center}
        \centering
        \begin{minipage}{12cm}
            $\lambda$ [м] -- длина волны, $v$ [м/с] -- скорость распространения волны, $T$ [с] -- период колебания волны, $\nu$ [1/с] -- частота колебания волны.
        \end{minipage}
    \end{center}
\end{formula}

\section{Статика}

\begin{law}
    Материальная точка находится в равновесии, если векторная сумма сил, действующих на неё, равна нулевому вектору:
    $$\sum_{i=1}^{n}\vv{F_i}=\vv{0}$$
\end{law}
\begin{definition}
    Плечом силы называется расстояние от линии действия силы до оси вращения тела.
\end{definition}
\begin{definition}
    Момент силы -- векторная физическая величина, характеризующая действие силы на механический объект, которое может вызвать его вращательное движение.
\end{definition}
\begin{formula}
    $$\vv{M}=l\cdot\vv{F}$$
    \begin{center}
        \centering
        \begin{minipage}{12cm}
            $\vv{M}$ [Н $\cdot$ м] -- момент силы, $l$ [м] -- плечо силы, $\vv{F}$ [Н] -- сила.
        \end{minipage}
    \end{center}
\end{formula}
\begin{law}
    Абсолютно твёрдое тело находится в равновесии, если векторная сумма сил, действующих на это тело, и алгебраическая сумма всех моментов этих сил, равны нулю:
    $$\sum_{i=1}^{n}\vv{F_i}=\vv{0};\,\,\sum_{i=1}^{n}M - \sum_{i=1}^m M'=0$$
     $M$ [Н $\cdot$ м] -- момент силы, стремящейся повернуть тело по часовой стрелке, $M'$ [Н $\cdot$ м] -- момент силы, стремящейся повернуть тело против часовой стрелки.
\end{law}

\section{Термодинамика}

\begin{definition}
    Механика изучает макроскопические объекты, состоящие из множества частиц. Термодинамика изучает микроскопические объекты, состоящие из молекул и атомов.
\end{definition}

\subsection{Молекулярная кинетическая теория}

\begin{definition}
    Тепловое движение -- беспорядочное движение частиц.
\end{definition}
\begin{definition}
    Абсолютный нуль -- температура, при которой частицы перестают двигаться ($\approx-273 ^\circ $C).
\end{definition}
\begin{law}[Основные положения МКТ]
    $ $\par\nobreak\ignorespaces
    \begin{multicols}{2}
        \begin{enumerate}[leftmargin=*, itemsep=0pt]
        \item Все тела состоят из мельчайших частиц.
        \item Частицы взаимодействуют друг с другом.
        \item Частицы находятся в беспорядочном движении, если температура превышает абсолютный нуль.
    \end{enumerate}
    \end{multicols}
\end{law}
\begin{definition}
    Относительной молекулярной (атомной) массой называется отношение массы молекулы (атома) к $\frac{1}{12}$ массы изотопа углерода-12:
    $$M_r=\frac{m_0}{\frac{1}{12}m_C}$$
    $M_r$ [а. е. м.] -- относительная молекулярная масса, $m_0$ [кг] -- масса молекулы вещества, $m_C$ [кг] -- масса атома углерода-12.
\end{definition}
\begin{definition}
    1 моль вещества -- количество частиц в 12 г изотопа углерода-12.
\end{definition}
\begin{definition}
    Молярная масса -- масса 1 моля вещества.\bigskip

    \textit{Численно молярная масса равна молекулярной.}
\end{definition}
\begin{formula}
    $$M=m_0\cdot N_A$$
    \begin{center}
        \begin{minipage}{12cm}
            $M$ [кг/моль] -- молярная масса вещества, $m_0$ [кг] -- масса молекулы вещества, $N_A$ -- постоянная Авогадро $\left(\approx \num{6.02}\cdot 10^{23}\text{ моль}^{-1}\right)$.
        \end{minipage}
    \end{center}
\end{formula}

\subsection{Строение атома}
\begin{law}
    $$m_p\approx m_n\approx 10^{-27}\text{ кг}\approx 1\text{ а. е. м.};\,\,m_e\approx 10^{-30} \text{ кг}$$
    \begin{center}
        \begin{minipage}{12cm}
            $m_p$ -- масса протона, $m_n$ -- масса нейтрона, $m_e$ -- масса электрона.
        \end{minipage}
    \end{center}
\end{law}
\begin{law}
    $$\overline{e}=q_p=-q_e\approx \num{1.6}\cdot 10^{-19}\text{ Кл}$$
    \begin{center}
        \begin{minipage}{13cm}
            \centering
            $\overline{e}$ -- элементарный заряд, $q_p$ -- заряд протона, $q_e$ -- заряд электрона.
        \end{minipage}
    \end{center}
\end{law}
\begin{definition}
    Зарядовое число ($Z$) -- количество протонов в атомном ядре, равное порядковому номеру химического элемента в таблице Менделеева.
\end{definition}
\begin{definition}
    $$M=N_p+N_n$$
    \begin{center}
        \begin{minipage}{12cm}
            $M$ -- массовое число (количество нуклонов), $N_p$ -- количество протонов, $N_n$ -- количество нейтронов.
        \end{minipage}
    \end{center}
\end{definition}

\subsection{Идеальный газ}

\begin{definition}
    Идеальный газ -- модель газа, в которой не учитывается взаимодействие молекул (молекулы считаются материальными точками).
\end{definition}
\begin{definition}
    Давление газа -- это сила, которую газ оказывает на стенки его сосуда.\bigskip

    \textit{Давление газа обуславливается ударением молекул газа о стенки сосуда.}
\end{definition}
\begin{definition}
    Концентрация -- отношение общего числа молекул газа к его объёму:
    $$n=\frac{N}{V}$$
    $n$ [м$^{-3}$] -- концентрация, $N$ -- число молекул в газе, $V$ [м$^3$] -- объём газа.
\end{definition}
\begin{definition}
    Средняя квадратичная скорость -- это скорость, равная корню квадратному из средней арифметической величины квадратов скоростей отдельных молекул:
    $$v_{\text{ср. кв.}}=\sqrt{\sum_{i=1}^{N}\frac{v_i^2}{N}}$$
    $v_{\text{ср. кв.}}$ [м/с] -- средняя квадратичная скорость, $v_i$ [м/с] -- скорость $i$-й частицы газа, $N$ -- количество частиц.
\end{definition}
\begin{law}[Основное уравнение МКТ]
    $$p=\frac{2}{3}n \cdot \overline{E}$$
    \begin{center}
        \begin{minipage}{12cm}
            $p$ [Па] -- давление, $n$ [м$^{-3}$] -- концентрация, $\overline{E}$ [Дж] -- средняя кинетическая энергия поступательного движения.
        \end{minipage}
    \end{center}
\end{law}
\begin{proof}
    Пусть имеем сосуд в форме прямоугольного параллелепипеда с площадью поперечного сечения $S$ и перпендикулярным ему ребром длины $l$, заполненный идеальным газом. Рассмотрим внутри него частицу массы $m_0$: если в проекции ребра длины $l$ она движется со скоростью $v_x$, то её импульс до соударения со стенкой сосуда будет равен $m_0v_x$, а после -- $-m_0v_x$, тогда стенке сосуда передастся импульс $p=2m_0v_x$. Время, через которое частица сталкивается с одной стенкой равно $t=\dfrac{2l}{v_x}$. Тогда сила, с которой частица взаимодействует со стенкой сосуда равна $F_x=\dfrac{p}{t}=\dfrac{m_0v_x^2}{l}$. По определению давление равно $p=\dfrac{F}{S}$, откуда:
    $$p_xS=\frac{m_0v_x^2}{l}\Longrightarrow p_x=\frac{m_0v_x^2}{lS}=p_x=\frac{m_0v_x^2}{V}$$
    Значит, для всех частиц давление можно посчитать, используя $\overline{v_x}$ среднюю скорость частицы и $N$ количество частиц. При этом из-за того, что все частицы движутся хаотично, и направления их движения равновероятны:
    $$\overline{v_x}^2=\overline{v_y}^2=\overline{v_z}^2=\frac{1}{3}\overline{v}^2$$
    Отсюда давление всех частиц равно:
    $$p=N\frac{m_0\overline{v}^2}{3V}=\frac{1}{3}nm_0\overline{v}^2=\frac{2}{3}n\overline{E}$$
\end{proof}
\begin{definition}
    макроскопические параметры -- характеристики макроскопической системы.\bigskip

    \textit{Например, для газа в сосуде: давление, температура, объём.}
\end{definition}
\begin{formula}
    $$\overline{E}=\frac{3}{2}K_\text{Б}\cdot T$$
    \begin{center}
        \begin{minipage}{12cm}
            $\overline{E}$ [Дж] -- средняя кинетическая энергия поступательного движения, $K_\text{Б}$ -- постоянная Больцмана $\left(\approx \num{1.38}\cdot 10^{-23}\text{ Дж/К}\right)$, $T$ [К] -- температура.
        \end{minipage}
    \end{center}
\end{formula}
\begin{proof}
    \begin{multline*}
        p=\frac{2}{3}n \cdot \overline{E}=\frac{2N}{3V} \cdot \overline{E}\Longrightarrow \frac{pV}{N}=\frac{2}{3}\overline{E}=\const\,\,(\text{при }T=\const)\Longrightarrow
        \frac{pV}{N}=K_\text{Б} \cdot T\Longrightarrow \overline{E}=\frac{3}{2}K_\text{Б}\cdot T
    \end{multline*}
\end{proof}
\setcounter{subsection}{4}
\begin{consequence}
    Температура -- мера кинетической энергии поступательного движения частиц.
\end{consequence}
\setcounter{subsection}{2}
\begin{definition}
    Тепловое равновесие -- состояние системы, при котором все макроскопические параметры не изменяются на протяжении долгого времени.\bigskip

    \textit{Теплообмен идёт до установления одинаковой температуры у термодинамических систем.}
\end{definition}
\begin{formula}
    $$v_{\text{ср. кв.}}=\sqrt{\frac{3K_\text{Б}\cdot T}{m_0}}$$
    \begin{center}
        \begin{minipage}{12cm}
            $v_{\text{ср. кв.}}$ [м/с] -- средняя квадратичная скорость, $K_\text{Б}$ -- постоянная Больцмана, $T$ [К] -- температура, $m_0$ [кг] -- масса частицы.
        \end{minipage}
    \end{center}
\end{formula}
\begin{proof}
    $$\overline{E}=\frac{m_0v_{\text{ср. кв.}}^2}{2}=\frac{3}{2}K_\text{Б}\cdot T\Longrightarrow v_{\text{ср. кв.}}=\sqrt{\frac{3K_\text{Б}\cdot T}{m_0}}$$
\end{proof}
\begin{formula}[Уравнение Менделеева-Клапейрона]
    $$pV=\nu RT$$
    \begin{center}
        \begin{minipage}{12.1cm}
            $p$ [Па] -- давление, $V$ [м$^3$] -- объём, $\nu$ [моль] -- количество вещества, $R$ -- универсальная газовая постоянная $\left(\approx \num{8.31}\text{ Дж/}(\text{моль}\cdot\text{К})\right)$, $T$ [К] -- температура.
        \end{minipage}
    \end{center}
\end{formula}
\begin{proof}
    $$ \frac{pV}{N}=K_\text{Б} \cdot T\Longrightarrow pV=\nu\cdot N_A\cdot K_\text{Б}\cdot T=\nu RT=\frac{m}{M}R\cdot T$$
\end{proof}

\subsection{Изопроцессы}

\begin{definition}
    Изопроцессы -- процессы, в которых один из параметров состояния не изменяется.
\end{definition}
\begin{law}[Закон Бойля-Мариотта]
    Давление газа в изотермическом процессе обратно пропорционально занимаемому газом объёму:
    $$T=\const\Longrightarrow pV=\const$$
\end{law}
\begin{law}[Закон Гей-Люссака]
    Объём газа в изобарическом процессе пропорционален абсолютной температуре газа:
    $$p=\const\Longrightarrow \frac{V}{T}=\const$$
\end{law}
\begin{law}[Закон Шарля]
    Абсолютная температура газа в изохорном процессе пропорциональна давлению газа:
    $$V=\const \Longrightarrow \frac{T}{P}=\const$$
\end{law}
\setcounter{subsection}{9}
\begin{consequence}[Графики термодинамических изопроцессов]
    $ $\par\nobreak\ignorespaces
    \begin{center}
        \begin{tikzpicture}[scale=0.9]
            \begin{scope}
                \clip (0,0) rectangle (5,4);
                \draw[domain=0.5:5, variable=\x, samples=50] plot ({(\x)},{(2/\x)});
                \draw[domain=0.5:5, variable=\x, samples=50] plot ({(\x)},{5/\x});
            \end{scope}
            \draw[-{Stealth[scale = 1.5]}] (-0.5,0) -- (5.5,0) node[below]{$V$};
            \draw[-{Stealth[scale = 1.5]}] (0,-0.5) -- (0,4.5) node[left]{$p$};
            \draw (0,0) node[below left] {$O$};
            \draw (1.8,1.7) node {$T_2$};
            \draw (3,2.3) node {$T_1$};
            \draw (2.7,-0.5) node[below] {Изотерма $(T_1>T_2)$};
            \draw[-{Stealth[scale = 1.5]}] (6,0) -- (12,0) node[below]{$T$};
            \draw[-{Stealth[scale = 1.5]}] (6.5,-0.5) -- (6.5,4.5) node[left]{$V$};
            \draw (6.5,0) node[below left] {$O$};
            \draw[dashed] (6.5,0) -- (11.5,2);
            \draw (7.5,0.4) -- (11.5,2);
            \draw[dashed] (6.5,0) -- (10,4);
            \draw (7,0.5714) -- (10,4);
            \draw (9.4,1.5) node {$p_1$};
            \draw (8,2.3) node {$p_2$};
            \draw (9.2,-0.5) node[below] {Изобара $(p_1>p_2)$};
            \draw[-{Stealth[scale = 1.5]}] (12.5,0) -- (18.5,0) node[below]{$T$};
            \draw[-{Stealth[scale = 1.5]}] (13,-0.5) -- (13,4.5) node[left]{$p$};
            \draw (13,0) node[below left] {$O$};
            \draw[dashed] (13,0) -- (18,2);
            \draw (14,0.4) -- (18,2);
            \draw[dashed] (13,0) -- (16.5,4);
            \draw (13.5,0.5714) -- (16.5,4);
            \draw (15.9,1.5) node {$V_1$};
            \draw (14.5,2.3) node {$V_2$};
            \draw (15.7,-0.5) node[below] {Изохора $(V_1>V_2)$};
            \foreach \point in {(0,0), (6.5,0), (13,0)}{
        \fill \point circle (1.8pt);
    }
        \end{tikzpicture}
    \end{center}
\end{consequence}
\setcounter{subsection}{3}

\subsection{Внутренняя энергия идеального газа}

\begin{definition}
    Внутренняя энергия идеального газа является кинетической энергией движения частиц.\bigskip

    \textit{Внутренняя энергия ИГ зависит от температуры. Движение состоит из поступательного и вращательного.}
\end{definition}
\begin{definition}
    Число степеней свободы -- минимальное необходимой число осей для полного описания движения частицы.
\end{definition}
\begin{formula}
    $$U=\frac{i}{2} \nu RT$$
    \begin{center}
        \begin{minipage}{12cm}
            $U$ [Дж] -- внутренняя энергия ИГ, $i$ -- число степеней свободы, $\nu$ [моль] -- количество вещества, $R$ -- универсальная газовая постоянная, $T$ [К] -- температура.\bigskip

            \textit{Для 1-ат молекулы: $i=3$, для 2-ат: $i=5$, для 3-ат: $i=6$ ($i=5$ в случае линейной молекулы.)}
        \end{minipage}
    \end{center}
\end{formula}
\begin{formula}[Первое начало термодинамики]
    $$Q=\Delta U + A_\text{газа}$$
    \begin{center}
        \begin{minipage}{12cm}
            $Q$ [Дж] -- полученное количество теплоты, $U$ [Дж] -- внутренняя энергия ИГ, $A_\text{газа}$ [Дж] -- работа газа.\bigskip

            \textit{Работа газа равна площади графика процесса на $pV$-диаграмме.}
        \end{minipage}
    \end{center}
\end{formula}
\setcounter{subsection}{11}
\begin{consequence}[Изотермический процесс]
    $$T=\const\Longrightarrow \Delta U=0\Longrightarrow Q=A_\text{газа}$$
\end{consequence}
\begin{consequence}[Изохорный процесс]
    $$V=\const\Longrightarrow A_\text{газа}=0\Longrightarrow Q=\Delta U$$
\end{consequence}
\begin{consequence}[Изобарный процесс]
    $$p=\const\Longrightarrow A_\text{газа}=p\Delta V;\,\,\Delta U=\frac{i}{2}\nu R\Delta T\Longrightarrow Q=\frac{i}{2}\nu R\Delta T + p\Delta V=\frac{i+2}{2}\nu R\Delta T=\frac{i+2}{2}p\Delta V$$
\end{consequence}
\begin{consequence}[Адиабатный процесс]
    $$Q=0\Longrightarrow A_\text{газа}=-\Delta U$$
\end{consequence}
\setcounter{subsection}{4}

\subsection{Циклы. Тепловые машины}

\begin{definition}
    Тепловая машина состоит из нагревателя, рабочего тела и холодильника. 
\end{definition}
\begin{formula}
    $$Q_\text{н}=Q_\text{х}+A_\text{мех}$$
    \begin{center}
        \begin{minipage}{12cm}
            $Q_\text{н}$ [Дж] -- теплота, отданная нагревателем, $Q_\text{х}$ [Дж] -- теплота, полученная холодильником, $A_\text{мех}$ [Дж] -- механическая работа тепловой машины.
        \end{minipage}
    \end{center}
\end{formula}
\begin{formula}
    $$\eta=1-\frac{Q_\text{х}}{Q_\text{н}}$$
    \begin{center}
        \begin{minipage}{12cm}
            $\eta$ -- КПД тепловой машины, $Q_\text{н}$ [Дж] -- теплота, отданная нагревателем, $Q_\text{х}$ [Дж] -- теплота, полученная холодильником.
        \end{minipage}
    \end{center}
\end{formula}
\begin{proof}
    $$\eta=\frac{A_\text{полезная}}{A_\text{затраченная}}=\frac{A_\text{мех}}{Q_\text{н}}=\frac{Q_\text{н}-Q_\text{х}}{Q_\text{н}}=1-\frac{Q_\text{х}}{Q_\text{н}}$$
\end{proof}
\begin{definition}
    Цикл -- процесс, в результате которого система возвращается в исходное состояние.\bigskip

    \textit{Циклы бывают прямые (по часовой стрелке, $A>0$) и обратные (против часовой стрелки, $A<0$). Работа газа за цикл равна площади фигуры, образованной графиком цикла на $pV$-диаграмме.}
\end{definition}
\begin{definition}
    Цикл Карно -- идеальный круговой процесс. КПД тепловой машины Карно равен:
    $$\eta=1-\frac{T_\text{x}}{T_\text{н}}$$
    \begin{center}
        \begin{tikzpicture}
            \begin{scope}
                \clip (0,0) rectangle (6,4);
                \draw[domain=0.1:6, variable=\x, samples=100] plot ({(\x)},{(1/\x)});
                \draw[domain=0.1:6, variable=\x, samples=100] plot ({(\x)},{6/\x});
            \end{scope}
            \begin{scope}
                \clip (0,0) rectangle (6,4);
                \draw[domain=1.8986:2.7169, variable=\x, samples=100] plot ({(\x)},{(2.3/\x)^(6)});
                \draw[domain=3.6884:5.278, variable=\x, samples=100] plot ({(\x)},{(4/\x)^(6)});
            \end{scope}
            \coordinate (O) at (0,0);
            \coordinate (A) at (1.8986,3.16);
            \coordinate (B) at (3.6884,1.6267);
            \coordinate (C) at (5.278,0.189);
            \coordinate (D) at (2.7169,0.368);
            \draw[-{Stealth[scale = 1.5]}] (-0.5,0) -- (6.5,0) node[below]{$V$};
            \draw[-{Stealth[scale = 1.5]}] (0,-0.5) -- (0,4.5) node[left]{$p$};
            \draw (O) node[below left] {$O$};
            \draw (A) node[above right] {1};
            \draw (B) node[above right] {2};
            \draw (C) node[above right] {3};
            \draw (D) node[above right] {4};
            \node[align=left] at (8.5,2.3){
                1-2; 3-4 -- изотермы\\
                2-3; 4-1 -- адиабаты
            };
            \foreach \point in {(O), (A), (B), (C), (D)}{
    \fill \point circle (1.8pt);
}
        \end{tikzpicture}
    \end{center}
\end{definition}

\subsection{Влажный воздух}

\begin{definition}
    Динамическое равновесие -- состояние системы (жидкость + пар), при котором число вылетающих молекул равно числу возвращающихся.
\end{definition}
\begin{definition}
    Насыщенный пар -- пар, находящийся в динамическом равновесии со своей жидкостью.\bigskip

    \textit{Ненасыщенный пар -- это ИГ с 6 степенями свободы. Давление насыщенного пара зависит от температуры газа.}
\end{definition}
\begin{definition}
    Кипение -- парообразование по всей поверхности жидкости.\bigskip

    \textit{Кипение происходит, когда давление насыщенного пара уравнивается с атмосферным.}
\end{definition}
\begin{definition}
    Абсолютная влажность воздуха -- содержание водяного пара в воздухе на единицу объёма.
    $$\rho=\frac{m_\text{пара}}{V_\text{воздуха}}$$
    $\rho$ [кг/м$^3$] -- абсолютная влажность воздуха, $m_\text{пара}$ [кг] -- содержание пара в воздухе, $V_\text{воздуха}$ [м$^3$] -- объём воздуха. 
\end{definition}
\begin{definition}
    Относительная влажность воздуха -- отношение парциального давления водяного пара, содержащегося в воздухе, к давлению насыщенного пара при той же температуре.
    $$\varphi=\frac{p_\text{вод.пар}}{p_\text{нас.пар}}\cdot 100\%$$
    $\varphi$ -- относительная влажность воздуха, $p_\text{вод.пар}$ [Па] -- парциальное давление водяного пара, $_\text{нас.пар}$ [Па] -- парциальное давление насыщенного пара.
\end{definition}
\begin{law}
    Влажный воздух состоит из сухого воздуха (\ce{N2 + O2 + CO2} + примеси) и \ce{H2O} водяного пара.
\end{law}
\begin{law}[Дальтона]
    $$p_{\text{вв}}=p_{\text{cв}}+p_{\text{вп}}$$
    \begin{center}
        \begin{minipage}{12cm}
            \centering
            $p_{\text{вв}}$ [Па] – давление влажного воздуха, $p_{\text{cв}}$ [Па] -- давление сухого воздуха, $p_{\text{вп}}$ [Па] – давление водяного пара.
        \end{minipage}
    \end{center}
\end{law}
\end{document}