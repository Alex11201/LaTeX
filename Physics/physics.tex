%Формат файла
\documentclass[12pt]{article} 
\usepackage[paperheight=297mm,
   paperwidth=210mm,
   top=20mm,
   bottom=20mm,
   left=15mm,
   right=15mm]{geometry}


%Текст
\usepackage[fontsize=12pt]{fontsize}
\usepackage[russian]{babel}
\usepackage{color}
\usepackage{transparent}
\usepackage{amsthm}
\usepackage{setspace}
\parindent=0cm

\theoremstyle{definition}
\newtheorem{theorem}{Теорема}[section]
\newtheorem{lemma}[theorem]{Лемма}
\newtheorem{definition}{Определение}
\newtheorem{law}[theorem]{Закон}
\newtheorem{formula}[theorem]{Формула}
\newtheorem{statement}[theorem]{Утверждение}
\newtheorem{consequence}{Следствие}[subsection]
\renewcommand\qedsymbol{$\blacksquare$}

%Картинки
\usepackage{graphicx}
\usepackage{wrapfig}
\usepackage{subcaption}
\usepackage{tikz}
\usepackage{tkz-euclide}
\usetikzlibrary {arrows.meta}
\usetikzlibrary{calc}
\usetikzlibrary{intersections}
\usepackage[european,siunitx]{circuitikz}


%Математика
\usepackage{amsmath}
\usepackage{amsfonts}
\usepackage{mathabx}
\usepackage{amssymb}

%Всякое
\usepackage{relsize}
\usepackage{enumerate}
\usepackage[inline]{enumitem}
\usepackage{hyperref}
\usepackage{multirow}
\usepackage{booktabs}
\usepackage{physics}

%Мат команды
\newcommand{\N}{\mathbb{N}}
\newcommand{\Z}{\mathbb{Z}}
\newcommand{\Q}{\mathbb{Q}}
\newcommand{\R}{\mathbb{R}}
\newcommand{\prob}{\mathbb{P}}
\newcommand{\verteq}{\rotatebox{90}{$\,=$}}
\newcommand{\equalto}[2]{\underset{\scriptstyle\overset{\mkern4mu\verteq}{#2}}{#1}}
\newcommand{\vertneq}{\rotatebox{90}{$\,\neq$}}
\newcommand{\notequalto}[2]{\underset{\scriptstyle\overset{\mkern4mu\vertneq}{#2}}{#1}}
\makeatletter
\newenvironment{sqcases}{%
  \matrix@check\sqcases\env@sqcases
}{%
  \endarray\right.%
}
\def\env@sqcases{%
  \let\@ifnextchar\new@ifnextchar
  \left\lbrack
  \def\arraystretch{1.2}%
  \array{@{}l@{\quad}l@{}}%
}
\makeatother

%Оглавление
\title{\textbf{Физика}}\date{}

\hypersetup{
    colorlinks,
    citecolor=black,
    filecolor=black,
    linkcolor=black,
    urlcolor=black
}

\begin{document}

\maketitle
\tableofcontents
\label{toc}
\newpage

\section{Теплопередача.}

\begin{definition}
    Теплота – кинетическая часть внутренней энергии вещества, определяемая интенсивным хаотическим движением молекул и атомов, из которых это вещество состоит.
\end{definition}
\begin{definition}
    Количество теплоты – часть внутренней энергии, которую тело получает или теряет при теплопередаче.
\end{definition}
\begin{definition}
    Теплопередача – физический процесс передачи тепловой энергии от более горячего тела к более холодному.
\end{definition}

\begin{center}
    \begin{tikzpicture}
        \draw[white] (-7,0) -- (7,0);
        \draw (0,0) node[above] {\textbf{Теплопередача}};
        \draw[-{Stealth[scale = 1.5]}] (-0.5,0) -- (-0.5,-0.5) -- (-5,-0.5) -- (-5,-1);
        \draw[-{Stealth[scale = 1.5]}] (0.5,0) -- (0.5,-0.5) -- (5,-0.5) -- (5,-1);
        \draw[-{Stealth[scale = 1.5]}] (0,0) -- (0,-1);
        \draw (-5,-1) node[below] {Излучение};
        \draw (0,-1) node[below] {Конвекция};
        \draw (5,-1) node[below] {Теплопроводность};
        \draw[-{Stealth[scale = 1.5]}] (-0.25,-1.7) -- (-0.25,-2.2) -- (-3,-2.2) -- (-3,-2.7);
        \draw[-{Stealth[scale = 1.5]}] (0.25,-1.7) -- (0.25,-2.2) -- (3,-2.2) -- (3,-2.7);
        \draw (-3,-2.7) node[below] {Естественная};
        \draw (3,-2.7) node[below] {Вынужденная};
    \end{tikzpicture}
\end{center}

\begin{definition}
    Излучение — вид теплопередачи, при котором происходит передача внутренней энергии c помощью энергии электромагнитных волн.
\end{definition}
\begin{definition}
    Конвекция — вид теплопередачи, обусловленный потоками жидкости или газа.
\end{definition}
\begin{definition}
    Теплопроводность — передача внутренней энергии от одной части тела к другой или от одного тела к другому при контакте.
\end{definition}

\begin{formula}
    $$Q=c\cdot m\cdot \Delta t\text{, где }Q\text{ [Дж] – количество теплоты,}$$
$$c\,\left[\frac{\text{Дж}}{\text{кг}\cdot {^\circ C}}\right]\text{ – удельная теплоемкость, }m\text{ [кг] – масса вещества, }\Delta t\,[^\circ C]\text{ – разность температур.}$$
\end{formula}

\subsection{Агрегатное состояние}

\begin{definition}
    Агрегатное состояние вещества — физическое состояние вещества, зависящее от соответствующего сочетания температуры и давления.
\end{definition}
\begin{definition}
    Переход вещества из жидкого состояния в твердое называется кристаллизацией.
\end{definition}
\begin{definition}
    Переход вещества из жидкого состояния в газообразное называется парообразованием.
\end{definition}
\begin{definition}
    Переход вещества из твердого состояния в жидкое называется плавлением.
\end{definition}
\begin{definition}
    Переход вещества из твердого состояния в газообразное называется сублимацией.
\end{definition}
\begin{definition}
    Переход вещества из газообразного состояния в жидкое называется конденсацией.
\end{definition}
\begin{definition}
    Переход вещества из газообразного состояния в твердое называется десублимацией.
\end{definition}
\begin{definition}
    Насыщенный пар — пар, находящийся в динамическом равновесии со своей жидкостью.
\end{definition}
\begin{center}
    \begin{tikzpicture}
        \draw[-{Stealth[scale = 1.5]}] (-0.5,0) -- (11,0) node[right] {Время [мин]};
        \draw[-{Stealth[scale = 1.5]}] (0, -0.5) -- (0,7) node[above] {Температура [$^\circ C$]};
        \coordinate (O) at (0,0);
        \coordinate (t0) at (0,1);
        \coordinate (t1) at (0,4);
        \coordinate (t2) at (0,6);
        \draw (O) node[below left] {$O$};
        \draw (t0) node[left] {$t_0$};
        \draw (t1) node[left] {$t_1$};
        \draw (t2) node[left] {$t_2$};
        \coordinate (A) at (2,4);
        \coordinate (B) at (4,4);
        \coordinate (C) at (5,6);
        \coordinate (D) at (6,4);
        \coordinate (E) at (8,4);
        \coordinate (F) at (10,1);
        \tkzLabelSegment[above left](t0,A){$a$};
        \tkzLabelSegment[above](A,B){$b$};
        \tkzLabelSegment[above left](B,C){$c$};
        \tkzLabelSegment[above right](C,D){$d$};
        \tkzLabelSegment[above](D,E){$e$};
        \tkzLabelSegment[above right](E,F){$f$};
        \node[right, align=left] at (10,4.5) {
              $a$ -- нагрев\\
              $b$ -- кипение\\
              $c$ -- нагрев пара\\
              $d$ -- охлаждение пара\\
              $e$ -- конденсация\\
              $f$ -- охлаждение жидкости
          };
        \draw (t0) -- (A) -- (B) -- (C) -- (D) -- (E) -- (F);
        \foreach \point in {(O), (t0), (t1), (t2), (A), (B), (C), (D), (E), (F)}{
    \fill \point circle (1.8pt);
}
    \end{tikzpicture}
\end{center}

\subsection{Удельная теплота}

\begin{definition}
    Удельная теплота -- физическая величина, показывающая, какое количество теплоты необходимо, чтобы преобразовать количество вещества с единичной массой, при данной температуре в ходе какого-либо процесса.
\end{definition}
\begin{formula}
    $$Q=q\cdot m\text{, где }Q\text{ [Дж] – количество теплоты,}$$
$$q\text{ [Дж/кг] -- удельная теплота сгорания, }m\text{ [кг] – масса вещества.}$$
\end{formula}
\begin{formula}
    $$Q=L\cdot m\text{, где }Q\text{ [Дж] – количество теплоты,}$$
$$L\text{ [Дж/кг] -- удельная теплота парообразования, }m\text{ [кг] – масса вещества.}$$
\end{formula}
\begin{formula}
    $$Q=\lambda\cdot m\text{, где }Q\text{ [Дж] – количество теплоты,}$$
$$\lambda\text{ [Дж/кг] -- удельная теплота плавления, }m\text{ [кг] – масса вещества.}$$
\end{formula}

\section{Электрический ток}

\begin{definition}
    Электрический ток -- упорядоченное движение заряженных частиц. Направление электрического тока определяется движением положительных зарядов.
\end{definition}
\begin{law}[Ома]
    $$I=\frac{U}{R}\text{, где }I\text{ [А] – сила тока,}$$
$$U\text{ [В] -- напряжение, }R\text{ [Ом] – сопротивление.}$$
\end{law}

\subsection{Электрическое поле}

\begin{definition}
    Поле -- материальная среда, передающая воздействие тел друг на друга.
\end{definition}
\begin{definition}
    Электростатическое поле -- поле, передающее взаимодействие одного неподвижного электрического заряда на другой.
\end{definition}
\begin{definition}
    Электрическая сила – сила, с которой электрическое поле одного заряда действует на внесенный в него другой электрический заряд. Сила воздействия электрического поля на заряд уменьшается по мере удаления.
\end{definition}

\subsection{Источник тока}

\begin{definition}
    Источник тока -- устройство, в котором происходит преобразование какого-либо вида энергии в электрическую энергию.
\end{definition}

\begin{center}
    \begin{tikzpicture}
        \draw[white] (-8,0) -- (8,0);
        \draw (0,0) node[above] {\textbf{Источник тока}};
        \draw[-{Stealth[scale = 1.5]}] (-0.75,0) -- (-0.75,-0.5) -- (-6,-0.5) -- (-6,-1) node[below] {Механический};
        \draw[-{Stealth[scale = 1.5]}] (-0.25,0) -- (-0.25,-1) -- (-3,-1) -- (-3,-1.5) node[below] {Тепловой};
        \draw[-{Stealth[scale = 1.5]}] (0.25,0) -- (0.25,-1) -- (3,-1) -- (3,-1.5) node[below] {Световой};
        \draw[-{Stealth[scale = 1.5]}] (0.75,0) -- (0.75,-0.5) -- (6,-0.5) -- (6,-1) node[below] {Химический};
    \end{tikzpicture}
\end{center}

\subsection{Проводники}

\begin{definition}
    Проводник — вещество, среда, материал, хорошо проводящие электрический ток вследствие наличия свободных носителей заряда.
\end{definition}

\begin{formula}
    $$A=U\cdot I\cdot t\text{, где }A\text{ [Дж] – работа тока,}$$
$$U\text{ [В] -- напряжение, }I\text{ [А] – сила тока, }t\text{ [с] -- время.}$$
\end{formula}
\begin{formula}
    $$P=U\cdot I \text{, где }P\text{ [Вт] – мощность тока,}$$
$$U\text{ [В] -- напряжение, }I\text{ [А] – сила тока.}$$
\end{formula}
\begin{law}[Джоуля Ленца]
    $$Q=I^2\cdot R\cdot t\text{, где }Q\text{ [Дж] – количество теплоты,}$$
$$I\text{ [А] -- сила тока, }R\text{ [Ом] – сопротивление, }t\text{ [с] -- время.}$$
\end{law}
\newsavebox{\circuita}
\sbox{\circuita}{
    \begin{circuitikz}
        \draw[white] (-8,0) -- (8,0);
        \draw (-7.5,0) to[R=$R_1$, o-*] (-5,0)
        to[R=$R_2$, *-o] (-2.5,0);
        \draw (2.5,0) to[short, o-*] (3.75,0) -- (3.75,0.5)
        to[R=$R_1$] (6.25,0.5) -- (6.25,0)
        to[short, *-o] (7.5, 0);
        \draw (3.75,0) -- (3.75,-0.5)
        to[R=$R_2$] (6.25,-0.5) -- (6.25,0);
    \end{circuitikz}
}
\spacing{1.7}
\begin{center}
    \begin{tikzpicture}
        \draw[white] (-8,0) -- (8,0);
        \draw (0,0) node[above] {\textbf{Соединение проводников}};
        \draw[-{Stealth[scale = 1.5]}] (-0.25,0) -- (-0.25,-0.5) -- (-5,-0.5) -- (-5,-1);
        \draw[-{Stealth[scale = 1.5]}] (0.25,0) -- (0.25,-0.5) -- (5,-0.5) -- (5,-1);
        \draw (-5,-1) node[below] {Последовательное};
        \draw (5,-1) node[below] {Параллельное};
        \node at (0,-2.75){\usebox{\circuita}};
        \node[align=center] at (-5,-5.5) {
            $I=I_1=I_2$\\
            $U=U_1+U_2$\\
            $R=R_1+R_2$
        };
        \node[align=center] at (5,-5.5) {
            $I=I_1+I_2$\\
            $U=U_1=U_2$\\
            $\dfrac{1}{R}=\dfrac{1}{R_1}+\dfrac{1}{R_2}$
        };
    \end{tikzpicture}
\end{center}
\singlespacing

\begin{formula}
    $$R=\frac{\rho\cdot l}{S}\text{, где }R\text{ [Ом] – сопротивление, }\rho\text{ [Ом}\cdot \text{м] -- удельное сопротивление,}$$
$$l\text{ [м] – длина проводника, }S\text{ [м}^2\text{] -- площадь поперечного сечения проводника.}$$
\end{formula}

\subsection{Напряженность электрического поля}

\begin{definition}
    Напряженность -- отношение силы, с которой поле воздействует на точечный заряд к величине этого заряда.
\end{definition}
\begin{formula}
    $$\vec{E}=\frac{\vec{F}}{q}\text{, где }\vec{E}\text{ [Н/Кл] – напряженность поля,}$$
$$\vec{F}\text{ [Н] -- сила воздействия поля, }q\text{ [Кл] – точечный заряд.}$$
\end{formula}
\begin{law}[Принцип суперпозиции]
    Если в данной точке пространства электрическое поле создано несколькими зарядами и напряженность поля каждого заряда равна $\vec{E_1},\,\vec{E_2},\ldots$, то результирующая напряженность этого поля равна векторной сумме напряженностей составляющих его полей.
\end{law}
\begin{law}[Кулона]
    $$\vec{F}=k\cdot \frac{|q_1|\cdot|q_2|}{\varepsilon\cdot r^2}\text{, где }\vec{F}\text{ [Н] -- сила взаимодействия зарядов, }k=9\cdot 10^9\,\frac{\text{Н}\cdot\text{м}^2}{\text{Кл}^2},$$
    $$q_1\text{ и }q_2\text{ [Кл] -- точечные заряды тел, }r\text{ [м] -- расстояние между зарядами,}$$
    $$\varepsilon\text{ -- относительная диэлектрическая проницаемость среды (равна 1 для воздуха).}$$
\end{law}

\subsection{Конденсаторы}

\begin{definition}
    Электроемкость -- физическая величина, характеризующая способность проводников накапливать заряд.
\end{definition}
\begin{formula}
    $$C=\frac{q}{U}\text{, где }C\text{ [Ф] – электроемкость,}$$
$$q\text{ [Кл] -- заряд пластины конденсатора, }U\text{ [В] – напряжение.}$$
\end{formula}
\begin{formula}
    $$C=\frac{\varepsilon\cdot\varepsilon_0\cdot S}{d}\text{, где }C\text{ [Ф] -- электроемкость, }\varepsilon\text{ -- диэлектрическая проницаемость,}$$
    $$\varepsilon_0\text{ -- электрическая постоянная, }d\text{ [м] -- расстояние между пластинами конденсатора.}$$
\end{formula}
\begin{formula}
    $$W=\dfrac{q\cdot U}{2}=\dfrac{C\cdot U^2}{2}=\dfrac{q^2}{2C}\text{, где }W\text{ [Дж] -- энергия заряженного конденсатора,}$$
    $$q\text{ [Кл] -- заряд пластины конденсатора, }U\text{ [В] -- разность потенциалов, }C\text{ [Ф] -- электроемкость.}$$
\end{formula}

\begin{center}
    \begin{circuitikz}
        \draw (-7.5,0) to[capacitor, o-*] (-5,0)
        to[capacitor, *-o] (-2.5,0);
        \draw (2.5,0) to[short, o-*] (3.75,0) -- (3.75,0.75)
        to[capacitor] (6.25,0.75) -- (6.25,0)
        to[short, *-o] (7.5, 0);
        \draw (3.75,0) -- (3.75,-0.75)
        to[capacitor] (6.25,-0.75) -- (6.25,0);
        \node[align=center] at (-5,-2.25) {
            $\dfrac{1}{C}=\dfrac{1}{C_1}+\dfrac{1}{C_2}$
        };
        \node[align=center] at (5,-2.25) {
            $C=C_1+C_2$
        };
    \end{circuitikz}
\end{center}

\subsection{Магнитное поле}

\begin{definition}
    Магнитное поле -- особый вид материи, существующий вокруг любого проводника с током. Неподвижные электрические заряды создают электрическое поле, а подвижные -- электрическое и магнитное поля.
\end{definition}
\begin{definition}
    Магнитные линии магнитного поля -- замкнутые прямые, охватывающие проводник.
\end{definition}
\begin{definition}
    Магнитная индукция -- силовая характеристика магнитного поля.
\end{definition}
\begin{center}
    \begin{tikzpicture}
        \draw[white] (-8,0) -- (8,0);
        \draw (0,0) node[above] {\textbf{Электромагнитная сила}};
        \draw[-{Stealth[scale = 1.5]}] (-0.25,0) -- (-0.25,-0.5) -- (-5,-0.5) -- (-5,-1);
        \draw[-{Stealth[scale = 1.5]}] (0.25,0) -- (0.25,-0.5) -- (5,-0.5) -- (5,-1);
        \draw (-5,-1) node[below] {Сила Лоренца};
        \draw (5,-1) node[below] {Сила Ампера};
    \end{tikzpicture}
\end{center}
\begin{definition}
    Сила Лоренца -- сила, с которой магнитное поле действует на движущуюся заряженную частицу.
\end{definition}

\begin{formula}
    $$\vec{F}_\text{Л}=|q|\cdot \vec{B}\cdot \vec{v}\cdot \sin \alpha\text{, где }\vec{F}_\text{Л}\text{ [Н] -- сила Лоренца, }q\text{ [Кл] -- заряд частицы,}$$
    $$\vec{B}\text{ [Тл] -- вектор магнитной индукции, }\vec{v}\text{ [м/с] -- скорость частицы, }\alpha\text{ -- угол между }\vec{v}\text{ и }\vec{B}.$$
\end{formula}
\begin{definition}
    Сила Ампера -- сила, с которой магнитное поле воздействует на проводник с током.
\end{definition}
\begin{formula}
    $$\vec{F}_\text{А}=I\cdot \vec{B}\cdot l\cdot \sin \alpha\text{, где }\vec{F}_\text{А}\text{ [Н] -- сила Ампера, }I\text{ [А] -- сила тока,}$$
    $$\vec{B}\text{ [Тл] -- вектор магнитной индукции, }l\text{ [м] -- длина проводника,}$$
    $$\alpha\text{ -- угол между проводником и линиями магнитной индукции.}$$
\end{formula}
\begin{law}[Правило правой руки]
    Если обхватить проводник правой рукой так, чтобы оттопыренный большой палец указывал направление тока, то остальные пальцы покажут направление огибающих проводник линий магнитной индукции поля, создаваемого этим током, а значит и направление вектора магнитной индукции, направленного везде по касательной к этим линиям.\bigskip

    \textit{Иными словами, если ток направлен от наблюдателя, линии магнитной индукции направлены по часовой стрелке.}
\end{law}
\begin{law}[Правило левой руки]
    Если расположить ладонь левой руки так, чтобы линии индукции магнитного поля входили во внутреннюю сторону ладони, перпендикулярно к ней, а четыре пальца направлены по току, то отставленный на 90° большой палец укажет направление силы, действующей со стороны магнитного поля на проводник с током.
\end{law}

\section{Оптика}

\begin{definition}
    Свет — электромагнитное излучение, воспринимаемое человеческим глазом.
\end{definition}
\begin{center}
    \begin{tikzpicture}
        \draw[white] (-8,0) -- (8,0);
        \draw (0,0) node[above] {\textbf{Источники света}};
        \draw[-{Stealth[scale = 1.5]}] (-0.25,0) -- (-0.25,-0.5) -- (-5,-0.5) -- (-5,-1);
        \draw[-{Stealth[scale = 1.5]}] (0.25,0) -- (0.25,-0.5) -- (5,-0.5) -- (5,-1);
        \draw (-5,-1) node[below] {Естественные};
        \draw (5,-1) node[below] {Исскуственные};
    \end{tikzpicture}
\end{center}
\begin{definition}
    Точечный источник света -- источник света, размерами которого можно пренебречь.
\end{definition}
\begin{definition}
    Световой луч -- линия, вдоль которой рассматривается свет.
\end{definition}
\begin{definition}
    Тень -- область пространства, куда не попадает свет.
\end{definition}
\begin{definition}
    Полутень -- слабоосвещенное пространство.
\end{definition}
\begin{definition}
    Плоское зеркало -- плоская поверхность, отражающая свет.
\end{definition}
\begin{definition}
    Абсолютный показатель преломления -- отношение скорости света в веществе к скорости света в вакууме.
\end{definition}
\begin{definition}
    Относительный показатель преломления -- отношение абсолютных показателей преломления двух сред.
\end{definition}
\begin{center}
    \begin{tikzpicture}
        \coordinate (O) at (0,0);
        \coordinate (A) at (-2.5,2);
        \coordinate (B) at (0,3);
        \coordinate (C) at (0,-3);
        \coordinate (D) at (1.5,-3);
        \coordinate (F) at (4,0);
        \draw (-4,0) -- (4,0);
        \draw (0,-3) -- (0,3);
        \draw (A) -- (O);
        \draw (D) -- (O);
        \fill (O) circle (1.8pt);
        \tkzMarkAngle[size=.65cm](B,O,A);
        \tkzLabelAngle(B,O,A){$\alpha$};
        \tkzMarkAngle[size=.65cm](C,O,D);
        \tkzMarkAngle[size=.55cm](C,O,D);
        \tkzLabelAngle(C,O,D){$\beta$};
        \tkzMarkRightAngle(F,O,B);
        \draw (-4,0) node[above right] {$n_1$};
        \draw (-4,0) node[below right] {$n_2$};
        \node[right, align=left] at (5,.75) {
              $\alpha$ -- угол падения\\
              $\beta$ -- угол преломления\\
              $n$ -- коэффициент преломления
        };
        \draw (6.25,-1) node[right] {$\dfrac{\sin \alpha}{\sin \beta}=n_{21}=\dfrac{n_2}{n_1}$};
    \end{tikzpicture}
\end{center}
\begin{definition}
    Действительное изображение -- изображение, находящееся на пересечении лучей, выходящих из источника света.
\end{definition}
\begin{definition}
    Мнимое изображение -- изображение, находящееся на пересечении продолжений лучей.
\end{definition}
\begin{formula}
    $$n=\left[\frac{360^\circ-\alpha}{\alpha}\right]\text{, где }n\text{ -- количество отражений, }\alpha\text{ -- угол падения.}$$
\end{formula}

\subsection{Линзы}

\begin{center}
    \begin{tikzpicture}
        \draw[white] (-8.7,0) -- (8.7,0);
        \def\uplen{4.25}
        \def\downlen{2}
        \def\step{0.5}
        \def\texth{0.7}
        \draw (0,0) node[above] {\textbf{Линзы}};

        \draw[-{Stealth[scale = 1.5]}] (-\step / 2,0) -- (-\step / 2,-\step) -- (-\step - \uplen,-\step) -- (-\step-\uplen,-\step * 2);

        \draw[-{Stealth[scale = 1.5]}] (\step / 2,0) -- (\step / 2,-\step) -- (\step + \uplen,-\step) -- (\step+\uplen,-\step * 2);

        \draw (-\step-\uplen,-\step * 2) node[below] {Собирающие};
        \draw (\step+\uplen,-\step * 2) node[below] {Рассеивающие};

        \draw[-{Stealth[scale = 1.5]}] (-\step-\uplen-\step,-\step*2 - \texth) -- (-\step-\uplen-\step,-\step*3 - \texth) -- (-2*\step-\uplen-\downlen,-\step*3 - \texth) -- (-2*\step-\uplen-\downlen,-\step*4 - \texth);

        \draw[-{Stealth[scale = 1.5]}] (\step-\uplen-\step,-\step*2 - \texth) -- (\step-\uplen-\step,-\step*3 - \texth) -- (-\uplen+\downlen,-\step*3 - \texth) -- (-\uplen+\downlen,-\step*4 - \texth);

        \draw[-{Stealth[scale = 1.5]}] (-\uplen-\step,-\step*2 - \texth) -- (-\uplen-\step,-\step*4 - \texth * 2);

        \draw[-{Stealth[scale = 1.5]}] (-\step+\uplen+\step,-\step*2 - \texth) -- (-\step+\uplen+\step,-\step*3 - \texth) -- (\uplen-\downlen,-\step*3 - \texth) -- (\uplen-\downlen,-\step*4 - \texth);

        \draw[-{Stealth[scale = 1.5]}] (\step+\uplen+\step,-\step*2 - \texth) -- (\step+\uplen+\step,-\step*3 - \texth) -- (\uplen+\downlen+2*\step,-\step*3 - \texth) -- (\uplen+\downlen+2*\step,-\step*4 - \texth);

        \draw[-{Stealth[scale = 1.5]}] (\uplen+\step,-\step*2 - \texth) -- (\uplen+\step,-\step*4 - \texth * 2);

        \draw (-2*\step-\uplen-\downlen,-\step*4 - \texth) node[below] {Двояковыпуклые};
        \draw (-\uplen-\step,-\step*4 - \texth * 2) node[below] {Выпуклые};
        \draw (-\uplen+\downlen,-\step*4 - \texth) node[below] {Выпукло-вогнутые};

        \draw (\uplen-\downlen,-\step*4 - \texth) node[below] {Вогнуто-выпуклые};
        \draw (\uplen+\downlen+2*\step,-\step*4 - \texth) node[below] {Двояковогнутые};
        \draw (\uplen+\step,-\step*4 - \texth * 2) node[below] {Вогнутые};
    \end{tikzpicture}
\end{center}

\end{document}